\chapter{Fundamental Structures}
\thispagestyle{empty}
\label{kap1}
%%%
\section{Algebras and Structures}
\label{kap1-1}
%
%
%
In this section we shall provide definitions of basic terms
and structures which we shall need throughout this book.
Among them are the notions of {\it algebra\/} and {\it structure}.
Readers for whom these are entirely new are advised
to read this section only cursorily and return to it only when
they hit upon something for which they need background
information.

We presuppose some familiarity with mathematical thinking,
in particular some knowledge of elementary set theory
%%%
\index{set}%%
%%%
and proof techniques such as induction. For basic concepts
in set theory see \cite{vaught:set} or
\cite{justweese:set1,justweese:set2}; for background
in logic see \cite{goldsternjudah}.
Concepts from algebra (especially universal algebra) can
be found in \cite{burris} and \cite{graetzer:algebra}, and in 
\cite{burmeister:partial} and \cite{burmeister:lecturenotes} 
for partial algebras; for general background on
lattices and orderings see \cite{graetzer:lattice}
and \cite{davey}.

We use the symbols
%%%
\index{$\cap$, $\cup$, $-$, $\varnothing$}%%
\index{$\wp$, $\wp_{fin}$}%%
%%%
$\cup$ for the union, $\cap$ for the intersection of two sets.
Instead of the difference symbol $M\backslash N$ we use $M - N$.
$\varnothing$ denotes the empty set. $\wp(M)$ denotes the set of
subsets of $M$, $\wp_{fin}(M)$ the set of finite subsets of $M$.
Sometimes it is necessary to take the union of two sets that does
not identify the common symbols from the different sets. In that
case one uses $+$. %
%%%
\index{$+$}%%
\index{$\in$, $\doteq$, $\subseteq$}%%
%%%%
We define $M + N := M \times \{0\} \cup N \times
\{1\}$ ($\times$ is defined below). This is called the 
\textbf{disjoint union}. For reference, we
fix the background theory of sets that we are using. This is the theory
{\sf ZFC} (Zermelo Fraenkel Set Theory with Choice). It is
essentially a first order theory with only two two place relation
symbols, $\in$ and $\doteq$. (See Section~\ref{kap3}.\ref{kap3-6} for a
definition of first order logic.) We define $x \subseteq y$ 
by $(\forall z)(z \in x \pf x \in y)$. Its axioms are as follows.
%%%
\begin{enumerate}
\item {\sl Singleton Set Axiom}.
    $(\forall x)(\exists y)(\forall z)(z \in y
    \dpf z \doteq x)$.  \\
    This makes sure that for every $x$ we have a set $\{x\}$.
\item {\sl Powerset Axiom}. $(\forall x)(\exists y)(\forall z)(%
    z \in y \dpf z \subseteq x)$. \\
    This ensures that for every $x$ the power set $\wp(x)$
    of $x$ exists.
\item {\sl Set Union}. $(\forall x)(\exists y)(%
    \forall z)(z \in y \dpf (\exists u)(z \in u \und u \in x))$.
    \\
    $u$ is denoted by $\bigcup_{z \in x} z$ or simply by
    $\bigcup x$. The axiom guarantees its existence.
\item {\sl Extensionality}. $(\forall x)(\forall y)(x \doteq y \dpf
    (\forall z)(z \in x \dpf z \in y))$.
\item {\sl Replacement.} If $f$ is a function with domain
    $x$ then the direct image of $x$ under $f$ is a set.
    (See below for a definition of {\it function}.)
\item {\sl Weak Foundation}. 
	$$(\forall x)(x \neq \varnothing \pf
    (\exists y)(y \in x \und (\forall z)(z \in x \pf
    z \not\in y)))$$
    This says that in every set there exists an element that
    is minimal with respect to $\in$.
\item {\sl Comprehension}. If $x$ is a set and $\varphi$ a
    first order formula with only $y$ occurring free, then 
	$\{y : y \in x \und \varphi(y)\}$ also is a set.
\item {\sl Axiom of Infinity}. There exists an $x$ and an injective
    function $f \colon x \pf x$ such that the direct image of
    $x$ under $f$ is not equal to $x$.
\item {\sl Axiom of Choice}. For every set of sets $x$ there
    is a function $f : x \pf \bigcup x$ with
    $f(y) \in y$ for all $y \in x$.
\end{enumerate}
%%%
We remark here that in everyday discourse, comprehension is
generally applied to all collections of sets, not just elementarily
definable ones. This difference will hardly matter here; we only
mention that in monadic second order logic this stronger from of 
comprehension is expressible and also the axiom of foundation.
%%%
\begin{quote}
%%%
\index{comprehension}
%%%%
{\sl Full Comprehension.} For every class $P$ and every set $x$, 
$\{y : y \in x \text{ and }x \in P\}$ is a set.
\end{quote}
%%%
Foundation is usually defined as follows 
%%%
\begin{quote}
%%%
\index{foundation}%%%
%%%
{\sl Foundation.} There is no infinite
chain $x_0 \ni x_1 \ni  x_2 \ni \dotsb$.
\end{quote}
%%%
In mathematical usage, one often forms certain collections of
sets that can be shown not to be sets themselves. One example
is the collection of all finite sets. The reason that it is not a
set is that for every set $x$, $\{x\}$ also is a set. The
function $x \mapsto \{x\}$ is injective (by extensionality), and so
there are as many finite sets as there are sets. If the collection of
finite sets were a set, say $y$, its powerset has strictly more elements
than $y$ by a theorem of Cantor. But this is impossible, since $y$
has the size of the universe. Nevertheless, mathematicians do use these
collections (for example, the collection of $\Omega$--algebras).
This is not a problem, if the following is observed. A collection 
of sets is called a \textbf{class}. %%
%%%
\index{class}%%
%%%%
A class is a set iff it is contained in a set as an element. (We 
use `iff' to abbreviate `if and only if'.)

In set theory, numbers are defined as follows.
%%
\begin{equation}
\begin{split}
0   & := \varnothing \\
n+1 & := \{k : k < n\} = \{0, 1, 2, \dotsc, n-1\}
\end{split}
\end{equation}
%%
\index{$\omega$, $\aleph_0$}%%
\index{$\wp(M)$}%%
%%%
The set of so--constructed numbers is denoted by $\omega$. It
is the set of \textbf{natural numbers}.
%%%
\index{natural number}%%
\index{ordinal}%%
%%%
In general, an \textbf{ordinal} (\textbf{number}) is a set that is
transitively and linearly ordered by $\in$. (See below for these
concepts.) For two ordinals $\kappa$ and $\lambda$, either $\kappa
\in \lambda$ (for which we also write $\kappa < \lambda$) or
$\kappa = \lambda$ or $\lambda \in \kappa$. 
%%%
\begin{thm}
For every set $x$ there exists an ordinal $\kappa$ and a bijective 
function $f \colon \kappa \pf x$.
\end{thm}
%%%
\index{well--ordering}%%%
%%%
$f$ is also referred to as a \textbf{well--ordering} of $x$.
The finite ordinals are exactly the natural numbers defined above.
%%%
\index{cardinal}%%
%%%
A \textbf{cardinal} (\textbf{number}) is an ordinal $\kappa$ such 
that for every ordinal $\lambda < \kappa$ there is no onto map 
$f \colon \lambda \pf \kappa$. It is not hard to see that every set 
can be well--ordered by a cardinal number, and this cardinal is unique. 
It is denoted by $|M|$ and called the \textbf{cardinality of} $M$. The 
smallest infinite cardinal is denoted by $\aleph_0$. The following is 
of fundamental importance.
%%%
\begin{thm}
For two sets $x$, $y$ exactly one of the following holds: $|x| < 
|y|$, $|x| = |y|$ or $|x| > |y|$.
\end{thm}
%%%
By definition, $\aleph_0$ is actually identical to $\omega$ so that 
it is not really necessary to distinguish the two. However, we shall 
do so here for reasons of clarity. (For example, infinite cardinals 
have a different arithmetic than ordinals.) If $M$ is finite, its 
cardinality is a natural number. If $|M| = \aleph_0$, $M$ is called 
\textbf{countable}; it is \textbf{uncountable} otherwise.
%%%
\index{set!countable}%%
%%%
If $M$ has cardinality $\kappa$, the cardinality of $\wp(M)$
is denoted by $2^{\kappa}$. $2^{\aleph_0}$ is the cardinality
of the set of all real numbers. $2^{\aleph_0}$ is strictly
greater than $\aleph_0$ (but need not be the smallest uncountable 
cardinal). We remark here that the set of finite sets of natural 
numbers is countable.

If $M$ is a set, a \textbf{partition} 
%%%
\index{partition}%%
%%%
of $M$ is a set $P \subseteq \wp(M)$ such that every member of $P$ 
is nonempty, $\bigcup P = M$ and for all $A, B \in P$ such that 
$A \neq B$, $A \cap B = \varnothing$. 
If $M$ and $N$ are sets, $M \times N$ denotes the set of all
pairs $\auf x, y\zu$, where $x \in M$ and $y \in N$.
A definition of $\auf x,y\zu$, which goes back to Kuratowski 
and Wiener, is as follows.
%%
\begin{equation}
\auf x, y\zu := \{x, \{x,y\}\}
\end{equation}
%%
\begin{lem}
$\auf x, y\zu = \auf u, v\zu$ iff $x = u$ and $y = v$.
\end{lem}
%%
\proofbeg
By extensionality, if $x = u$ and $y = v$ then $\auf x, y\zu =
\auf u, v\zu$. Now assume that $\auf x, y\zu = \auf u, v\zu$.
Then either $x = u$ or $x = \{u, v\}$, and $\{x, y\} = u$ or $\{x,y\} =
\{u,v\}$. Assume that $x = u$. If $u = \{x,y\}$ then $x = \{x, y\}$, 
whence $x \in x$, in violation to foundation. Hence we have 
$\{x,y\} = \{u, v\}$. Since $x = u$, we must have $y = v$. This 
finishes the first case. Now assume that $x = \{u,v\}$. Then 
$\{x,y\} = u$ cannot hold, for then $u = \{\{u,v\},y\}$, whence 
$u \in \{u,v\} \in u$. So, we must have $\{x,y\} = \{u,v\}$. 
However, this gives $x = \{x,y\}$, once again a contradiction.
So, $x = u$ and $y = v$, as promised.
\proofend

With these definitions, $M \times N$ is a set if $M$ and $N$ are
sets. A
%%%
\index{relation}%%
%%%
\textbf{relation} from $M$ to $N$ is a subset of
$M \times N$. We write $x\, R\, y$ if $\auf x, y\zu
\in R$. Particularly interesting is the case $M = N$.
A relation $R \subseteq M \times M$ is called
%%%
\index{relation!reflexive}%%
\index{relation!symmetric}%%
\index{relation!transitive}%%
\index{equivalence relation}%%
%%%
\textbf{reflexive} if $x \, R\, x$ for all $x \in M$;
\textbf{symmetric} if from $x\, R\, y$ follows that
$y \, R\, x$. $R$ is called \textbf{transitive} if from
$x \, R\, y$ and $y \, R\, z$ follows $x \, R \, z$.
An \textbf{equivalence relation} on $M$ is a reflexive,
symmetric and transitive relation on $M$.
%%%
\index{ordered set}%%
%%%
A pair $\auf M, <\zu$ is called an \textbf{ordered set}
if $M$ is a set and $<$ a transitive, irreflexive binary relation
on $M$. $<$ is then called a
%%%
\index{ordering}%%
%%%
(\textbf{strict}) \textbf{ordering on} $M$ and $M$ is then called
\textbf{ordered by $<$}. $<$ is \textbf{linear} if for any two
elements $x, y \in M$ either $x < y$ or $x = y$ or $y < x$.
A \textbf{partial ordering} is a relation which is reflexive,
transitive and antisymmetric; the latter means that
from $x\, R \, y$ and $y\, R\, x$ follows $x = y$.

If $R \subseteq M \times N$ is a relation, we write
$R^{\smallsmile} := \{\auf x, y\zu : y\, R\, x\}$ for the
so--called \textbf{converse of} $R$. This is a relation from
$N$ to $M$. If $S \subseteq N \times P$ and $T \subseteq M \times N$
are relations, put
%%%%
\index{$R \circ S$, $R \cup S$}%%%
%%%%
\begin{align}
R \circ S & := \{\auf x,y\zu : \text{ for some } 
	z\colon x\, R\, z \, S\, y\} \\\notag
R \cup T  & := \{\auf x,y\zu : x \, R\, y \text{ or } x \, T\, y\}
\end{align}
%%
We have $R \circ S \subseteq M \times P$ and $R \cup T \subseteq
M \times N$. In case $M = N$ we still make further definitions.
We put $\Delta_M := \{\auf x,x\zu : x \in M\}$ and call this set
the \textbf{diagonal on} $M$. Now put
%%%
\index{diagonal}%%
\index{$\Delta$, $R^n$, $R^+$, $R^{\ast}$}%%%
%%
\begin{align}
R^0      & := \Delta_M & R^{n+1}  & := R \circ R^n \\\notag
R^+      & := \bigcup_{0 < i \in \omega} R^i & R^{\ast} 
& := \bigcup_{i \in \omega} R^i
\end{align}
%%
$R^+$ is the smallest transitive relation which contains $R$.
%%%
\index{transitive closure}%%
%%%
It is therefore called the \textbf{transitive closure of} $R$.
$R^{\ast}$ is the smallest reflexive and transitive relation
containing $R$.

%%%
\index{function}%%
\index{function!partial}%%
%%%%
A \textbf{partial function} from $M$ to $N$ is a relation $f \subseteq
M \times N$ such that if $x \, f \, y$ and $x \, f \, z$
then $y = z$. $f$ is a \textbf{function} if for every $x$ there is a 
$y$ such that $x\; f\; y$. We write $y = f(x)$ to say that $x\, f\, y$
and $f \colon M \pf N$ to say that $f$ is a function from $M$ to
$N$. If $P \subseteq M$ then $f \restriction P :=
f \cap (P \times N)$.
%%%
\index{$f \restriction P$}%%
%%%
Further, $f \colon M \epi N$ abbreviates that $f$ is a
%%%
\index{function!surjective}%%
\index{$\epi$, $\mono$}%%
%%%
\textbf{surjective} function, that is, every $y \in N$ is of
the form $y = f(x)$ for some $x \in M$. And we write
%%%
\index{function!injective}%%
%%%
$f \colon M \mono N$ to say that $f$ is \textbf{injective}, that is,
for all $x, x' \in M$, if $f(x) = f(x')$ then $x = x'$.
%%%
\index{function!bijective}%%
%%%
$f$ is \textbf{bijective} if it is injective as well as surjective.
Finally, we write $f \colon x \mapsto y$ if $y = f(x)$. If
$X \subseteq M$ then $f[X] := \{f(x) : x \in X\}$ is the
%%%
\index{direct image}%%
\index{$f[X]$}%%
%%%
so--called \textbf{direct image of} $X$ \textbf{under}
$f$. We warn the reader of the difference between
$f(X)$ and $f[X]$. For example, let
$\Suc \colon \omega \pf \omega \colon x \mapsto x + 1$. Then
according to the definition of natural numbers above
we have $\Suc(4) = 5$ and $\Suc[4] = \{1,2,3,4\}$, since
$4 = \{0,1,2,3\}$. Let $M$ be an arbitrary set. There
is a bijection between the set of subsets of $M$ and the 
set of functions from $M$ to $2 = \{0,1\}$, which is defined 
as follows.  For $N \subseteq M$ we call $\chi_N \colon M \pf \{0,1\}$ 
the \textbf{characteristic function}
%%%%
\index{characteristic function}%%
%%%%
of $N$ if $\chi_N(x) = 1$ iff $x \in N$.
%%%
\index{$\chi_N$}%%
%%
Let $y \in N$ and $Y \subseteq N$; then put
$f^{-1}(y) := \{x : f(x) = y\}$ and
$f^{-1}[Y]  := \{x : f(x) \in Y\}$. If $f$ is injective,
$f^{-1}(y)$ denotes the unique $x$ such that $f(x) = y$
(if that exists). We shall see to it that this overload
in notation does not give rise to confusions.

$M^n$, $n \in \omega$, denotes the set of $n$--tuples
of elements from $M$. 
%%
\begin{align}
M^1     & := M & M^{n+1} & := M^n \times M
\end{align}
%%
In addition, $M^0 := 1 (= \{\varnothing\})$.
Then an $n$--tuple of elements from $M$ is an element of
$M^n$. Depending on need we shall write
$\auf x_i : i < n\zu$ or $\auf x_0, x_1, \dotsc, x_{n-1}\zu$
for a member of $M^n$.

An $n$--\textbf{ary relation} on $M$ is a subset of $M^n$,
an $n$--\textbf{ary function} on $M$ is a function $f \colon M^n \pf M$.
$n = 0$ is admitted. A 0--ary relation is a subset of $1$, hence 
it is either the empty set or the set $1$ itself. A 0--ary function 
on $M$ is a function $c \colon 1 \pf M$. We also call it a 
\textbf{constant}.
%%%
\index{constant}%%
%%%
The \textbf{value} of this constant is the element $c(\varnothing)$.
Let $R$ be an $n$--ary relation and $\vec{x} \in M^n$. Then we
write $R(\vec{x})$ in place of $\vec{x} \in R$.

Now let $F$ be a set and $\Omega \colon F \pf \omega$.
%%%
\index{signature}%%
%%%
The pair $\auf F, \Omega\zu$, also denoted by $\Omega$
alone, is called a \textbf{signature} and $F$ the set of
\textbf{function symbols}.
%%
\begin{defn}
%%%
\index{algebra}%%
\index{algebra!$\Omega$--}%%%
%%%
Let $\Omega \colon F \pf \omega$ be a signature and $A$
a nonempty set. Further, let $\Pi$ be a mapping
which assigns to every $f \in F$ an $\Omega(f)$--ary
function on $A$. Then we call the pair $\GA := \auf
A, \Pi\zu$ an \textbf{$\Omega$--algebra}. $\Omega$--algebras
are in general denoted by upper case German letters.
\end{defn}
%%
In order not to get drowned in notation we write $f^{\GA}$ 
for the function $\Pi(f)$. In place of denoting $\GA$ by the 
pair $\auf A, \Pi\zu$ we shall denote it somewhat ambiguously 
by $\auf A, \{f^{\GA} : f \in F\}\zu$. We warn the reader that 
the latter notation may give rise to confusion since
functions of the same arity can be associated with different
function symbols. However, this problem shall not arise.

%%%
\index{$\Tm_{\Omega}$}%%
\index{term}%%%
%%%
The set of $\Omega$--terms is the smallest set $\Tm_{\Omega}$ 
such that if $f \in F$ and $t_i \in \Tm_{\Omega}$, 
$i < \Omega(f)$, also $f(t_0, \dotsc, t_{\Omega(f)-1}) \in
    \Tm_{\Omega}$.
Terms are abstract entities; they are not to be equated
with functions nor with the strings by which we denote them.
%%%
\index{term!level of a \faul}%
%%%
To begin we define the \textbf{level} of a term. If $\Omega(f) = 0$,
then $f()$ is a term of level 0, which we also denote by `$f$'.
If $t_i$, $i < \Omega(f)$, are terms of level $n_i$,
then $f(t_0, \dotsc, t_{\Omega(f)-1})$ is a term of level
$1 + \max \{n_i : i < \Omega(f)\}$. Many proofs
run by induction on the level of terms, we therefore speak about
{\it induction on the construction of the term}. Two terms $u$ and
$v$ are equal, in symbols $u = v$, if they have identical level and
either they are both of level 0 and there is an $f \in F$ such
$u = v = f()$ or there is an $f \in F$, and terms $s_i$, $t_i$,
$i < \Omega(f)$, such that $u = f(s_0, \dotsc, s_{\Omega(f)-1})$
and $v = f(t_0, \dotsc, t_{\Omega(f) -1})$ as well as $s_i = t_i$
for all $i < \Omega(f)$.

An important example of an $\Omega$--algebra is the so--called {\it
term algebra}. We choose an arbitrary set $X$ of symbols, which
must be disjoint from $F$. The signature is extended to $F \cup X$
such that the symbols of $X$ have arity 0. The terms over this new
signature are called $\Omega$--\textbf{terms over} $X$.
%%%
\index{term!$\Omega$--\faul}%%
%%%
The set of $\Omega$--terms over $X$ is denoted by
$\Tm_{\Omega}(X)$. Then we have $\Tm_{\Omega} = 
\Tm_{\Omega}(\varnothing)$. For many purposes (indeed 
most of the purposes of this book) the terms $\Tm_{\Omega}$ 
are sufficient. For we can always resort to the following trick. 
For each $x \in X$ add a 0--ary function symbol $\uli{x}$ to $F$. 
This gives a new signature $\Omega_X$, also called the 
%%%
\index{signature!constant expansion}%%%
%%%
\textbf{constant expansion of} $\Omega$ by $X$. Then 
$\Tm_{\Omega_X}$ can be canonically identified with 
$\Tm_{\Omega}(X)$.

There is an algebra which has as its objects the terms and 
which interprets the function symbols as follows.
%%
\begin{equation}
\Pi(f) \colon \auf t_i : i < \Omega(f)\zu \mapsto
    f(t_0, \dotsc, t_{\Omega(f)-1}) 
\end{equation}
%%
\index{$\goth{Tm}_{\Omega}(X)$}%%
%%%
Then $\goth{Tm}_{\Omega}(X) := \auf \Tm_{\Omega}(X), \Pi\zu$ is
%%%
\index{term algebra}%%
%%%
an $\Omega$--algebra, called the \textbf{term algebra generated by}
$X$. It has the following property. For any $\Omega$--algebra
$\GA$ and any map $v \colon X \pf A$ there is exactly one homomorphism
$\oli{v} \colon \Tm_{\Omega}(X) \pf \GA$ such that $\oli{v}
\restriction X = v$. This will be restated in 
Proposition~\ref{prop:freegen}.
%%%
\begin{defn}
Let $\GA$ be an $\Omega$--algebra and $X \subseteq A$. We say that
$X$ \textbf{generates} $\GA$ if $A$ is the smallest subset which
contains $X$ and which is closed under all functions $f^{\GA}$. If
$|X| = \kappa$ we say that $\GA$ is $\kappa$--\textbf{generated}.
Let $\CK$ be a class of $\Omega$--algebras and $\GA \in \CK$. We
say that $\GA$ is \textbf{freely generated by $X$ in $\CK$} if for 
every $\GB \in \CK$ and maps $v \colon X \pf B$ there is exactly 
one homomorphism $\oli{v} \colon \GA \pf \GB$ such that $\oli{v} 
\restriction X = v$. If $|X| = \kappa$ we say that $\GA$ is 
%%%%
\index{algebra!freeely ($\kappa$-)generated}%%%
%%%%
\textbf{freely $\kappa$--generated in} $\CK$.
\end{defn}
%%%
\begin{prop}
\label{prop:freegen}
Let $\Omega$ be a signature, and let $X$ be disjoint from $F$. Then
the term algebra over $X$, $\goth{Tm}_{\Omega}(X)$, is freely generated
by $X$ in the class of all $\Omega$--algebras.
\end{prop}
%%%
The following is left as an exercise. It is the justification
for writing $\goth{Fr}_{\CK}(\kappa)$ for the (up to isomorphism
unique) freely $\kappa$--generated algebra of $\CK$. In varieties 
such an algebra always exists.
%%%
\begin{prop}
\label{prop:free}
Let $\CK$ be a class of $\Omega$--algebras and $\kappa$ a
cardinal number. If $\GA$ and $\GB$ are both freely $\kappa$--generated
in $\CK$ they are isomorphic.
\end{prop}
%%%
Maps of the form $\sigma \colon X \pf \Tm_{\Omega}(X)$,
as well as their homomorphic extensions are called
%%%%
\index{substitutions}%%
%%%
\textbf{substitutions}. If $t$ is a term over $X$, we also write
$\sigma(t)$ in place of $\oli{\sigma}(t)$. Another notation,
frequently employed in this book, is as follows. Given terms
$s_i$, $i < n$, we write $[s_i/x_i \colon i < n]t$ in place of
$\sigma(t)$, where $\sigma$ is defined as follows.
%%%
\begin{equation}
\sigma(y) := \begin{cases}
    s_i & \text{ if $y = x_i$,} \\
    y   & \text{ else.}
\end{cases}
\end{equation}
%%%
(Most authors write $t[s_i/x_i \colon i < n]$, but this notation will
cause confusion with other notation that we use.)

Terms induce term functions on a given $\Omega$--algebra $\GA$.
Let $t$ be a term with variables $x_i$, $i < n$. (None of these
variables has to occur in the term.) Then $t^{\GA} \colon A^n \pf A$ is
defined inductively as follows (with $\vec{a} = \auf a_i : i <
\Omega(f)\zu$).
%%%
\index{term function}%%%
\index{term function!clone of}%%%
%%%
\begin{align}
x_i^{\GA}(\vec{a}) & := a_i \\\notag
(f(t_0, \dotsc, t_{\Omega(f)-1}))^{\GA}
    (\vec{a}) & := f^{\GA}(t_0^{\GA}(\vec{a}), \dotsc,
    t_{\Omega(f)-1}^{\GA}(\vec{a}))
\end{align}
%%
We denote by $\Clo_n(\GA)$ the set of $n$--ary term
functions on $\GA$. This set is also called the \textbf{clone of}
$n$--\textbf{ary term functions of} $\GA$. 
%%%
\index{polynomial}%%
%%%
A \textbf{polynomial of}
$\GA$ is a term function over an algebra that is like $\GA$ but
additionally has a constant for each element of $A$. (So, we form
the constant expansion of the signature with every $a \in A$.
Moreover, $\uli{a}$ (more exactly, $\uli{a}()$) shall have value
$a$ in $A$.) The clone of $n$--ary term functions of this algebra 
is denoted by $\Pol_n(\GA)$. For example, $((x_0 + x_1) 
\cdot x_0)$ is a term and
denotes a binary term function in an algebra for the signature
containing only $\cdot$ and $+$. However, $(2 + (x_0 \cdot x_0))$
is a polynomial but not a term. Suppose that we add a constant
\textbf{1} to the signature, with denotation 1 in the natural numbers.
Then $(2 + (x_0 \cdot x_0))$ is still not a term of the expanded
language (it lacks the symbol $2$), but the associated function
actually is a term function, since it is identical with the function
induced by the term $((\mbox{\bf 1}+\mbox{\bf 1}) + (x_0 \cdot x_0))$.
%%
\begin{defn}
%%%
\index{homomorphism}%%
\index{isomorphism}%%
\index{automorphism}%%
\index{endomorphism}%%
%%%
Let $\GA = \auf A, \{f^{\GA} \colon f \in F\}\zu$ and
$\GB = \auf B, \{f^{\GB} \colon f \in F\}\zu$ be $\Omega$--algebras
and $h \colon A \pf B$. $h$ is called a \textbf{homomorphism}
if for every $f \in F$ and every $\Omega(f)$--tuple
$\vec{x} \in A^{\Omega(f)}$ we have
%%
\begin{equation}
h(f^{\GA}(\vec{x})) = f^{\GB}(h(x_0), h(x_1), \dotsc, h(x_{\Omega(f)-1}))
\end{equation}
%%
We write $h \colon \GA \pf \GB$ if $h$ is a homomorphism from
$\GA$ to $\GB$. Further, we write $h \colon \GA \epi \GB$ if $h$
is a surjective homomorphism and $h \colon \GA \mono \GB$ if $h$
is an injective homomorphism. $h$ is an \textbf{isomorphism}
if $h$ is injective as well as surjective. $\GB$ is called
\textbf{isomorphic} to $\GA$, in symbols $\GA \cong \GB$ if there 
is an isomorphism from $\GA$ to $\GB$. If $\GA = \GB$
we call $h$ an \textbf{endomorphism of} $\GA$; if $h$ is 
additionally bijective then $h$ is called an
\textbf{automorphism} of $\GA$.
\end{defn}
%%
If $h \colon A \pf B$ is an isomorphism from $\GA$ to $\GB$ then
$h^{-1} \colon B \pf A$ is an isomorphism from $\GB$ to $\GA$.
%%
\begin{defn}
%%%
\index{congruence relation}%%
%%%
Let $\GA$ be an $\Omega$--algebra and $\Theta$ a
binary relation on $A$. $\Theta$ is called a
\textbf{congruence relation on} $\GA$ if $\Theta$ is an
equivalence relation and for all $f \in F$ and
all $\vec{x}, \vec{y} \in A^{\Omega(f)}$ we have:
%%
\begin{equation}
\label{eq:congr}
\text{If $x_i\, \Theta\, y_i$ for all $i < \Omega(f)$
then $f^{\GA}(\vec{x})\, \Theta\, f^{\GA}(\vec{y})$.}
\end{equation}
%%
\end{defn}
%%
We also write $\vec{x}\; \Theta\; \vec{y}$ in place of `$x_i \; \Theta \; 
y_i$ for all $i < \Omega(f)$'. If $\Theta$ is an equivalence relation put
%%%
\index{$[x]\Theta$}%%%
%%%
\begin{equation}
[x]\Theta := \{y : x \, \Theta\, y\} 
\end{equation}
%%
%%%
\index{equivalence class}%%
%%%
We call $[x]\Theta$ the \textbf{equivalence class of}
$x$. Then for all $x$ and $y$ we have either $[x]\Theta = [y]\Theta$
or $[x]\Theta \cap [y]\Theta = \varnothing$. Further,
we always have $x \in [x]\Theta$. If $\Theta$ additionally is
a congruence relation then the following holds: if
$y_i \in [x_i]\Theta$ for all $i < \Omega(f)$ then
$f^{\GA}(\vec{y}) \in [f^{\GA}(\vec{x})]\Theta$. Therefore
the following definition is independent of representatives.
%%%%
\begin{equation}
[f^{\GA}]\Theta([x_0]\Theta, [x_1]\Theta, \dotsc,
[x_{\Omega(f)-1}]\Theta)  
:= [f^{\GA}(\vec{x})]\Theta 
\end{equation}
%%
Namely, let $y_0 \in [x_0]\Theta,\dotsc, y_{\Omega(f)-1} \in 
[x_{\Omega(f)-1}]\Theta$. Then $y_i \; \Theta\; x_i$ for all 
$i < \Omega(f)$.  Then because of \eqref{eq:congr} we immediately
have $f^{\GA}(\vec{y}) \; \Theta\; f^{\GA}(\vec{x})$.
This simply means $f^{\GA}(\vec{y}) \in [f^{\GA}(\vec{x})]\Theta$.
Put 
%%%%
\index{$\GA/\Theta$}%%%%
%%%%
\begin{align}
A/\Theta & := \{[x]\Theta : x \in A\} \\
 \GA/\Theta & := \auf A/\Theta, \{[f^{\GA}]\Theta : f \in F\}\zu
\end{align}
%%%%
\index{factorization}%%%
%%%%
We call $\GA/\Theta$ the \textbf{factorization of} $\GA$ \textbf{by}
$\Theta$. The map $h_{\Theta} \colon x \mapsto [x]\Theta$ is easily
proved to be a homomorphism.

Conversely, let $h \colon \GA \pf \GB$ be a homomorphism.
Then put 
%%%
\index{$\ker$}%%%
\begin{equation}
\ker(h) := \{\auf x, y\zu \in A^2 :  h(x) = h(y)\}
\end{equation}
%%%
$\ker(h)$ is a congruence relation on $\GA$. Furthermore,
$\GA/\ker(h)$ is isomorphic to $\GB$ if $h$ is surjective. 
A set $B \subseteq A$ is \textbf{closed under} $f \in F$ if for 
all $\vec{x} \in B^{\Omega(f)}$ we have $f^{\GA}(\vec{x}) \in B$.
%%
\begin{defn}
%%%
\index{subalgebra}%%
%%%
Let $\auf A, \{f^{\GA} : f \in F\}\zu$ be an
$\Omega$--algebra and $B \subseteq A$ closed under all
$f \in F$. Put $f^{\GB} := f^{\GA} \restriction B^{\Omega(f)}$.
The pair $\auf B, \{f^{\GB} : f \in F\}\zu$ is called a
\textbf{subalgebra} of $\GA$.
\end{defn}
%%
The product of the algebras $\GA_i$, $i \in I$, is defined as 
follows.  The carrier set is the set of functions $\alpha \colon I % 
\pf \bigcup_{i \in I} A_i$ such that $\alpha(i) \in A_i$ for all 
$i \in I$. Call this set $P$. For an $n$--ary function symbol $f$ 
put
%%%
\index{algebra!product}%%
\index{product of algebras}%%
%%
\begin{multline}
f^{\GP}(\alpha_0, \dotsc, \alpha_{n-1})(i)  \\
      := \auf f^{\GA_i}(\alpha_0(i)), f^{\GA_i}(\alpha_1(i)),
    \dotsc, f^{\GA_i}(\alpha_{n-1}(i))\zu
\end{multline}
%%
The resulting algebra is denoted by $\prod_{i \in I} \GA_i$.
One also defines the product $\GA \times \GB$ in the following
way. The carrier set is $A \times B$ and for an $n$--ary function
symbol $f$ we put
%%
\begin{multline}
f^{\GA\times\GB}(\auf a_0,b_0\zu, \dotsc, \auf a_{n-1}, b_{n-1}\zu)
    \\
    := \auf f^{\GA}(a_0, \dotsc, a_{n-1}), f^{\GB}(b_0, \dotsc,
    b_{n-1})\zu
\end{multline}
%%
The algebra $\GA\times \GB$ is isomorphic to the algebra
$\prod_{i \in 2} \GA_i$, where $\GA_0 := \GA$, $\GA_1 := \GB$.
However, the two algebras are not identical. (Can you verify
this?)

A particularly important concept is that of a {\it variety}
or {\it equationally definable class of algebras}.
%%%
\begin{defn}
%%%
\index{variety}%%
%%%
Let $\Omega$ be a signature. A class of $\Omega$--algebras is
called a \textbf{variety} if it is closed under isomorphic copies,
subalgebras, homomorphic images, and (possibly infinite)
products.
\end{defn}
%%%
Let $V := \{x_i : i \in \omega\}$ be the set of variables.
%%%%
\index{equation}%%%
%%%%
An \textbf{equation} is a pair $\auf s, t\zu$ of $\Omega$--terms
(involving variables from $V$). We introduce a formal symbol 
`$\boldsymbol{\doteq}$' and write $s \boldsymbol{\doteq} t$ 
for this pair. An algebra $\GA$ satisfies the equation 
$s \boldsymbol{\doteq} t$ iff for all maps $v : V \pf A$, 
$\oli{v}(s) = \oli{v}(t)$. We then write 
$\GA \vDash s \boldsymbol{\doteq} t$. A class $\CK$ of 
$\Omega$--algebras satisfies this equation if every algebra of 
$\CK$ satisfies it.  We write $\CK \vDash s \boldsymbol{\doteq} t$.
%%%
\begin{prop}
\label{prop:eqcalc}
The following holds for all classes $\CK$ of $\Omega$--algebras.
%%
\begin{dingautolist}{192}
\item $\CK \vDash s \boldsymbol{\doteq} s$.
\item If $\CK \vDash s \boldsymbol{\doteq} t$ then 
$\CK \vDash t \boldsymbol{\doteq} s$.
\item If $\CK \vDash s \boldsymbol{\doteq} t; t \boldsymbol{\doteq} u$ 
	then $\CK \vDash s \boldsymbol{\doteq} u$.
\item If $\CK \vDash s_i \boldsymbol{\doteq} t_i$ for all $i < \Omega(f)$ 
	then $\CK \vDash f(\vec{s}) \boldsymbol{\doteq} f(\vec{t})$.
\item If $\CK \vDash s \boldsymbol{\doteq} t$ and $\sigma \colon V \pf
    \Tm_{\Omega}(V)$ is a substitution, then
    $\CK \vDash \sigma(s) \boldsymbol{\doteq} \sigma(t)$.
\end{dingautolist}
%%%
\end{prop}
%%%
The verification of this is routine. It follows from the first three
facts that equality is an equivalence relation on the algebra
$\goth{Tm}_{\Omega}(V)$, and together with the fourth that
the set of equations valid in $\CK$ form a congruence
on $\goth{Tm}_{\Omega}(V)$. There is a bit more we can say. Call
%%%
\index{congruence!fully invariant}%%
%%%
a congruence $\Theta$ on $\GA$ \textbf{fully invariant} if for all
endomorphisms $h \colon \GA \pf \GA$: if $x \; \Theta\; y$ then
$h(x) \; \Theta\; h(y)$. The next theorem follows immediately
once we observe that the endomorphisms of $\goth{Tm}_{\Omega}(V)$
are exactly the substitution maps. To this end, let $h\colon 
\goth{Tm}_{\Omega}(V) \pf \goth{Tm}_{\Omega}(V)$. Then
$h$ is uniquely determined by $h \restriction V$, since
$\goth{Tm}_{\Omega}(V)$ is freely generated by $V$. It is easily
computed that $h$ is the substitution defined by $h \restriction V$.
Moreover, every map $v \colon V \pf \goth{Tm}_{\Omega}(V)$ induces
a homomorphism $\oli{v} \colon \goth{Tm}_{\Omega}(V) \pf
\goth{Tm}_{\Omega}(V)$, which is unique. Now write $\mbox{\rm Eq}(\CK) :=
\{\auf s, t\zu : \CK \vDash s \boldsymbol{\doteq} t\}$.
%%%
\begin{cor}
Let $\CK$ be a class of $\Omega$--algebras. Then $\mbox{\rm Eq}(\CK)$
is a fully invariant congruence on $\goth{Tm}_{\Omega}(V)$.
\end{cor}
%%%
Let $E$ be a set of equations. Then put
%%
\begin{equation}
%%%
\index{$\mathsf{Alg}(E)$}%%
%%%
\mathsf{Alg}(E) := \{\GA : \text{ for all }
\auf s, t\zu \in E:  \GA \vDash s \boldsymbol{\doteq} t\}
\end{equation}
%%
This is a class of $\Omega$--algebras. Classes of $\Omega$--algebras
that have the form $\mathsf{Alg}(E)$ for some $E$ are
%%%%
\index{equationally definable class}%%%
%%%
called \textbf{equationally definable}. 
%%%
\begin{prop}
Let $E$ be a set of equations. Then $\mathsf{Alg}(E)$ is
a variety.
\end{prop}
%%%
We state without proof the following result.
%%%
\begin{thm}[Birkhoff]
Every variety is an equationally definable class. Furthermore,
there is a biunique correspondence between varieties and fully
invariant congruences on the algebra $\goth{Tm}_{\Omega}(\aleph_0)$.
\end{thm}
%%%
The idea for the proof is as follows. It can be shown that every
variety has free algebras. For every cardinal number $\kappa$,
$\goth{Fr}_{\CK}(\kappa)$ exists. Moreover, a variety is uniquely
characterized by $\goth{Fr}_{\CK}(\aleph_0)$. In fact, every
algebra is a subalgebra of a direct image of some product of
$\goth{Fr}_{\CK}(\aleph_0)$. Thus, we need to investigate the
equations that hold in the latter algebra. The other algebras will
satisfy these equations, too. The free algebra is the image of
$\goth{Tm}_{\Omega}(V)$ under the map $x_i \mapsto i$. The induced
congruence is fully invariant, by the freeness of
$\goth{Fr}_{\CK}(\aleph_0)$. Hence, this congruence simply {\it is\/}
the set of equations valid in the free algebra, hence in the whole
variety. Finally, if $E$ is a set of equations, we write $E \vDash
t \boldsymbol{\doteq} u$ if $\GA \vDash t \boldsymbol{\doteq} u$ 
for all $\GA \in \mathsf{Alg}(E)$.
%%%
\begin{thm}[Birkhoff]
$E \vDash t \boldsymbol{\doteq} u$ iff $t \boldsymbol{\doteq} u$ 
can be derived from $E$ by means of the rules given in
Proposition~\ref{prop:eqcalc}.
\end{thm}

The notion of an algebra can be extended into two directions, both
of which shall be relevant for us. The first is the concept of a
many--sorted algebra.
%%%
\begin{defn}
\index{signature!sorted}%%
\index{sort}%%
A \textbf{sorted signature} is a triple $\auf F, \CS, \Omega\zu$,
where $F$ and $\CS$ are sets, the set of \textbf{function symbols}
and of \textbf{sorts}, respectively, and $\Omega \colon F \pf \CS^+$ a
function assigning to each element of $F$ its so--called 
\textbf{signature}. We shall denote the signature by the letter 
$\Omega$, as in the unsorted case.%%
\end{defn}%%
%%
So, the signature of a function is a (nonempty) sequence of sorts.
The last member of that sequence tells us what sort the result
has, while the others tell us what sort the individual arguments
of that function symbol have.
%%
\begin{defn}%%
%%%
\index{algebra!many--sorted}%%
%%%
A (\textbf{sorted}) \textbf{$\Omega$--algebra} is a pair $\GA = \auf
\{A_{\sigma} : \sigma \in \CS\}, \Pi\zu$ such that for every
$\sigma \in \CS$ $A_{\sigma}$ is a set and for every $f \in F$
such that $\Omega(f) = \auf \sigma_i : i < n+1\zu$
%%%
\begin{equation}
\Pi(f) \colon A_{\sigma_0} \times A_{\sigma_1} \times \cdots \times
A_{\sigma_{n-1}} \pf A_{\sigma_n} 
\end{equation}
%%
If $\GB = \auf \{B_{\sigma} : \sigma \in \CS\}, \Sigma\zu$ is another 
%%%
\index{homomorphism!sorted}%%
%%%
$\Omega$--algebra, a (\textbf{sorted}) \textbf{homomorphism from} $\GA$ 
\textbf{to} $\GB$ is a set $\{h_{\sigma} \colon A_{\sigma} \pf B_{\sigma} 
: \sigma \in \CS\}$ of functions such that for each $f \in F$ with 
signature $\auf \sigma_i : i < n+1\zu$:
%%%
\begin{equation}
h_{\sigma_n}(f^{\GA}(a_0, \dotsc, a_{n-1})) =
f^{\GB}(h_{\sigma_0}(a_0), \dotsc, h_{\sigma_{n-1}}(a_{n-1}))
\end{equation}
%%%
A \textbf{many--sorted algebra} is an $\Omega$--algebra of some
signature $\Omega$.
\end{defn}
%%
Evidently, if $\CS = \{\sigma\}$ for some $\sigma$, then the
notions coincide (modulo trivial adaptations) with those of
unsorted algebras. Terms are defined as before, but now they are 
sorted. First, for each sort we assume a countably infinite set 
$V_{\sigma}$ of variables. Moreover, $V_{\sigma} \cap V_{\tau} 
= \varnothing$ whenever $\sigma \neq \tau$. Now, every term is 
given a unique sort in the following way.
%%%
\begin{dingautolist}{192}
\item If $x \in V_{\sigma}$, then $x$ has sort $\sigma$.
\item $f(t_0, \dotsc, t_{n-1})$ has sort $\sigma_n$, if
$\Omega(f) = \auf \sigma_i : i < n+1\zu$ and $t_i$ has sort 
$\sigma_i$ for all $i < n$.
\end{dingautolist}
%%
The set of terms over $V$ is denoted by $\Tm_{\Omega}(V)$. This can 
be turned into a sorted $\Omega$--algebra; simply let 
$\Tm_{\Omega}(V)_{\sigma}$
be the set of terms of sort $\sigma$. Again, given a map $v$ that
assigns to a variable of sort $\sigma$ an element of $A_{\sigma}$,
there is a unique homomorphism $\oli{v}$ from the
$\Omega$--algebra of terms into $\GA$. If $t$ has sort $\sigma$,
then $\oli{v}(t) \in A_{\sigma}$. A 
%%%
\index{equation!sorted}%%%
%%%
\textbf{sorted equation} is a pair
$\auf s, t\zu$, where $s$ and $t$ are of equal sort. We denote
this pair by $s \boldsymbol{\doteq} t$. We write $\GA \vDash s 
\boldsymbol{\doteq} t$ if for all maps $v$ into $\GA$, 
$\oli{v}(s) = \oli{v}(t)$. The Birkhoff Theorems have direct 
analogues for the many sorted algebras, and can be proved in the 
same way.

Sorted algebras are one way of introducing partiality. To be able
to compare the two approaches, we first have to introduce partial
algebras. We shall now return to the unsorted notions, although it
is possible --- even though not really desirable --- to introduce
partial many--sorted algebras as well.
%%%
\begin{defn}
Let $\Omega$ be an unsorted signature. A \textbf{partial
$\Omega$--algebra} is a pair $\auf A, \Pi\zu$, where $A$ is a set
and for each $f \in F$: $\Pi(f)$ is a partial function from
$A^{\Omega(f)}$ to $A$.
\end{defn}
%%%
The definitions of canonical terms split into different notions in
the partial case.
%%%
\begin{defn}
%%%
\index{homomorphism}%%
%%%
Let $\GA$ and $\GB$ be partial $\Omega$--algebras, and $h \colon A \pf
B$ a map. $h$ is a 
%%%
\index{homomorphism!weak}%%
%%%%
\textbf{weak homomorphism from} $\GA$ \textbf{to} $\GB$
if for every $\vec{a} \in A^{\Omega(f)}$ we have $h(f^{\GA}(\vec{a})) =
f^{\GB}(h(\vec{a}))$ if both sides are defined. $h$ is a 
\textbf{homomorphism} if it is a weak homomorphism and for every
$\vec{a}\in A^{\Omega(f)}$ if $h(f^{\GA}(\vec{a}))$ is defined
then so is $f^{\GB}(h(\vec{a}))$. Finally, $h$ is a
%%%
\index{homomorphism!strong}%%
%%%%
\textbf{strong homomorphism} if it is a homomorphism and
$h(f^{\GA}(\vec{a}))$ is defined iff $f^{\GB}(h(\vec{a}))$ is. 
$\GA$ is a 
%%%%
\index{subalgebra!strong}%%
%%%%
\textbf{strong subalgebra of} $\GB$ if $A \subseteq B$ 
and the identity map is a strong homomorphism.
\end{defn}
%%%
\begin{defn}
\label{defn:pcongruence}
An equivalence relation $\Theta$ on $A$ is called a 
%%%%
\index{congruence!weak}%%
%%%
\textbf{weak congruence of} $\GA$ if for every $f \in F$ and every 
$\vec{a}, \vec{c} \in A^{\Omega(f)}$ if $\vec{a}\; \Theta\; \vec{c}$ 
and $f^{\GA}(\vec{a})$, $f^{\GA}(\vec{c})$ are both
defined, then $f^{\GA}(\vec{a})\; \Theta\; f^{\GA}(\vec{c})$.
$\Theta$ is 
%%%%
\index{congruence!strong}%%
%%%%
\textbf{strong} if in addition $f^{\GA}(\vec{a})$
is defined iff $f^{\GA}(\vec{c})$ is.
\end{defn}
%%%
It can be shown that the equivalence relation induced by a weak (strong)
homomorphism is a weak (strong) congruence, and that every weak (strong)
congruence defines a surjective weak (strong) homomorphism.

%%%%
\index{$\vDash^w$, $\vDash^s$}%%
%%%%
Let $v \colon V \pf A$ be a function, $t = f(s_0, \dotsc, s_{\Omega(f)-1})$ 
a term. Then $\oli{v}(t)$ is defined iff 
(a) $\oli{v}(s_i)$ is defined for every $i < \Omega(f)$ and (b)
$f^{\GA}$ is defined on $\auf \oli{v}(s_i) : i < n\zu$. Now, we
write  $\auf \GA, v\zu \vDash^w s \boldsymbol{\doteq} t$ if $\oli{v}(s) =
\oli{v}(t)$ in case both are defined and equal; $\auf \GA, v\zu \vDash^s s
\boldsymbol{\doteq} t$ if $\oli{v}(s)$ is defined iff $\oli{v}(t)$
is and if one is defined the two are equal. An equation 
$s \boldsymbol{\doteq} t$ is said to hold in $\GA$ in the \textbf{weak} 
(\textbf{strong}) \textbf{sense}, if 
$\auf \GA, v\zu \vDash^w s \boldsymbol{\doteq} t$ 
($\auf \GA, v\zu \vDash^s s \boldsymbol{\doteq} t$) for all 
$v \colon V \pf A$. Proposition~\ref{prop:eqcalc}
holds with respect to $\vDash^s$ but not with respect to
$\vDash^w$. Also, algebras satisfying an equation in the strong
sense are closed under products, strong homomorphic images and
under strong subalgebras.

The relation between classes of algebras and sets of equations is
called a 
%%%%
\index{Galois correspondence}%%
%%%%
\textbf{Galois correspondence}. It is useful to know a few
facts about such correspondences. Let $A$, $B$ be sets and $R
\subseteq A \times B$ ($A$ and $B$ may in fact also be classes). 
The triple $\auf A, B, R\zu$ is called
%%%
\index{context}%%
%%%
a \textbf{context}. Now define the following operators:
%%
\begin{align}
^{\uparrow} \colon \wp(A) \pf \wp(B) \colon & O \mapsto
    \{y \in B : \mbox{ for all }x \in O: x\; R\; y\} \\
^{\downarrow} \colon \wp(B) \pf \wp(A) \colon & P \mapsto
    \{x \in A : \mbox{ for all }y \in P: x\; R\; y\}
\end{align}
%%
One calls $O^{\uparrow}$ the 
%%%
\index{intent}%%%
%%%%
\textbf{intent of} $O \subseteq A$ and $P^{\downarrow}$ the 
%%%
\index{extent}%%%
%%%%
\textbf{extent} of $P \subseteq B$.
%%%
\begin{thm}
Let $\auf A, B, R\zu$ be a context. Then the following holds
for all $O, O^{\ast} \subseteq A$ and all $P, P^{\ast} \subseteq B$.
%%
\begin{dingautolist}{192}
\item
$O \subseteq P^{\downarrow}$ iff $O^{\uparrow} \supseteq P$.
\item
If $O \subseteq O^{\ast}$ then $O^{\uparrow} \supseteq O^{\ast\uparrow}$.
\item
If $P \subseteq P^{\ast}$ then $P^{\downarrow} \supseteq P^{\ast\downarrow}$.
\item
$O \subseteq O^{\uparrow\downarrow}$.
\item
$P \subseteq P^{\downarrow\uparrow}$.
\end{dingautolist}
\end{thm}
%%%
\proofbeg 
Notice that if $\auf A, B, R\zu$ is a context, $\auf B,
A, R^{\smallsmile}\zu$ also is a context, and so we only need to show
\ding{192}, \ding{193} and \ding{195}. \ding{192}. $O \subseteq 
P^{\downarrow}$ iff every $x \in O$ stands in relation $R$ to every 
member of $P$ iff $P \subseteq O^{\uparrow}$. \ding{193}. If $O
\subseteq O^{\ast}$ and $y \in O^{\ast\uparrow}$, then for every $x \in
O^{\ast}$: $x\; R \; y$. This means that for every $x \in O$: $x\; R\;
y$, which is the same as $y \in O^{\uparrow}$. \ding{195}. Notice that
$O^{\uparrow} \supseteq O^{\uparrow}$ by \ding{192} implies $O \subseteq
O^{\uparrow\downarrow}$. \proofend
%%
\begin{defn}
%%%
\index{closure operator}%%
%%%
Let $M$ be a set and $H \colon \wp(M) \pf \wp(M)$ a function.
$H$ is called a \textbf{closure operator on} $M$ if
for all $X, Y \subseteq M$ the following holds.
%%
\begin{dingautolist}{192}
\item $X \subseteq H(X)$.
\item If $X \subseteq Y$ then $H(X) \subseteq H(Y)$.
\item $H(X) = H(H(X))$.
\end{dingautolist}
%%%
A set $X$ is called \textbf{closed} if $X = H(X)$.
\end{defn}
%%
\begin{prop}
\label{prop:closure} Let $\auf A, B, R\zu$ be a context. Then $O
\mapsto O^{\uparrow\downarrow}$ and $P \mapsto P^{\downarrow\uparrow}$ 
are closure operators on $A$ and $B$, respectively. The closed sets 
are the sets of the form $P^{\downarrow}$ for the first, and 
$O^{\uparrow}$ for the second operator.
\end{prop}
%%
\proofbeg
We have $O \subseteq O^{\uparrow\downarrow}$, from which
$O^{\uparrow} \supseteq O^{\uparrow\downarrow\uparrow}$. On
the other hand, $O^{\uparrow} \subseteq O^{\uparrow\downarrow\uparrow}$,
so that we get $O^{\uparrow} = O^{\uparrow\downarrow\uparrow}$.
Likewise, $P^{\downarrow} = P^{\downarrow\uparrow\downarrow}$ is
shown. The claims now follow easily.
\proofend
%%
\begin{defn}
\index{concept}%%
Let $\auf A, B, R\zu$ be a context. A pair $\auf O, P\zu
\in \wp(A) \times \wp(B)$ is called a \textbf{concept} if
$O = P^{\downarrow}$ and $P = O^{\uparrow}$.
\end{defn}
%%
\begin{thm}
Let $\auf A, B, R\zu$ be a context. The concepts are exactly the
pairs of the form $\auf P^{\downarrow}, P^{\downarrow\uparrow}\zu$,
$P \subseteq B$, or, alternatively, the pairs of the form
$\auf O^{\uparrow\downarrow}, O^{\uparrow}\zu$, $O \subseteq A$.
\end{thm}
%%%
As a particular application we look again at the connection
between classes of $\Omega$--algebras and sets of equations
over $\Omega$--terms. (It suffices to take the set of
$\Omega$--algebras of size $< \kappa$ for a suitable $\kappa$
to make this work.) Let $\mathsf{Alg}_{\Omega}$ denote the
class of $\Omega$--algebras, $\mbox{\sf Eq}_{\Omega}$ the
set of equations. The triple $\auf \mathsf{Alg}_{\Omega},
\mbox{\sf Eq}_{\Omega}, \vDash\zu$ is a context, and the
map $^{\uparrow}$ is nothing but $\mbox{\sf Eq}$ and the map
$^{\downarrow}$ nothing but $\mathsf{Alg}$. The classes
$\mathsf{Alg}(E)$ are the equationally definable classes,
$\mathsf{Eq}(\CK)$ the equations valid in $\CK$. Concepts
are pairs $\auf \CK, E\zu$ such that $\CK = \mathsf{Alg}(E)$
and $E = \mathsf{Eq}(\CK)$.

Often we shall deal with structures in which there are
also relations in addition to functions. The definitions,
insofar as they still make sense, are carried over
analogously. However, the notation becomes more clumsy.
%%
\begin{defn}
Let $F$ and $G$ be disjoint sets and $\Omega \colon F \pf
\omega$ as well as $\Xi \colon G \pf \omega$ functions. A
pair $\GA = \auf A, \GI\zu$ is called an $\auf \Omega,
\Xi\zu$--\textbf{structure} 
%%%%
\index{structure}%%
%%%%
if for all $f \in F$ $\GI(f)$
is an $\Omega(f)$--ary function on $A$ and for each
$g \in G$ $\GI(g)$ is a $\Xi(g)$--ary relation
on $A$. $\Omega$ is called the 
%%%
\index{signature!functional}%%%
\index{signature!relational}%%%
%%%
\textbf{functional signature}, $\Xi$ the \textbf{relational signature}
of $\GA$.
\end{defn}
%%
Whenever we can afford it we shall drop the qualification `$\auf
\Omega, \Xi\zu$' and simply talk of `structures'. If $\auf A,
\GI\zu$ is an $\auf \Omega, \Xi\zu$--structure, then $\auf A, \GI
\restriction F\zu$ is an $\Omega$--algebra. An $\Omega$--algebra
can be thought of in a natural way as a $\auf \Omega,
\varnothing\zu$--structure, where $\varnothing$ is the empty
relational signature. We use a convention similar to that of
algebras. Furthermore, we denote relations by upper case Roman
letters such as $R$, $S$ and so on. Let $\GA = \auf A,
\{f^{\GA} : f \in F\}, \{R^{\GA} : R \in G\}\zu$ and $\GB = \auf
B, \{f^{\GB} : f \in F\}, \{R^{\GB} : R \in G\}\zu$ be structures
of the same signature. A map $h \colon A \pf B$ is called an 
\textbf{isomorphism} from $\GA$ to $\GB$, if $h$ is bijective and for all
$f \in F$ and all $\vec{x} \in A^{\Omega(f)}$ we have
%%
\begin{equation}
h(f^{\GA}(\vec{x})) = f^{\GB}(h(\vec{x})) 
\end{equation}
%%
as well as for all $R \in G$ and all $\vec{x} \in A^{\Xi(R)}$
%%
\begin{equation}
R^{\GA}(\vec{x}) \quad\Dpf\quad
R^{\GB}(h(x_0), h(x_1), \dotsc, h(x_{\Xi(R)-1})) 
\end{equation}
%%%
\vplatz
\exercise
Since $y \mapsto \{y\}$ is an embedding of $x$ into $\wp(x)$, we 
have $|x| \leq |\wp(x)|$. Show that $|\wp(x)| > |x|$ for every set. 
{\it Hint.} Let $f : x \pf \wp(x)$ be any function. Look at the set 
$\{y : y \not\in f(y)\} \subseteq x$. Show that it is not in 
$\im(f)$.
%%%
\vplatz
\exercise
Let $f \colon M \pf N$ and $g \colon N \pf P$. Show that if 
$g \circ f$ is surjective, $g$ is surjective, and that if $g \circ f$ 
is injective, $f$ is injective. Give in each case an example that the 
converse fails.
%%%
\vplatz
\exercise
In set theory, one writes ${^N}M$ for the set of functions
from $N$ to $M$. Show that if $|N| = n$ and $|M| = m$, then
$|{^N}M| = m^n$. Deduce that $|{^N}M| = |M^n|$. Can you find
a bijection between these sets?
%%%
\vplatz
\exercise
Show that for relations $R, R' \subseteq M \times N$,
$S, S' \subseteq N \times P$ we have
%%%
\begin{subequations}
\begin{align}
(R \cup R') \circ S & = (R \circ S) \cup (R' \circ S) \\
R \circ (S \cup S') & = (R \circ S) \cup (R \circ S')
\end{align}
\end{subequations}
%%%
Show by giving an example that analogous laws for $\cap$ do not hold.
%%%
\vplatz 
\exercise 
Let $\GA$ and $\GB$ be $\Omega$--algebras for
some signature $\Omega$. Show that if $h \colon \GA \epi \GB$ is a
surjective homomorphism then $\GB$ is isomorphic to $\GA/\Theta$
with $x \; \Theta\; y$ iff $h(x) = h(y)$.
%%%
\vplatz
\exercise
Show that every $\Omega$--algebra $\GA$ is the homomorphic image
of a term algebra. {\it Hint.} Take $X$ to be the set underlying
$\GA$.
%%
\vplatz
\exercise
Show that $\GA\times\GB$ is isomorphic to $\prod_{i \in \{0,1\}} \GA_i$,
where $\GA_0 = \GA$, $\GA_1 = \GB$. Show also that $(\GA\times\GB)
\times \GC$ is isomorphic to $\GA \times (\GB \times \GC)$.
%%%
\vplatz
\exercise
Prove Proposition~\ref{prop:free}.
