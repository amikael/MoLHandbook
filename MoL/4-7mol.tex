\section{Partiality and Discourse Dynamics}
\label{kap4x7}
%
%
%
After having outlined the basic setup of Montague Semantics, we
%%%
\index{Montague Semantics}%%%
%%%
shall deal with an issue that we have so far tacitly ignored, namely
{\it partiality}. The name `partial logic' covers a wide variety of 
logics that deal with radically different problems. We shall look at 
two of them. The first is that of partiality as undefinedness. The
second is that of partiality as ignorance. We start with
partiality as undefinedness.

Consider the assignment $y := (x+1)/(u^2 - 9)$ to $y$ in a program.
This clause is potentially dangerous, since $u$ may equal 3, in
which case no value can be assigned to $y$. Similarly, for a sequence
$\Ga = (a_n)_{n \in \omega}$, $\lim \Ga := \lim_{n \pf \infty} a_n$
is defined only if the series is convergent. If not, no value can be
given. Or in type theory, a function $f$ may only be applied to $x$ if
$f$ has type $\alpha\pf\beta$ for certain $\alpha$ and $\beta$ $x$ has
type $\alpha$. In the linguistic and philosophical literature, this
phenomenon is known as \textbf{presupposition}. It is defined as a 
relation between propositions (see \cite{vandersandt:presupposition}).
%%
\begin{defn}
%%%
\index{presupposition}%%
\index{$\gg_{\vdash}$}%%
%%%
A proposition $\varphi$ \textbf{presupposes} $\chi$ if both
$\varphi \vdash \chi$ and $\nicht\varphi \vdash \chi$.
We write $\varphi \gg_{\vdash} \chi$ (or simply $\varphi \gg
\chi$) to say that $\varphi$ presupposes $\chi$.
\end{defn}
%%
The definition needs only the notion of a negation in order to be
well--defined. Clearly, in boolean logic this definition gives
pretty uninteresting results. $\varphi$ presupposes $\chi$ in 
$\mathsf{PC}$ iff $\chi$ is a tautology. However, if we have more
than two truth--values, interesting results appear. First, notice
that we have earlier remedied partiality by assuming a `dummy'
element $\star$ that a function assumes as soon as it is not
defined on its regular input. Here, we shall remedy the situation
by giving the expression itself the truth--value $\star$. That is
to say, rather than making functions themselves total, we make the
assignment of truth--values a total function. This has different
consequences, as will be seen. Suppose that we totalize the
operator $\lim_{n \pf \omega}$ so that it can be applied to all
sequences. Then if $(a_n)_{n \in \omega}$ is not a convergent
series, say $a_n = (-1)^n$, $3 = \lim_{n \pf \infty} a_n$ is not
true, since $\lim_{n \pf \omega} a_n = \star$ and $3 \neq \star$.
The negation of the statement will then be true. This is
effectively what Russel~\shortcite{russell:denoting} and
%%%
\index{Russell, Bertrand}%%
%%%
Kempson~\shortcite{kempson:presupposition} 
%%%
\index{Kempson, Ruth}%%
%%%
claim. Now suppose we say
that $3 = \lim_{n \pf \infty} a_n$ has no truth--value; then $3
\neq \lim_{n \pf \infty} a_n$ also has no truth--value. To
nevertheless be able to deal with such sentences rather than
simply excluding them from discourse, we introduce a third 
truth--value, $\star$. The question is now: how do we define the
3--valued counterparts of $-$, $\cap$ and $\cup$? In order to keep
confusion at a minimum, we agree on the following conventions.
%%%
\index{$\vdash_3$}%%%
%%%
$\vdash_3$ denotes the 3--valued consequence relation determined
by a matrix $\GM = \auf \{0,1,\star\}, \Pi, \{1\}\zu$, where $\Pi$
is an interpretation of the connectives. We shall assume that $F$
consists of a subset of the set of the 9 unary and 27 binary
symbols, which represent the unary and binary functions on the
three element set. This defines $\Omega$. Then $\vdash_3$ is
uniquely fixed by $F$, and the logical connectives will receive a
distinct name every time we choose a different function on
$\{0,1,\star\}$. What remains to be solved, then, is not what
logical language to use but rather by what connective to translate
the ordinary language connectors {\tt not}, {\tt and}, {\tt or},
and {\tt if$\dotsb$then}. Here, we assume that whatever interprets
them is a function on $\{0,1,\star\}$ (or $\{0,1,\star\}^{2}$),
whose restriction to $\{0,1\}$ is its boolean counterpart, which
is already given. For those functions, the 2--valued consequence
is also defined and denoted by $\vdash_2$.
%%%%
\index{$\vdash_2$}%%%
%%%

Now, if $\star$ is the truth--value reserved for the otherwise truth
valueless statements, we get the following three valued logic, 
due to Bochvar~\shortcite{bochvar:three}. Its characteristics is the
fact that undefinedness is strictly hereditary.
%%
\begin{equation}
\begin{array}{l|l}
        & - \\\hline
0       & 1 \\
1       & 0 \\
\star   &  \star
\end{array}
\qquad
\begin{array}{l|lll}
\cap    & 0     & 1     & \star \\\hline
0       & 0     & 0     & \star \\
1       & 0     & 1     & \star \\
\star   & \star & \star & \star
\end{array}
\qquad
\begin{array}{l|lll}
\cup    & 0     & 1     & \star \\\hline
0       & 0     & 1     & \star \\
1       & 1     & 1     & \star \\
\star   & \star & \star & \star
\end{array}
\end{equation}
%%%
The basic connectives are $\nicht$, $\und$ and $\oder$, which
are interpreted by $-$, $\cap$ and $\cup$. Here is a characterization
of presupposition in Bochvar's Logic. Call a connective
%%%
\index{connective!Bochvar}%%
%%%%
\ding{67} a \textbf{Bochvar}--\textbf{connective} if 
$\Pi(\mbox{\ding{67}})(\vec{x}) = \star$ iff $x_i = \star$ 
for some $i < \Omega(\mbox{\ding{67}})$.
%%
\begin{prop}
Let $\Delta$ and $\chi$ be composed using only Boch\-var--con\-nec\-tives.
Then $\Delta \vdash_3 \chi$ iff (i) $\Delta$ is not classically satisfiable 
or (ii) $\Delta \vdash_2 \chi$ and $\var(\chi) \subseteq \var[\Delta]$.
\end{prop}
%%
\proofbeg
Suppose that $\Delta \vdash_3 \chi$ and that
$\Delta$ is satisfiable. Let $\beta$ be a valuation such that
$\oli{\beta}(\delta) = 1$ for all $\delta \in \Delta$. Put
$\beta^+(p) := \beta(p)$ for all $p \in \var(\Delta)$
and $\beta^+(p) := \star$ otherwise. Suppose that
$\var(\chi) - \var(\Delta) \neq \varnothing$.
Then $\oli{\beta^+}(\chi) = \star$, contradicting our assumption.
Hence, $\var(\chi) \subseteq \var(\Delta)$.
It follows that every valuation that satisfies $\Delta$ also
satisfies $\chi$, since the valuation does not assume $\star$
on its variables (and can therefore be assumed to be a classical
valuation). Now suppose that $\Delta \nvdash_3 \chi$.
Then clearly $\Delta$ must be satisfiable. Furthermore, by the
argument above either $\var(\chi) - \var(\Delta)
\neq \varnothing$ or else $\Delta \vdash_2 \chi$.
\proofend
%%%

This characterization can be used to derive the following corollary.
%%%
\begin{cor}
Let $\varphi$ and $\chi$ be composed by Bochvar--connectives.
Then $\varphi \gg \chi$ iff $\var(\chi) \subseteq \var(\varphi)$ 
and $\vdash_2 \chi$.
\end{cor}
%%%
Hence, although Bochvar's logic makes room for undefinedness, the
notion of presupposition is again trivial. Bochvar's Logic seems
nevertheless adequate as a treatment of the $\iota$--operator. It
is formally defined as follows.
%%
\begin{defn}
$\iota$ is a partial function from predicates to objects such that
$\iota x.\chi(x)$ is defined iff there is exactly one
$b$ such that $\chi(b)$, and in that case $\iota x. \chi(x) := b$.
\end{defn}
%%
Most mathematical statements which involve presuppositions are instances
of a (hidden) use the $\iota$--operator. Examples are the derivative, 
the integral and the limit. In ordinary language, $\iota$ corresponds 
to the definite
determiner {\tt the}. Using the $\iota$--operator, we can bring out
the difference between the bivalent interpretation and the three
valued one. Define the predicate $\textsf{cauchy}'$ on infinite
sequences of real numbers as follows:
%%%
\begin{equation}
\textsf{cauchy}'(\Ga) :=
    (\forall \varepsilon > 0)(\exists n)(\forall m \geq n)
    |\Ga(m) - \Ga(n)| < \varepsilon
\end{equation}
%%%
This is in formal terms the definition of a Cauchy sequence.
Further, define a predicate $\textsf{cum}'$ as follows.
%%%
\begin{equation}
\textsf{cum}'(\Ga)(x) := (\forall \varepsilon > 0)(\exists n)%
    (\forall m \geq n)|\Ga(m) - x| < \varepsilon
\end{equation}
%%%
This predicate says that $x$ is a cumulation point of $\Ga$.
Now, we may set $\lim \Ga := \iota x.\textsf{cum}'(\Ga)(x)$.
Notice that $\textsf{cauchy}'(\Ga)$ is equivalent to
%%%
\begin{equation}
(\exists x)(\textsf{cum}'(\Ga)(x) \und (\forall y)(\textsf{cum}'(\Ga)(y) 
\pf y \doteq x)) 
\end{equation}
%%%
This is exactly what must be true for $\lim \Ga$ to be defined.
%%%
\begin{align}
\label{eq:471} & \mbox{\tt The limit of $\Ga$ equals three.} \\
\label{eq:472} & \iota x.\textsf{cum}'(\Ga)(x) \doteq 3 \\
\label{eq:473} & (\exists x)(\textsf{cum}'(\Ga)(x) \und
    (\forall y)(\textsf{cum}'(\Ga)(y) \pf y \doteq x)
    \und x \doteq 3).
\end{align}
%%%
Under the analysis \eqref{eq:472} the sentence \eqref{eq:471}
presupposes that $\Ga$ is a Cauchy--sequence. \eqref{eq:473} does
not presuppose that. However, the dilemma for the translation
\eqref{eq:473} is that the negation of \eqref{eq:471} is also
false (at least in ordinary judgement). What this means is that
the truth--conditions of \eqref{eq:474} are not expressed by 
\eqref{eq:476}, but by \eqref{eq:477} which in three valued 
logic is \eqref{eq:475}.
%%%
\begin{align}
\label{eq:474} & \mbox{\tt The limit of $\Ga$ does not equal three.} \\
\label{eq:475} & \nicht (\iota x.\textsf{cum}'(\Ga)(x) \doteq 3) \\
\label{eq:476} & \nicht (\exists x)(\textsf{cum}'(\Ga)(x) \und
    (\forall y)(\textsf{cum}'(\Ga)(y) \pf y \doteq x)
    \und x \doteq 3) \\
\label{eq:477} & (\exists x)(\textsf{cum}'(\Ga)(x) \und
    (\forall y)(\textsf{cum}'(\Ga)(y) \pf y \doteq x)
    \und \nicht(x \doteq 3))
\end{align}
%%%
It is difficult to imagine how to get the translation \eqref{eq:477} 
in a bivalent approach, although a proposal is made below. The problem 
with a bivalent analysis is that it can be shown to be inadequate, 
because it rests on the assumption that the primitive predicates are
bivalent. However, this is problematic. The most clear--cut case is that 
of the truth--predicate. Suppose we define the semantics of $\mathsf{T}$ 
on the set of natural numbers as follows.
%%
\begin{equation}
\label{eq:47dagger}
\mathsf{T}(\ulcorner \varphi \urcorner) \dpf \varphi
\end{equation}
%%
Here, $\ulcorner \varphi \urcorner$ is, say, the G\"odel code of
$\varphi$. It can be shown that there is a $\chi$ such
that $\mathsf{T}(\ulcorner \chi \urcorner) \dpf \nicht\chi$ is
true in the natural numbers. This contradicts 
\eqref{eq:47dagger}. The sentence $\chi$ corresponds to the 
following liar paradox.
%%
\begin{align}
\label{eq:478} & \mbox{\tt This sentence is false.}
\end{align}
%%
Thus, as Tarski has observed, a truth--predicate that is consistent
with the facts in a sufficiently rich theory must be partial. As
sentence \eqref{eq:478} shows, natural languages are sufficiently
rich to produce the same effect. Since we do not want to give up
the correctness of the truth--predicate (or the falsity predicate),
the only alternative is to assume that it is partial. If so, however,
there is no escape from the use of three valued logic, since
bivalence must fail.

Let us assume therefore that we three truth--values. What Bochvar's
logic gives us is called the logic of hereditary undefinedness.
For many reasons it is problematic, however. Consider the
following two examples.
%%%
\begin{align}
\label{eq:479} & \mbox{\tt If $\Ga$ and $\Gb$ are convergent sequences,
    $\lim (\Ga + \Gb)$} \\\notag
        & \quad \mbox{\tt $= \lim \Ga + \lim \Gb$.} \\
\label{eq:4710} & \textsf{if $u \neq 3$ then $y := (x+1)/(u^2 - 9)$
    else $y := 0$ fi}
\end{align}
%%%
By Bochvar's Logic, \eqref{eq:479} presupposes that
$\Ga$, $\Gb$ and $\Ga + \Gb$ are convergent series. \eqref{eq:4710}
presupposes that $u \neq 3$. However, none of
the two sentences have nontrivial presuppositions. Let us illustrate
this with \eqref{eq:479}. Intuitively, the if--clause preceding the
equality statement excludes all sequences from consideration
where $\Ga$ and $\Gb$ are nonconvergent sequences. One can show
that $\Ga + \Gb$, the pointwise sum of $\Ga$ and $\Gb$, is then
also convergent. Hence, the if--clause covers all cases of
partiality. The statement $\lim (\Ga + \Gb) = \lim \Ga + \lim \Gb$
never fails. Similarly, {\tt and} has the power to eliminate
presuppositions.
%%%
\begin{align}
\label{eq:4711} & \mbox{\tt $\Ga$ and $\Gb$ are convergent series
    and $\lim (\Ga + \Gb)$} \\\notag
        & \quad \mbox{\tt $= \lim \Ga + \lim \Gb$.} \\
\label{eq:4712} & u:= 4; y := (x+1)/(u^2 - 9)
\end{align}
%%%
As it turns out, there is an easy fix for that. Simply associate
the following connectives with {\tt if$\dotsb$then} and {\tt and}.
%%
\index{$\und'$, $\oder'$, $\pf'$}%%%
\begin{equation}
\begin{array}{l|lll}
\pf' &  0    &  1    & \star  \\\hline
0     &  1    &  1    &  1    \\
1     &  0    &  1    & \star \\
\star & \star & \star & \star
\end{array}
\qquad
\begin{array}{l|lll}
\und' &  0    &  1    & \star \\\hline
0     &  0    &  0    &  1    \\
1     &  0    &  1    & \star \\
\star & \star & \star & \star
\end{array}
\qquad
\begin{array}{l|lll}
\oder'  &  0    &  1    & \star    \\\hline
0       &  0    &  1    & \star    \\
1       &  1    &  1    & 1        \\
\star   & \star & \star & \star
\end{array}
\end{equation}
%%
The reader may take notice of the fact that while $\und'$ and
$\pf'$ are reasonable candidates for {\tt and} and 
{\tt if$\dotsb$then}, $\oder'$ is not as good for {\tt or}.

In the linguistic literature, various attempts have been made to
explain these facts. First, we distinguish the {\it presupposition\/} 
of a sentence from its {\it assertion}. The definition of
these terms is somewhat cumbersome. The general idea is that the
presupposition of a sentence is a characterization of those
circumstances under which it is either true or false, and the
assertion is what the sentence says when it is either true of
false (that is to say, the assertion tells us when the sentence is
true given that it is either true or false). Let us attempt to define
this.  Let $\varphi$ be a proposition. Call $\chi$ a 
%%%
\index{presupposition!generic}%%
%%%%
\textbf{generic presupposition of} $\varphi$ if the following holds. 
(a) $\varphi \gg \chi$, (b) if $\varphi \gg \psi$ then 
$\chi \vdash_3 \psi$.  If $\chi$ is a generic presupposition of 
$\varphi$, $\chi \pf \varphi$ is called an 
%%%
\index{assertion}%%
%%%
\textbf{assertion} of $\varphi$. First, notice
that presuppositions are only defined up to interderivability.
This is not a congruence. We may have $\varphi {\dashv\vdash}_3
\chi$ without $\varphi$ and $\chi$ receiving the same truth--value
under all assignments. Namely, $\varphi {\dashv\vdash}_3 \chi$ 
iff $\varphi$ and $\chi$ are truth--equivalent, that is,
$\oli{\beta}(\varphi) = 1$ exactly when $\oli{\beta}(\chi) = 1$.
In order to have full equivalence, we must also require
$\nicht \varphi {\dashv\vdash}_3 \nicht \chi$. Second, notice 
that $\varphi \oder \nicht \varphi$ satisfies (a) and (b).
However, $\varphi\oder\nicht\varphi$ presupposes itself, 
something that we wish to avoid. 
%%%
\index{bivalence}%%%
%%%%
So, we additionally require the generic presupposition to be 
\textbf{bivalent}. Here, $\varphi$ is \textbf{bivalent} if for 
every valuation $\beta$ into $\{0,1,\star\}$: $\oli{\beta}(\varphi) 
\in \{0,1\}$. Define the following connective.
%%
\begin{equation}
%%%
\index{$\downarrow$}%%%
%%%%
\begin{array}{l|lll}
\downarrow & 0     & 1     & \star \\\hline
0          & \star & 0     & \star \\
1          & \star & 1     & \star \\
\star      & \star & \star & \star
\end{array}
\end{equation}
%%
\begin{defn}
%%%
\index{presupposition!generic}%%
\index{assertion}%%
%%%
Let $\varphi$ be a proposition. $\chi$ is a \textbf{generic
presupposition of} $\varphi$ \textbf{with respect to} $\vdash_3$ if
(a) $\varphi \gg \chi$, (b) $\chi$ is bivalent and (c) if $\varphi
\gg \psi$ then $\chi \vdash_3 \psi$. $\chi$ is the \textbf{assertion}
of $\varphi$ if (a) $\chi$ is bivalent and (b) $\chi \dashv\vdash_3
\varphi$. Write $P(\varphi)$ for the generic presupposition (if it
exists), and $A(\varphi)$ for the assertion.
\end{defn}
%%%
It is not a priori clear that a proposition has a generic
presupposition. A case in point is the truth--predicate.
%%%
\index{$\equiv_3$}%%%
%%%%
Write $\varphi \equiv_3 \chi$ if $\beta(\varphi) = \beta(\chi)$ 
for all $\beta$.
%%%
\begin{prop}
$\varphi \equiv_3 \; {A(\varphi)\downarrow P(\varphi)}$.
\end{prop}
%%%
\index{projection algorithm}%%%
%%%%
The \textbf{projection algorithm} is a procedure that assigns generic
presuppositions to complex propositions by induction over their
structure. Table~\ref{tab:projection} shows a projection algorithm
for the connectives defined so far.
%%%
\begin{table}
\caption{The Projection Algorithm}%%
\label{tab:projection}
$$\begin{array}{l@{\; = \;}l@{\qquad}l@{\; = \;}l}
A(\nicht \varphi) & \nicht A(\varphi) & P(\nicht \varphi) &
        P(\varphi) \\
A(\varphi\und\chi) & A(\varphi)\und A(\chi) &
    P(\varphi\und\chi) & P(\varphi) \und P(\chi) \\
A(\varphi\oder\chi) & A(\varphi)\oder A(\chi) &
    P(\varphi\oder\chi) & P(\varphi) \und P(\chi) \\
A(\varphi\pf\chi) & A(\varphi) \pf A(\chi) &
    P(\varphi\pf\chi) & P(\varphi) \und P(\chi) \\
A(\varphi\und'\chi) & A(\varphi)\und A(\chi) &
    P(\varphi\und'\chi) & P(\varphi) \\
    \multicolumn{2}{r}{} &
    \multicolumn{2}{r}{\und (A(\varphi) \pf P(\chi))} \\
A(\varphi\oder'\chi) & A(\varphi) \oder A(\chi) &
    P(\varphi\oder'\chi) & P(\varphi) \\
    \multicolumn{2}{r}{} &
    \multicolumn{2}{r}{\und (\nicht A(\varphi) \pf P(\chi))} \\
A(\varphi\pf'\chi) & (A(\varphi)\und P(\varphi)) &
    P(\varphi\pf'\chi) & P(\varphi) \\
    \multicolumn{2}{r}{\pf A(\chi)} & \multicolumn{2}{r}{\und
    (A(\varphi) \pf P(\chi))}
\end{array}$$
\end{table}
%%%%
It is an easy matter to define projection algorithms for all
connectives. The prevailing intuition is that the three valued
character of {\tt and}, {\tt or} and {\tt if$\dotsb$then}
%%%%
\index{context change}%%%
%%%
is best explained in terms of \textbf{context change}. A
%%%
\index{text}%%
%%%
\textbf{text} is a sequence of propositions, say $\Delta =
\auf \delta_i : i < n\zu$. A text is \textbf{coherent}
%%%
\index{text!coherent}%%
%%%
if for every $i < n$: $\auf \delta_j : j < i\zu \vdash_3
\delta_i \oder \nicht\delta_i$. In other words, every member
is either true or false given that the previous propositions are
considered true. (Notice that the order is important now.)
In order to extend this to parts of the $\delta_j$
we define the \textbf{local context} as follows.
%%%
\begin{defn}
%%%
\index{context!local}%%
%%%%
Let $\Delta = \auf \delta_i : i < n\zu$. The \textbf{local context} of
$\delta_j$ is $\auf \delta_i : i < j\zu$. For a subformula occurrence
of $\delta_j$, the \textbf{local context} of that occurrence is defined
as follows.
%%%
\begin{dingautolist}{192}
\item If $\Sigma$ is the local context of $\varphi\und'\chi$ then
    (a) $\Sigma$ is the local context of $\varphi$ and (b)
    $\Sigma;\varphi$ is the local context of $\chi$.
\item If $\Sigma$ is the local context of $\varphi\pf'\chi$ then
    (a) $\Sigma$ is the local context of $\varphi$ and (b)
    $\Sigma;\varphi$ is the local context of $\chi$.
\item If $\Sigma$ is the local context of $\varphi\oder'\chi$ then
    (a) $\Sigma$ is the local context of $\varphi$ and (b)
    $\Sigma;\nicht\varphi$ is the local context of $\chi$.
\end{dingautolist}
%%%
\index{bivalence}%%%%%
%%%%
$\delta_j$ is \textbf{bivalent in its local context} if for all
valuations that make all formulae in the local context true,
$\delta_j$ is true or false.
\end{defn}
%%%
The presupposition of $\varphi$ is the formula $\chi$ such that
$\varphi$ is bivalent in the context $\chi$, and which implies all
other formulae that make $\varphi$ bivalent. It so turns out that
the context dynamics define a three valued extension of a
2--valued connective, and conversely. The above rules are an exact
match. Such formulations have been given by Irene
Heim~\shortcite{heim:context}, %%
%%%
\index{Heim, Irene}%%%
%%%
Lauri Karttunen~\shortcite{karttunen:context} 
%%%
\index{Kartttunen, Lauri}%%%
%%%
and also Jan van Eijck~\shortcite{vaneijck:presupposition}. 
%%%
\index{van Eijck, Jan}%%%
%%%
It is easy to understand 
this in computer programs. A computer program may contain clauses 
carrying presuppositions (for example clauses involving divisions), 
but it need not fail. For if whenever a clause carrying a 
presupposition is evaluated, that presupposition 
is satisfied, no error ever occurs at runtime. In
other words, the local context of that clause satisfies the
presuppositions. What the local context is, is defined by the
evaluation procedure for the connectives. In computer languages, 
the local context is always to the left. But this is not necessary.
The computer evaluates $\alpha\; \mathsf{and}\; \beta$ by first
evaluating $\alpha$ and then $\beta$ only if $\alpha$ is true --- 
but it could also evaluate $\beta$ first and then evaluate $\alpha$ whenever 
$\beta$ is true. In \cite{kracht:control} it is shown
that in the definition of the local context all that needs to be
specified is the directionality of evaluation. The rest follows 
from general principles. Otherwise one gets connectives that extend 
the boolean connectives in an improper way (see below on that notion).

The behaviour of presuppositions in quantification and propositional 
attitude reports is less straightforward. We shall only give a sketch.
%%%
\begin{align}
\label{eq:4713} & \mbox{\tt Every bachelor of the region 
	got a letter from} \\\notag
   & \quad \mbox{\tt that marriage agency.} \\
\label{eq:4714} & \mbox{\tt Every person in that region is a bachelor.} \\
\label{eq:4715} & \mbox{\tt John believes that his neighbour is a
    bachelor.}
\end{align}
%%%
We have translated {\tt every} using the quantifier $\forall$ in 
predicate logic. We wish to extend it to a three--valued quantifier, 
which we also call $\forall$. \eqref{eq:4713} is true even if not 
everybody in the region is a bachelor; in fact, it is true exactly 
if there is no non--bachelor. Therefore we say that $(\forall x)\varphi$ 
is true iff there is no $x$ for which $\varphi(x)$ is false. 
$(\forall x)\varphi$ is false if there is an $x$ for which 
$\varphi(x)$ is false. Thus, the presupposition effectively 
restricts the range of the quantifier.  $(\forall x)\varphi$ 
is bivalent. This predicts that \eqref{eq:4714}
has no presuppositions. On \eqref{eq:4715} the intuitions vary.
One might say that it does not have any presuppositions, or else
that it presupposes that the neighbour is a man (or perhaps: that
John believes that his neighbour is a man). This is deep water
(see \cite{geurts:presupposition}).

Now we come to the second interpretation of partiality, namely
{\it ignorance}. Let $\star$ now stand for the fact that the
truth--value is not known. Also here the resulting logic is not
unique. Let us take the example of a valuation $\beta$ that is
only defined on some of the variables. Now let us be given
a formula $\varphi$. Strictly speaking $\oli{\beta}(\varphi)$
is not defined on $\varphi$ if the latter contains a variable
that is not in the domain of $\beta$. On the other hand, there
are clear examples of propositions that receive a definite
truth--value no matter how we extend $\beta$ to a total function.
For example, even if $\beta$ is not defined on $p$, every extension
of it must make $p \oder \nicht p$ true. Hence, we might say
that $\beta$ also makes $p \oder \nicht p$ true. This is the idea
%%%
\index{supervaluation}%%
%%%%
of \textbf{supervaluations} by Bas van Fraassen. 
%%%
\index{van Fraassen, Bas}%%%
%%%
Say that
$\varphi$ is \textbf{sv--true} (\textbf{sv--false}) under $\beta$ if
$\varphi$ is true under every total $\gamma \supseteq \beta$.
If $\varphi$ is neither sv--true nor sv---false, call it
\textbf{sv--indeterminate}. Unfortunately, there is no logic
to go with this approach. Look at the interpretation of {\tt or}.
Clearly, if either $\varphi$ or $\chi$ is sv--true, so is
their disjunction. However, what if both $\varphi$ and $\chi$
are sv--indeterminate?
%%
\begin{equation}
\begin{array}{l|lll}
\cup  & 0     & 1 & \star \\\hline
0     & 0     & 1 & \star \\
1     & 1     & 1 & 1     \\
\star & \star & 1 & ?
\end{array}
\end{equation}
%%
The formula $p \oder q$ is sv--indeterminate under the empty valuation.
It has no definite truth--value, because both $p$ and $q$ could turn out
to be either true or false. On the other hand, $p \oder \nicht p$ is
sv--true under the empty valuation, even though both $p$ and $\nicht p$
are sv--indeterminate. So, the supervaluation approach is not so
well--suited. Stephen Kleene actually had the idea of doing a worst
case interpretation: if you can't always say what the value is, fill
%%%
\index{Kleene, Stephen C.}%%%
\index{connective!strong Kleene}%%%
%%%
in $\star$. This gives the so--called \textbf{Strong Kleene Connectives}
(the weak ones are like Bochvar's).
%%
\index{$\cap^{\diamond}$, $\cup^{\diamond}$}%%%
%%
\begin{equation}
\begin{array}{l|l}
        & - \\\hline
0       & 1 \\
1       & 0 \\
\star   &  \star
\end{array}
\qquad
\begin{array}{l|lll}
\cap^{\diamond}    & 0     & 1     & \star \\\hline
0         & 0     & 0     & 0     \\
1         & 0     & 1     & \star \\
\star     & 0     & \star & \star
\end{array}
\qquad
\begin{array}{l|lll}
\cup^{\diamond}  & 0     & 1     & \star \\\hline
0       & 0     & 1     & \star \\
1       & 1     & 1     & 1     \\
\star   & \star & 1     & \star
\end{array}
%%\qquad
%\begin{array}{l|lll}
%\supset^{\diamond}  & 0     & 1     & \star \\\hline
%0          & 1     & 1     & 1     \\
%1          & 0     & 1     & \star \\
%\star      & \star & 1     & \star
%\end{array}
\end{equation}
%%%
These connectives can be defined in the following way.
Put $1^{\diamond} := \{1\}$, $0^{\diamond} := \{0\}$
and $\star := \{0,1\}$. Now put
%%
\begin{equation}
f^{\diamond}(x_0^{\diamond},x_1^{\diamond}) :=
f[x_0^{\diamond} \times x_1^{\diamond}]
\end{equation}
%%
For example, $\cup^{\diamond}(\auf 0^{\diamond},\star^{\diamond}\zu)
= \cup[\{0\} \times \{0,1\}] = \{0 \cup 0, 0 \cup 1\} = \{0,1\} =
\star^{\diamond}$. So, we simply take sets of truth--values and
calculate with them.

A more radical account of ignorance is presented by constructivism 
and intuitionism. A constructivist denies that the truth or falsity 
of a statement can always be assessed directly. In particular, an 
existential statement is true only if we produce an instance that 
satisfies it. A universal statement can be considered true only if we
possess a proof that any given element satisfies it. For example,
Goldbach's conjecture is that every even number greater than 2 is
the sum of two primes. According to a constructivist, at present
it is neither true nor false. For on the one hand we have no proof
that it holds, on the other hand we know of no even number greater
than 2 which is not the sum of two primes. Both
constructivists and intuitionists unanimously reject axiom (a2).
(Put $\bot$ for $p_1$. Then in conjunction with the other rules 
this gives $(\nicht p_0 \pf p_0) \pf p_0$. This corresponds to 
the \textbf{Rule of Clavius}: from $\nicht p_0 \pf p_0$ conclude 
$p_0$.) They also reject $p \oder \nicht p$, the so--called 
%%%
\index{law of the excluded middle}%%
%%%
\textbf{Law of the Exluded Middle}. The difference between a 
constructivist and an intuitionist is the treatment of negative 
evidence. While a
constructivist accepts basic negative evidence, for example, that
this lemon is not green, for an intuitionist there is no such
thing as direct evidence to the contrary. We only witness the
absence of the fact that the lemon is green. Both, however, are
reformist in the sense that they argue that the mathematical
connectives {\tt and}, {\tt or}, {\tt not}, and {\tt if$\dotsb$then} 
have a different meaning. However, one can actually give a
reconstruction of both inside classical mathematics. We shall deal
first with intuitionism. Here is a new set of connectives, 
defined with the help of $\qu$, which satisfies $\mathsf{S4}$.
%%
\begin{equation}
\begin{split}
\nicht^i \varphi & := \qu (\nicht \varphi) \\
\varphi \oder^i \chi & := \varphi \oder \chi \\
\varphi \und^i \chi & := \varphi \und \chi \\
\varphi \pf^i \chi & := \qu (\varphi \pf \chi)
\end{split}
\end{equation}
%%
Call an I--proposition a proposition formed from variables and
$\bot$ using only the connectives just defined.
%%
\begin{defn}
%%%
\index{I--model}%%
%%%
An \textbf{I--model} is a pair $\auf P, \leq, \beta\zu$, where
$\auf P, \leq\zu$ is a partially ordered set and $\beta(p) =
\, \uparrow\!\beta(p)$ for all variables $p$.
\end{defn}
%%
Intuitively, the nodes of $P$ represent stages in the development
of knowledge. Knowledge develops in time along $\leq$. We say that
$x$ \textbf{accepts} $\varphi$ if $\auf P, \leq, x, \beta\zu \vDash
\varphi$, and that $x$ \textbf{knows} $\varphi$ if $\auf P, \leq,
x, \beta\zu \vDash \qu \varphi$. By definition of $\beta$, once a
proposition $p$ is accepted, it is accepted for good and therefore
considered known. Therefore G\"odel simply translated variables $p$ 
by the formula $\qu p$. Thus, intuitionistically the statement that 
$p$ may therefore be understood as `$p$ is known' rather than `$p$ 
is accepted'. The systematic conflation of knowledge and simple 
temporary acceptance as true is the main feature of intuitionistic 
logic.
%%
\begin{prop}
Let $\auf P, \leq, \beta\zu$ be an I--model and $\varphi$ an
I--pro\-po\-si\-tion. Then $\oli{\beta}(\varphi) =\; {\uparrow
\oli{\beta}(\varphi)}$ for all $\varphi$.
\end{prop}
%%
Constructivism in the definition by Nelson adds to intuitionism
a second valuation for those variables that are definitely
rejected, and allows for the possibility that neither is the
case. (However, nothing can be both accepted and rejected.)
This is reformulated as follows.
%%
\begin{defn}
%%%
\index{C--model}%%
%%%
A \textbf{C--model} is a pair $\auf P, \leq, \beta\zu$, where
$\auf P, \leq\zu$ is a partially ordered set and $\beta \colon V \times P
\pf \{0,1,\star\}$ such that if $\beta(p,v) = 1$ and $v \leq w$
then also $\beta(p,w) = 1$, and if $\beta(p,v) = 0$ and $v  \leq w$
then $\beta(p,w) = 0$. We write $\auf P, \leq, x, \beta\zu \vDash^+
p$ if $\beta(p,x) = 1$ and $\auf P, \leq, x, \beta\zu \vDash^- p$
if $\beta(p,x) = 0$.
\end{defn}
%%
We can interpret any propositional formula over 3--valued logic that
we have defined so far. We have to interpret $\qu$ and $\wD$,
however.
%%
\begin{equation}
\begin{split}
x \vDash^+ \qu \varphi & :\Dpf \text{ for no }
    y \geq x: y\vDash^- \varphi \\
x \vDash^- \qu \varphi & :\Dpf \text{ there is }
    y \geq x: y \vDash^- \varphi \\
x \vDash^+ \wD \varphi & :\Dpf \text{ there is }
    y \geq x: y\vDash^+ \varphi \\
x \vDash^- \wD \varphi & :\Dpf \text{ for no }
    y \geq x: y\vDash^+ \varphi
\end{split}
\end{equation}
%%
Now define the following new connectives.
%%
\begin{equation}
\begin{split}
\nicht^c \varphi & := \nicht \varphi \\
\varphi \oder^c \chi & := \varphi \oder^{\diamond} \chi \\
\varphi \und^c \chi & := \varphi \und^{\diamond} \chi \\
\varphi \pf^c \chi & := \qu (\varphi \pf^{\diamond} \chi)
\end{split}
\end{equation}
%%
In his data semantics (see \cite{veltman:conditionals}), Frank
Veltman 
%%%
\index{Veltman, Frank}%%%
%%%
uses constructive logic and proposes to interpret {\tt
must} and {\tt may} as $\qu$ and $\wD$, respectively. What is
interesting is that the set of points accepting $\wD \varphi$ is
lower closed but {\it not\/} necessarily upper closed, while the
set of points rejecting it is upper but not necessarily lower
closed. The converse holds with respect to $\qu$. This is natural,
since if our knowledge grows there are less things that {\it
may\/} be true but more that {\it must\/} be.

The interpretation of the arrow carries the germ of the relational
interpretation discussed here. A different strand of thought is
the theory of conditionals (see again \cite{veltman:conditionals}
and also \cite{gaerdenfors:flux}). The conditional $\varphi >
\chi$ is accepted as true under Ramsey's interpretation 
%%%
\index{Ramsey, Frank}%%%
%%%
if, on taking $\varphi$ as a hypothetical assumption (doing as if
$\varphi$ is the case) and performing the standard reasoning, we
find that $\chi$ is true as well. Notice that after this routine
of hypothetical reasoning we retract the assumption that
$\varphi$. In G\"ardenfors models there are no assignments in the
ordinary sense. A proposition $\varphi$ is mapped directly onto a
function, the update function $U_{\varphi}$. The states in the
G\"ardenfors model carry no structure.
%%%
\begin{defn}
%%%
\index{G\"ardenfors model}%%
%%%%
\index{G\"ardenfors, Peter}%%%
%%%
A \textbf{G\"ardenfors model} is a pair $\auf G, U\zu$, where
$G$ is a set, and $U \colon \Tm_{\Omega} \pf G^G$ subject to the
following constraints.
%%%
\begin{dingautolist}{192}
\item For all $\chi$: $U_{\chi} \circ U_{\chi} = U_{\chi}$.
\item For all $\varphi$ and $\chi$:
    $U_{\varphi} \circ U_{\chi} = U_{\chi} \circ U_{\varphi}$.
\end{dingautolist}
%%%
We say that $x \in G$ \textbf{accepts} $\varphi$ if $U_{\varphi}(x) = x$.
%%%
\end{defn}
%%%
Put $x \leq y$ iff
there is a finite set $\{\chi_i : i < n\}$ such that 
%%%
\begin{equation}
y = U_{\chi_0} \circ U_{\chi_1} \circ \dotsb \circ U_{\chi_{n-1}}(x)
\end{equation}
%%%
The reader may verify that this relation is reflexive and transitive.
If we require that $x = y$ iff $x$ and $y$ accept the
same propositions, then this ordering is also a partial ordering.
We can define as follows. If $U_{\chi} = U_{\varphi} \circ
U_{\psi}$ then we write $\chi = \varphi\und\psi$. Hence, if our
language actually has a conjunction, the latter is a condition on
the interpretation of it. To define $\pf$, G\"ardenfors does the
following. Suppose that for all $\varphi$ and $\chi$ there exists
a $\delta$ such that $U_{\varphi} \circ U_{\delta} = U_{\chi}
\circ U_{\delta}$. Then we simply put $U_{\varphi \dpf \psi}
:= U_{\delta}$. Finally, since $\varphi\pf\psi$ is equivalent to
$\varphi\dpf \varphi\und \psi$, once we have $\und$ and $\dpf$, we
can also define $\pf$. For negation we need to assume the
existence of an inconsistent state. The details need not concern
us here. Obviously, G\"ardenfors models are still more general
than data semantics. In fact, any kind of logic can be modelled by
a G\"ardenfors model (see the exercises).

{\it Notes on this section.} It is an often discussed problem whether
or not a statement of the form $(\forall x)\varphi$ is true if there
are no $x$ at all. Equivalently, in three valued logic, it might be
said that $(\forall x)\varphi$ is undefined if there is no $x$ such
that $\varphi(x)$ is defined.
%%%
\vplatz
\exercise
A three valued binary connective \ding{67} satisfies the
%%%
\index{Deduction Theorem}%%%
%%%%
Deduction Theorem if for all $\Delta$, $\varphi$ and
$\chi$: $\Delta; \varphi \vdash_3 \chi$ iff
$\Delta \vdash_3 \varphi \mbox{\ding{67}} \chi$. Establish all 
truth--tables for binary connectives that satisfy the Deduction 
Theorem. Does any of the implications defined above have this 
property?
%%%
\vplatz \exercise Let $\CL$ be a language and $\vdash$ a
structural consequence relation over $\CL$. Let $G_{\vdash}$ be
the set of theories of $\vdash$. For $\varphi \in \CL$, let
$U_{\varphi} \colon T \mapsto (T \cup \{\varphi\})^{\vdash}$. Show that
this is a G\"ardenfors model. Show that the set of formulae
accepted by all $T \in G_{\vdash}$ is exactly the set of
tautologies of $\vdash$.
