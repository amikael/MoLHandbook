\section{Parikh's Theorem}
\label{kap2-5}
%%%
Now we shall turn to the already announced embedding of context
free tree sets into tree sets generated by UTAGs.
(The reader may wonder why we speak of sets and not
of classes. In fact, we shall tacitly assume that trees are
really tree domains, so that classes of finite trees are automatically
sets.) Let $G = \auf \mbox{\tt S}, N, A, R\zu$  be a CFG. We
want to define a tree adjunction grammar $\mathsf{Ad}_G =
\auf \BC_G, N, A, \BA_G\zu$ such that $L_B(G) = L_B(\mathsf{Ad}_G)$.
We define $\BC_G$ to be the set of all (ordered labelled) tree (domains)
$\GB$ which can be generated by $L_B(G)$ and which are centre trees
and in which on no path not containing the root some nonterminal symbol
occurs twice. Since there are only finitely many symbols and the
branching is finite, this set is actually finite. Now we define
$\BA_G$. Let $\BA_G$ contain all adjunction trees $\GB_X$, $X \in N$, 
(modulo identification of $Y^0, Y^1$ with $Y$ for all $Y \in N$)
such that (1) $\GB_X$ can be derived from $X$ in $\gamma G$,  
(2) no symbol occurs twice along a path that does contain the root.  
Also $\BA_G$ is finite. It is not hard to show that 
$L_B(\mathsf{Ad}_G) \subseteq L_B(G)$. The reverse inclusion 
we shall show by induction on the number of nodes in the tree 
(domain). Let $\GB$ be in $L_B(G)$. Either there is a path
not containing the root along which some symbol occurs twice, or
there is not.  In the second case the tree is in $\BC_G$.
Hence $\GB \in L_B(\mathsf{Ad}_G)$ and we are done.
In the first case we choose an $x \in B$ of minimal height such that
there is a $y < x$ with identical label; let the label be $X$.
Consider the subtree $\GU$ induced by the set
$(\low{x} - \low{y}) \cup \{y\}$. We claim that
$\GU \in \BA_G$.  For this we have to show the following.
(a) $\GU$ is an adjunction tree, (b) $\GU$ can be deduced from $X$,
(c) no symbol symbol occurs twice along a path  which does not contain
$x$. Ad (a). A leaf of $\GU$ is either a leaf of $\GB$ or $= y$. In
the first case the label is a terminal symbol in the second case
it is identical to that of the root. Ad (b). If $\GB$ is a tree of
$\gamma G$ then $\GU$ can be derived from $X$. Ad (c).
Let $\pi$ be a path which does not contain $x$ and let
$u,v \in \pi$ nodes with identical label and $u < v$.
Then $v < x$, and this contradicts the minimality of
$x$. Hence all three conditions are met.  So we can disembed
$\GU$. This means that there is a tree $\GB'$ such that $\GB$
is derived from $\GB'$ by adjoining $\GU$. We have $\GB' \in L_B(G)$
and by induction hypothesis $\GB' \in L_B(\mathsf{Ad}_G)$.
Hence $\GB \in L_B(\mathsf{Ad}_G)$, which had to be shown.
\nocite{joshilevytakahashi:adjunct}
%%
\index{Joshi, Aravind}%%%
\index{Takahashi, Masako}%%
\index{Levy, Leon S.}%%%
%%%
\begin{thm}[Joshi \& Levy \& Takahashi]
Every set of labelled ordered tree domains generated by a
CFG is also one generated by a UTAG.
\proofend
\end{thm}
%%
Now we shall prove Parikh's Theorem for UTAGs.
Let $\alpha$ be a letter and $\GB$ a tree. Then 
$\sigma_{\alpha}(\GB)$ 
%%%%
\index{$\sigma_a(\GB)$, $\mu(\GB)$}%%%
%%%%
is the number of nodes whose label 
is $\alpha$. If $\GB$ is an adjunction tree then the label 
of the root is {\it not counted}. Now let $\auf \BC, N, A, \BA\zu$ 
be a UTAG and $\BC = \{\GC_i : i < \alpha\}$, $\BA = \{\GA_j : j < \beta\}$.
%%
\begin{lem}
Let $\GB'$ result from $\GB$ by adjoining the tree
$\GA$. Then $\sigma_{\alpha}(\GB') = \sigma_{\alpha}(\GB)
+ \sigma_{\alpha}(\GA)$.
\end{lem}
%%
The proof of this lemma is easy. From this it follows that
we only need to know for an arbitrarily derived tree how
many times which tree has been adjoined and what the starting
tree was. So let $\GB$ be a tree which resulted from $\GC_i$
by adjoining $\GA_j$ $p_j$ times, $j < \beta$. Then
%%
\begin{equation}\sigma_{\alpha}(\GB) = \sigma_{\alpha}(\GC_i) +
\sum_{i < \beta} p_j \cdot \sigma_{\alpha}(\GA_j)
\end{equation}
%%
Let now $\mu(\GB) := \sum_{a \in A} \sigma_a(\GB) \cdot a$.
Then 
%%
\begin{equation}
\mu(\GB) = \mu(\GC_i) + \sum_{i < \beta} p_j \cdot \mu(\GB_j)
\end{equation}
%%
We define the following sets
%%
\begin{equation}
\Sigma_i := \mu(\GC_i) + \sum_{j < \beta} \omega \mu(\GA_j) 
\end{equation}
%%
Then $\mu[L_B(\auf \BC, \BA\zu)] \subseteq \bigcup_{i < n} \Sigma_i$.
However, equality need not always hold. We have to notice the following
problem. A tree $\GA_j$ can be adjoined to a tree $\GB$ only if its root
label actually occurs in the tree $\GB$. Hence not all values of
$\bigcup \Sigma_i$ are among the values under $\mu$ of a derived
tree. However, if a tree can be adjoined {\it once\/} it can
be adjoined any number of times and to all trees that result
from this tree by adjunction. Hence we modify our starting set
of trees somewhat. We consider the set $D$ of all pairs
$\auf k, W\zu$ such that $k < \alpha$, $W \subseteq \beta$ and
there is a derivation of a tree that starts with $\GC_k$ and
uses exactly the trees from $W$. For $\auf k, W\zu \in D$
%%
\begin{equation}
L(k,W) = \mu(\GC_i) + \sum_{j \in W} \omega \cdot \mu(\GA_j)
\end{equation}
%%
Then $L := \bigcup \auf L(k,W) : \auf k,W\zu \in D\zu$
is semilinear. At the same time it is the set of
all $\mu(\GB)$ where $\GB$ is derivable from $\auf \BC, N, A, %
\BA\zu$. 
%%
\begin{thm}
Let $L$ be the language of an unregulated tree adjunction grammar
then $L$ is semilinear.
\proofend
\end{thm}
%%
\index{Parikh, Rohit}%%
%%%
\begin{cor}[Parikh]
Let $L$ be context free. Then $L$ is semilinear. \proofend
\end{cor}
%%
This theorem is remarkable is many respects. We shall
meet it again several times. Semilinear sets are closed under
complement (Theorem~\ref{thm:semabschluss}) and hence also
under intersection. We shall show, however, that this does not hold
for semilinear languages.
%%
\begin{prop}
There are CFLs $L_1$ and $L_2$ such that $L_1 \cap L_2$ is not semilinear.
\end{prop}
%%
\proofbeg
Let $M_1 := \{\mbox{\tt a}^n \mbox{\tt b}^n :
n \in \omega\}$ and $M_2 := \{\mbox{\tt b}^n \mbox{\tt a}^{2n} :
n \in \omega\}$. Put
%%
\begin{align}
L_1 & := \mbox{\tt b} M_1^{\ast} \mbox{\tt a}^{\ast} &
L_2 & := M_2^+
\end{align}
%%
Because of Theorem~\ref{thm:afl} $L_1$ and $L_2$ are context free.
Now look at $L_1 \cap L_2$. It is easy to see that the intersection
consists of the following strings.
%%
\begin{equation}
\mbox{\tt ba}^2, \quad \mbox{\tt ba}^2 \mbox{\tt b}^2\mbox{\tt a}^4,
\quad \mbox{\tt ba}^2\mbox{\tt b}^2\mbox{\tt a}^4\mbox{\tt
    b}^4\mbox{\tt a}^8, \quad
\mbox{\tt ba}^2\mbox{\tt b}^2 \mbox{\tt a}^4\mbox{\tt
    b}^4 \mbox{\tt a}^8 \mbox{\tt b}^8 \mbox{\tt a}^{16}, \dotsc
\end{equation}
%%
The Parikh image is $\{(2^{n+2}-2)\mbox{\tt a} +
(2^{n+1} - 1)\mbox{\tt b} : n \in \omega\}$.  This set is not
semilinear, since the result of deleting the symbol {\tt b} (that
is, the result of applying the projection onto
$\mbox{\tt a}^{\ast}$) is not almost periodical.
\proofend

We know that for every semilinear set $N \subseteq M(A)$ there
is a regular grammar $G$ such that $\mu[L(G)] = N$. However
$G$ can be relatively complex. Now the question arises whether
the complete preimage $\mu^{-1}[N]$ under $\mu$ is at least
regular or context free. This is not the case. However, we
do have the following.
%%
\begin{thm}
\label{thm:urbild}
The full preimage of a semilinear set over a single letter
alphabet is regular.
\end{thm}
%%
This is the best possible result. The theorem becomes false as soon
as we have two letters.
%%
\begin{thm}
The full preimage of $\omega (\mbox{\tt a} + \mbox{\tt b})$
is not regular; it is however context free. The full
preimage of $\omega (\mbox{\tt a} + \mbox{\tt b} + \mbox{\tt c})$
is not context free.
\end{thm}
%%
\proofbeg
We show the second claim first. Let
%%
\begin{equation}
W := \mu^{-1}[\omega (\mbox{\tt a} + \mbox{\tt b} +
\mbox{\tt c})] 
\end{equation}
%%
Assume that $W$ is context free. Then the intersection with the
regular language $\mbox{\tt a}^{\ast}\mbox{\tt b}^{\ast}
\mbox{\tt c}^{\ast}$ is again context free. This is precisely
the set $\{\mbox{\tt a}^n \mbox{\tt b}^n \mbox{\tt c}^n : n \in %
\omega\}$. Contradiction. Now for the first claim. Denote by
$b(\vec{x})$ the number of occurrences of {\tt a} in $\vec{x}$
minus the number of occurrences of {\tt b} in $\vec{x}$. Then
$V := \{\vec{x} : b(\vec{x}) = 0\}$ is the full preimage of
$\omega (\mbox{\tt a} + \mbox{\tt b})$. $V$ is not regular;
otherwise the intersection with $\mbox{\tt a}^{\ast} %
\mbox{\tt b}^{\ast}$ is also regular. However, this is 
$\{\mbox{\tt a}^n \mbox{\tt b}^n : n \in \omega\}$.
Contradiction. However, $V$ is context free. To show this we 
shall construct a CFG $G$ over $A = \{\mbox{\tt a}, \mbox{\tt b}\}$ 
which generates $V$. We have three
nonterminals, {\tt S}, {\tt A}, and {\tt B}. The rules are
%%
\begin{equation}
\begin{array}{l@{\quad\pf\quad}l}
\mbox{\tt S} & \mbox{\tt SS} \mid \mbox{\tt AB} \mid \mbox{\tt BA}
    \\
\mbox{\tt A} & \mbox{\tt AS} \mid \mbox{\tt SA} \mid \mbox{\tt a} \\
\mbox{\tt B} & \mbox{\tt BS} \mid \mbox{\tt SB} \mid \mbox{\tt b} \\
\end{array}
\end{equation}
%%
The start symbol is {\tt S}. We claim:
$\mbox{\tt S} \vdash_G \vec{x}$ iff $b(\vec{x}) = 0$,
$\mbox{\tt A} \vdash_G \vec{x}$ iff
$b(\vec{x}) = 1$ and $\mbox{\tt B} \vdash_G \vec{x}$ iff
$b(\vec{x}) = - 1$. The directions from left to right are
easy to verify. It therefore follows that
$V \subseteq L(G)$. The other directions we show by induction
on the length of $\vec{x}$. It suffices to show the following
claim.
%%
\begin{quote}
If $b(\vec{x}) \in \{1,0,-1\}$ there are
$\vec{y}$ and $\vec{z}$ such that $|\vec{y}|, |\vec{z}| <
|\vec{x}|$ and such that $\vec{x} = \vec{y}\,\vec{z}$ as well as
$|b(\vec{y})|, |b(\vec{z})| \leq 1$.
\end{quote}
%%
Hence let $\vec{x} = \prod_{i < n} x_i$  be given.
Define $k(\vec{x}, j) := b(^{(j)}\vec{x})$,
and $K := \{k(\vec{x}, j) : j < n+1\}$. As is easily seen,
$K = [m,m']$ with $m \leq 0$.
Further, $k(\vec{x},n) = b(\vec{x})$. (a) Let $b(\vec{x}) = 0$.
Then put $\vec{y} := x_0$ and $\vec{z} :=
\prod_{0 < i < n} x_i$. This satisfies the conditions.
(b) Let $b(\vec{x}) = 1$. Case 1: $x_0 = \mbox{\tt a}$.
Then put again $\vec{y} := x_0$ and $\vec{z} :=
\prod_{0 < i < n} x_i$. Case 2: $x_0 = \mbox{\tt b}$.
Then $k(\vec{x},1) = -1$ and there is a $j$ such that
$k(\vec{x}, j) = 0$. Put $\vec{y} := \prod_{i < j} x_i$,
$\vec{z} := \prod_{j \leq i < n} x_i$.  Since $0 < j < n$,
we have $|\vec{y}|, |\vec{z}| < |\vec{x}|$.
Furthermore, $b(\vec{y}) = 0$ and $b(\vec{z}) = 1$. (c)
$b(\vec{x}) = -1$. Similar to (b).
\proofend
%%
\vplatz
\exercise
Let $|A| = 1$ and $\SA\Sd$ be a UTAG. Show that the language 
generated by $\SA\Sd$ over $A^{\ast}$ is regular.
%%%
\vplatz
\exercise
Prove Theorem~\ref{thm:urbild}. {\it Hint.} Restrict
your attention first to the case that $A = \{\mbox{\tt a}\}$.
%%
\vplatz
\exercise
Let $N \subseteq M(A)$ be semilinear. Show that the full
preimage is of Type 1 (that is, context sensitive).
{\it Hint.} It is enough to show this for linear sets.
%%
\vplatz
\exercise
In this exercise we sketch an alternative proof of Parikh's Theorem.
Let $A = \{\mbox{\tt a}_i : i < n\}$ be an alphabet. In analogy to the
regular terms we define semilinear terms. (a) $\mbox{\tt a}_i$, $i < n$,
is a semilinear term, with interpretation $\{\vec{e}_i\}$. (b) If
$A$ and $B$ are semilinear terms, so is $A \oplus B$ with
interpretation $\{\vec{u} + \vec{v} : \vec{u} \in A,
\vec{v} \in B\}$, $A \cup B$, with interpretation $\{\vec{u} :
\vec{u} \in A \mbox{ or }\vec{u} \in B\}$ and
$\omega A$ with interpretation $\{k \vec{u} : k \in \omega,
\vec{u} \in A\}$. The first step is to translate a CFG
into a set of equations of the form
$X_i = C_i(X_0, X_1, \dotsc, X_{q-1})$, $q$ the number
of nonterminals, $C_i$ semilinear terms. This is done as follows.
Without loss of generality we can assume that in a rule $X \pf
\vec{\alpha}$, $\vec{\alpha}$ contains a given variable at most
once. Now, for each nonterminal $X$ let $X \pf \vec{\alpha}_i$,
$i < p$, be all the rules of $G$. Corresponding to these rules
there is an obvious equation of the form
%%
\begin{equation}
X = A \cup (B \oplus X) \text{ or } X = A
\end{equation}
%%
where $A$ and $B$ are semilinear terms that do not contain $X$.
The second step is to prove the following lemma:
%%
\begin{quote}
{\it Let $X = A \cup (B \oplus X) \cup (C \oplus \omega X)$,
with $A$, $B$ and $C$ semilinear terms not containing $X$. Then the
least solution of that equation is $A \cup \omega B \cup \omega C$.
If $B \oplus X$ is missing from the equation, the solution is
$A \cup \omega C$, and if $C \oplus \omega X$ is missing the
solution is $A \cup \omega B$.}
\end{quote}
%%
Using this lemma it can be shown that the system of equations
induced by $G$ can be solved by constant semilinear terms for
each variable.
%%%
\vplatz
\exercise
Show that the UTAG $\auf \{\GC\}, \{\mbox{\tt S}\}, \{\mbox{\tt a}, 
\mbox{\tt b}, \mbox{\tt c}, \mbox{\tt d}\}, \{\GA\}\zu$ generates 
exactly the strings of the form $\vec{x}\mbox{\tt d}\mbox{\tt c}^n$, 
where $\vec{x}$ is a string of $n$ {\tt a}'s and $n$ {\tt b}'s such 
that every prefix of $\vec{x}$ has at least as many {\tt a}'s as 
{\tt b}'s. 
%%%
\begin{center}\begin{picture}(7,10)
\put(1,9){\makebox(0,0){$\GC$}}
\put(3,4.5){\makebox(0,0){\makebox{\tt d}}}
\put(3,5.5){\line(0,1){3}}
\put(3,9){\makebox(0,0){\makebox{\tt S}}}
\end{picture}
%%
\begin{picture}(10,10)
\put(1,9){\makebox(0,0){$\GA$}}
\put(1,5){\makebox(0,0){\makebox{\tt a}}}
\put(1.5,5.5){\line(1,1){3}}
\put(5,9){\makebox(0,0){\makebox{\tt S}}}
\put(5.5,8.5){\line(1,-1){3}}
\put(9,5){\makebox(0,0){\makebox{\tt S}}}
\put(8.5,4.5){\line(-1,-1){3}}
	\put(5,1){\makebox(0,0){\makebox{\tt b}}}
\put(9,4.5){\line(0,-2){3}}
	\put(9,1){\makebox(0,0){\makebox{\tt S}}}
\put(9.5,4.5){\line(1,-1){3}}
	\put(13,1){\makebox(0,0){\makebox{\tt c}}}
\end{picture}
\end{center}
%%%
Show also that this language is not context free. (This 
example is due to \cite{joshilevytakahashi:adjunct}.)
