\section{Axiomatic Classes II: Exhaustively Ordered Trees}
\label{kap5-4}%%
\nocite{doner:tree}%%
\nocite{thatcherwright}%%
%
%
%
The theorem by B\"uchi on axiomatic classes of strings has a
very interesting analogon for exhaustively ordered trees.
We shall prove it here; however, we shall only show those
facts that are not proved in a completely similar way.
Subsequently, we shall outline the importance of this
theorem for syntactic theory. The reader should consult
Section~\ref{kap1}.\ref{kap1-4} for notation. Ordered
trees are structures over a language that has two binary
relation symbols, $\sqsubset$ and $<$. We also take
labels from $A$ and $N$ (!) in the form of constants
and get the language $\mathsf{MSO}^b$. 
%%%
\index{$\mathsf{MSO}^b$}%%%
%%%
In this language the
set of exhaustively ordered trees is a finitely
axiomatizable class of structures. We consider first
the postulates. $<$ is transitive and irreflexive,
$\uppx{x}$ is linear for every $x$, and there is a largest
element, and every subset has a largest and a smallest element
with respect to  $<$. From this it follows in particular that
below an arbitrary element there is a leaf. Here are now the
axioms listed in the order just described.
%%
\begin{subequations}
\begin{align}
& (\forall xyz)(x < y \und y < z .\pf. x < z) \\
& (\forall x)\nicht (x < x) \\
& (\forall xyz)(x < y \und x < z. \pf . y < z \oder y \doteq z \oder
    y > z) \\
& (\exists x)(\forall y)(y < x \oder y \doteq x) \\
& (\forall P)(\exists x)(\forall y)(P(x) \und y < x .\pf.
    \nicht P(y)) \\
& (\forall P)(\exists x)(\forall y)(P(x) \und y < x .\pf.
    \nicht P(y))
\end{align}
\end{subequations}
%%
In what is to follow we use the abbreviation $x \leq y := x < y \oder
x \doteq y$. Now we shall lay down the axioms for the ordering.
$\sqsubset$ is transitive and irreflexive, it is linear on the
leaves, and we have $x \sqsubset y$ iff for all leaves
$u$ below $x$ and all leaves $v \leq y$ we have $u \sqsubset v$.
Finally, there are only finitely many leaves, a fact which we can
express by requiring that every set of nodes has a smallest
and a largest member (with respect to $\sqsubset$).
We put $b(x) := \nicht (\exists y)(y < x)$.
%%
\begin{subequations}
\begin{align}
& (\forall xyz)(x \sqsubset y \und y \sqsubset z. \pf .x \sqsubset z) \\
& (\forall x)\nicht (x \sqsubset x) \\
& (\forall xy)(b(x) \und b(y) .\pf.
    x \sqsubset y \oder x \doteq y \oder y \sqsubset x) \\
& (\forall xy)(x \sqsubset y. \dpf .
    (\forall uv)(b(u) \und u \leq x \und b(v) \und v \leq y.
    \pf .u \sqsubset v)) 
\\
& (\forall P)\{(\forall x)(P(x) \pf b(x)).  \\\notag
& \quad \pf. \phantom{\mbox{}\und\mbox{}}
    (\exists y)(P(y) \und (\forall z)(P(z) \pf
    \nicht (y \sqsubset z))) \\\notag
& \quad \phantom{\mbox{}\pf. \mbox{}}
    \und (\exists y)(P(y) \und (\forall z)(P(z) \pf
    \nicht (z \sqsupset y)))\}
\end{align}
\end{subequations}
%%
Thirdly,  we must regulate the distribution of the labels.
%%
\begin{subequations}
\begin{align}
%& (\forall x)(\goder \auf \uli{a}(x) : a \in N\zu \dpf 
%	\nicht \goder \auf \uli{A}(x) : A \in N\zu) \\
& (\forall x)(b(x) \dpf \goder \auf \uli{a}(x) : a \in A\zu) \\
%& (\forall x)(b(x) \pf \gund \auf \uli{a}(x) \pf \nicht \uli{b}(x) :
%    a \neq b\zu) \\
& (\forall x)(\nicht b(x) \pf \goder \auf \uli{A}(x) : A \in N\zu) \\
& (\forall x)(\gund \auf \uli{\alpha}(x) \pf \nicht \uli{\beta}(x)
    : \alpha \neq \beta\zu)
\end{align}
\end{subequations}
%%
The fact that a tree is exhaustively ordered is described by the
following formula.
%%
\begin{equation}
(\forall xy)(\nicht (x \leq y \oder y \leq x) .\pf .
    x \sqsubset y \oder y \sqsubset x)
\end{equation}
%%
\begin{prop}
The following are finitely $\mathsf{MSO}^b$--axiomatizable
classes.
%%
\begin{dingautolist}{192}
\item The class of ordered trees.
\item The class of finite exhaustively ordered trees.
\end{dingautolist}
\end{prop}
%%
Likewise we can define a quantified modal language.
However, we shall change the base as follows, using the
results of Exercise~\ref{ueb:schwester}. We assume 8 operators,
$M_8 := \{\oben, \oben^+, \unten, \unten^+, \rechts, \rechts^+,
\links, \links^+\}$, 
%%%
\index{$\oben$, $\oben^+$, $\unten$, $\unten^+$, $\rechts$, $\rechts^+$, $\links$, $\links^+$}%%% 
%%%
which correspond to the 
relations $\prec$, $<$, $\succ$, $>$, {\it immediate left
sister of}, {\it left sister of}, {\it immediate right sister
of}, as well as {\it right sister of}. These relations are
MSO--definable from the original ones, and conversely the original 
relations can be MSO--defined from the present ones. Let 
$\GT = \auf T, <, \sqsubset\zu$ be an exhaustively ordered tree. 
Then we define $R \colon M_8 \pf \wp(T)$ as follows.
%%
\begin{equation}
\begin{split}
x\; R(\rechts^+)\; y & := x \sqsubset y \und (\exists z)(x \prec z
\und y \prec z) \\
x\; R(\rechts)\; y & := x\; R(\rechts^+)\; y \und
    \nicht (x\; R(\rechts^+) \circ R(\rechts^+)\; y) \\
x\; R(\links^+)\; y & := x \sqsupset y \und (\exists z)(x \prec z
	\und y \prec z) \\
x\; R(\links)\; y & := x\; R(\links^+)\; y \und
    \nicht (x\; R(\links^+) \circ R(\links^+)\; y) \\
x\; R(\oben^+)\; y & := x < y \\
x\; R(\oben)\; y & := x \prec y \\
x\; R(\unten^+)\; y & := x > y \\
x\; R(\unten)\; y & := x \succ y 
\end{split}
\end{equation}
%%
The resulting structure we call $M(\GT)$. Now if $T$ as well as
$R$ are given, then the relations $\prec$, $\succ$, $<$, $>$,
and $\sqsubset$, as well as $\sqsupset$ are definable.
First we define $\oben^{\ast} \varphi := \varphi \oder
\oben^+ \varphi$, and likewise for the other relations.
Then $R(\oben^{\ast}) = \Delta \cup R(\oben^+)$.
%%
\begin{equation}
\begin{split}
\prec & = R(\oben) & \succ & = R(\unten) \\
< & = R(\oben^+) & > & = R(\unten^+) \\
\sqsubset & = R(\oben^{\ast}) \circ R(\rechts^+) \circ
    R(\unten^{\ast}) \\
\sqsupset & = R(\oben^{\ast}) \circ R(\links^+) \circ
    R(\unten^{\ast}) 
\end{split}
\end{equation}
%%
Analogously, as with the strings we can show that the following
properties are axiomatizable: (a) that $R(\links^+)$
is transitive and irreflexive with converse relation
$R(\rechts^+)$; (b) that $R(\links^+)$ is the transitive
closure of $R(\links)$ and $R(\rechts^+)$ the transitive closure
of $R(\rechts)$. Likewise for $R(\oben^+)$ and $R(\oben)$,
$R(\unten^+)$ and $R(\unten)$. With the help of the axiom
below we axiomatically capture the condition that $\uppx{x}$
is linear:
%%
\begin{equation}
\oben^+ p \und \oben^+ q .\pf.
\oben^+ (p \und q) \oder \oben^+ (p \und \oben^+ q)
\oder \oben^+ (q \und \oben^+ p)
\end{equation}
%%
The other axioms are more complex. Notice first the
following.
%%
\begin{lem}
Let $\auf T, < , \sqsubset\zu$ be an exhaustively ordered tree
and $x, y \in T$. Then $x \neq y$ iff
(a) $x < y$ or (b) $x > y$ or (c) $x \sqsubset y$ or
(d) $x \sqsupset y$.
\end{lem}
%%
Hence the following definitions.
%%
\begin{align}
\auf\neq\zu \varphi & := \oben^+ \varphi \oder \unten^+ \varphi
         \oder
    \oben^{\ast} \rechts^+ \unten^{\ast} \varphi
         \oder
    \oben^{\ast} \links^+ \unten^{\ast} \varphi \\
\master \varphi & := \varphi \und [\neq] \varphi
\end{align}
%%
So we add the following set of axioms.
%%
\begin{multline}
\{\master \varphi \pf \qrechts^+ \varphi,
\master \varphi \pf \qlinks^+ \varphi,  
\master \varphi \pf \qoben^+ \varphi,  
\master \varphi \pf \qunten^+ \varphi \\
\master \varphi \pf \master\master \varphi,  
\master \varphi \pf \varphi, 
\varphi \pf \master \nicht\master\nicht \varphi\}
\end{multline}
%%
(Most of them are already derivable. The axiom system is therefore
not minimal.) These axioms see to it that in a connected structure 
every node is reachable from any other by means of the basic relations,
moreover, that it is reachable in one step using $R(\master)$.
Here we have
%%
\begin{equation}
R(\master) =
\{\auf x,y\zu : \mbox{\it there is \/} z:
x \leq z \geq y\}
\end{equation}
%%
Notice that this always holds in a tree and that conversely it
follows from the above axioms that $R(\oben^+)$ possesses a
largest element. Now we put
%%
\begin{equation}
b(\varphi) := \varphi \und \qunten \bot \und [\neq] \nicht \varphi
\end{equation}
%%
$b(\varphi)$ holds at a node $x$  iff
$x$ is a leaf and $\varphi$ is true exactly at $x$. Now we can
axiomatically capture the conditions that
$R(\rechts^+)$ must be linear on the set of leaves.
%%
\begin{equation}
\qunten \bot \und \auf\neq\zu b(q). \pf.
\rechts^+ p \oder \links^+ p
\end{equation}
%%
Finally, we have to add axioms which constrain the distribution of
the labels. The reader will be able to supply them. A 
forest is defined here as the disjoint union of trees.
%%
\begin{prop}
The class of exhaustively ordered forests is fi\-ni\-te\-ly
$\mathsf{QML}^b$--axi\-o\-ma\-ti\-sa\-ble.
\end{prop}
%%
We already know that $\mathsf{QML}^b$ can be embedded into
$\mathsf{MSO}^b$. The converse is as usual somewhat difficult.
To this end we proceed as in the case of strings. We introduce
an analogon of restricted quantifiers. We define
functions $\oben$, $\oben^+$, $\unten$, $\unten^+$,
$\rechts$, $\rechts^+$, $\links$, $\links^+$, as well as
$\auf\neq\zu$ on unary predicates, whose meaning should
be self explanatory. For example
%%
\begin{subequations}
\begin{align}
(\oben \varphi)(x) & := (\exists y \succ x)\varphi(y) \\
(\oben^+ \varphi)(x) & := (\exists y > x)\varphi(y)
\end{align}
\end{subequations}
%%
where $y \not\in \fr(\varphi)$.
Finally let $O$ be defined by
%%
\begin{equation}
O(\varphi) := (\forall x)\nicht \varphi(x)
\end{equation}
%%
$O(\varphi)$ says that $\varphi(x)$ is nowhere satisfiable. Let 
$P_x$ be a  predicate variable which
does not occur in $\varphi$. Define $\{P_x/x\}\varphi$ inductively
as described in Section~\ref{kap5}.\ref{kap5-2}. Let $\gamma(P_x) =
\{\beta(x)\}$. Then we have
%%
\begin{equation}
\auf \GM, \gamma, \beta\zu \vDash \varphi\qquad
\Dpf\qquad \auf \GM, \gamma, \beta\zu \vDash
\{P_x/x\}\varphi 
\end{equation}
%%
Therefore put
%%
\begin{equation}
(E x)\varphi(x) := (\exists P_x)(\nicht O(P_x) \und
    O(P_x \und \auf\neq\zu P_x).  \pf .\{P_x/x\}\varphi)
\end{equation}
%%
Because of this we have for all exhaustively
ordered trees $\GT$
%%
\begin{equation}
\auf \GT, \gamma, \beta\zu \vDash (\exists x)\varphi\qquad
\Dpf\qquad
\auf \GT, \gamma, \beta\zu \vDash
(E x)\varphi
\end{equation}
%%
Let again $h \colon P \pf \mbox{\it PV\/}$ be a bijection from
the set of predicate variables of $\mathsf{MSO}^b$ onto
the set of proposition variables or $\mathsf{QML}^b$.
%%
\begin{equation}
\begin{split}
(\uli{a}(x))^{\diamond} & := Q^a & (P(y))^{\diamond} & := h(P) \\
(\oben \varphi)^{\diamond} & := \oben \varphi^{\diamond} &
(\unten \varphi)^{\diamond} & := \unten \varphi^{\diamond} \\
(\links\varphi)^{\diamond} & := \links \varphi^{\diamond} &
(\rechts\varphi)^{\diamond} & := \rechts \varphi^{\diamond} \\
(\oben^+\varphi)^{\diamond} & := \oben^+ \varphi^{\diamond} &
(\unten^+\varphi)^{\diamond} & := \unten^+ \varphi^{\diamond} \\
(\rechts^+\varphi)^{\diamond} & := \rechts^+ \varphi^{\diamond} &
(\links^+ \varphi))^{\diamond} & := \links^+ \varphi^{\diamond} \\
(\nicht \varphi)^{\diamond} & := \nicht \varphi^{\diamond} &
(O(\varphi))^{\diamond} & := \master \nicht \varphi^{\diamond} \\
(\varphi_1 \und \varphi_2)^{\diamond} &
    := \varphi_1^{\diamond} \und \varphi_2^{\diamond} &
(\varphi_1 \oder \varphi_2)^{\diamond} &
    := \varphi_1^{\diamond} \oder \varphi_2^{\diamond} \\
((\exists P)\varphi)^{\diamond} & := (\exists h(P))\varphi^{\diamond} &
((\forall P)\varphi)^{\diamond} & := (\forall h(P))\varphi^{\diamond}
\end{split}
\end{equation}
%%
Then the desired embedding of $\mathsf{MSO}^b$
into $\mathsf{QML}^b$ is shown.
%%
\begin{thm}
Let $\varphi$ be an $\mathsf{MSO}^b$--formula with at most
one free variable, the object variable $x_0$. Then there exists
a $\mathsf{QML}^b$--formula $\varphi^M$ such that for all
exhaustively ordered trees $\GT$:
%%
\begin{equation}
\auf \GT, \beta\zu \vDash \varphi(x_0)
\text{ iff }\auf M(\GT), \beta(x_0)\zu \vDash
\varphi(x_0)^M
\end{equation}
%%
\end{thm}
%%
\begin{cor}
\label{cor:mqlb} Modulo the identification $\GT \mapsto M(\GT)$
$\mathsf{MSO}^b$ and $\mathsf{QML}^b$ define the same model
classes of exhaustively ordered trees. Further: $\CK$ is a
finitely axiomatizable class of $\mathsf{MSO}^b$--struc\-tu\-res 
iff $M(\CK)$ is a finitely axiomatizable class of
$\mathsf{QML}^b$--structures.
\end{cor}
%%
For the purpose of definition of a code we suspend the difference
between terminal and nonterminal symbols.
%%
\begin{defn}
\index{code}%%
\index{faithfulness}%%
%%
Let $G = \auf \Sigma, N,A,R\zu$ be a CFG$^{\ast}$
and $\varphi \in \mathsf{QML}^b$ a constant formula (with constants
over $A$). We say, $G$ is \textbf{faithful for} $\varphi$ if there is a
set $H_{\varphi} \subseteq N$ such that for every tree $\GT$ and
every node $w \in T$: $\auf \GT, w\zu \vDash \varphi$ iff
$\ell(w) \in H_{\varphi}$. We also say that $H_{\varphi}$
\textbf{codes} $\varphi$ \textbf{with respect to} $G$.
Let $\varphi$ be a $\mathsf{QML}^b$--formula and $n$ a
natural number. An $n$--\textbf{code for} $\varphi$ is a pair
$\auf G, H\zu$ such that $L_B(G)$ is the set of all at most
$n$--ary branching, finite, exhaustively ordered trees over
$A \cup N$ and $H$ codes $\varphi$ in $G$. $\varphi$ is called
$n$--\textbf{codable} 
%%%
\index{formula!codable}%%%
%%%
if there is an $n$--code for $\varphi$.  $\varphi$ is called 
\textbf{codable} if there is an $n$--code
for $\varphi$ for every $n$.
\end{defn}
%%
Notice that for technical reasons we must restrict ourselves
to at most $n$--branching trees since we can otherwise not write
down a CFG$^{\ast}$ as a code. Let
$G = \auf \Sigma, N, A, R\zu$ and $G' = \auf \Sigma', N', A, R'\zu$
be grammars$^{\ast}$ over $A$. 
%%%
\index{grammar$^{\ast}$!product of {\faul}s}%%
%%%
The \textbf{product} is defined by
%%
\begin{equation}
G \times G' = \auf \Sigma\times\Sigma', N \times N', A, R \times R'\zu
\end{equation}
%%
where
%%
\begin{multline}
R \times R' := 
     \{\auf X,X'\zu \pf \auf \alpha_0, \alpha'_0\zu \dotsb
    \auf \alpha_{n-1},\alpha'_{n-1}\zu : \\
    X \pf \alpha_0\dotsb \alpha_{n-1} \in R,
    X' \pf \alpha'_0\dotsb \alpha'_{n-1} \in R'\} 
\end{multline}
%%
To prove the analogon of the Coding Theorem (\ref{thm:code}) for 
strings we shall have to use a trick. As one can easily show
the direct extension on trees is false since we have
also taken the nonterminal symbols as symbols of the
language. So we proceed as follows. Let $h \colon N \pf N'$ be
a map and $\GT = \auf T, <, \sqsubset, \ell\zu$ a tree with
labels in $A \cup N$. Then let $h[\GT] :=
\auf T, <, \sqsubset, h_A \circ \ell\zu$ where
$h_A \restriction N := h$ and $h_A(a) := a$ for all $a \in A$.
%%%
\index{projection}%%
%%%
Then $h[\GT]$ is called a \textbf{projection of} $\GT$.
If $\CK$ is a class of trees, then let $h[\CK] :=
\{h[\GT] : \GT \in \CK\}$. 
%%
\begin{thm}[Thatcher \& Wright, Doner]
%%%
\index{Thatcher, J.~W.}%%
\index{Wright, J.~B.}%%
\index{Doner, J.~E.}%%
Let $A$ be a terminal alphabet, $N$ a nonterminal alphabet 
and $n \in \omega$. A class of exhaustively ordered, at most
$n$--branching finite trees over $A \cup N$ is finitely
axiomatizable in $\mathsf{MSO}^b$ iff it is the
projection onto $A \cup N$ of a context free$^{\ast}$ class of
ordered trees over some alphabet.
\end{thm}
%%
Here a class of trees is \textbf{context free}$^{\ast}$ if it
is the class of trees generated by some CFG$^{\ast}$.
Notice that the symbol $\varepsilon$ is not problematic as it
was for regular languages. We may look at it as an
independent symbol which can be the label of a leaf. However,
if this is to be admitted, we must assume that the terminal
alphabet may be $A_{\varepsilon}$ and not $A$. Notice that
the union of two context free sets of trees is not necessarily
itself context free. (This again is different for regular
languages, since the structures did not contain the nonterminal
symbols.)

From now on the proof is more or less the same. First one shows
the codability of $\mathsf{QML}^b$--formulae. Then one argues
as follows. Let $\auf G, H\zu$ be the code of a formula $\varphi$.
We restrict the set of symbols (that is, both $N$ as well as $A$)
to $H$. In this way we get a grammar$^{\ast}$ which only generates
trees that satisfy $\varphi$. Finally we define the projection
$h \colon H \pf A \cup N$ as follows. Put $h(a) := a$, $a \in A$,
and $h(Y) := X$ if $L_B(G) \vDash (\forall x)( \uli{Y}(x) \pf %
\uli{X}(x))$. In order for this to be well defined we must 
therefore have for all $Y \in H$ an $X \in N$ with this property. 
In this
%%%
\index{code!uniform}%%
%%%
case we call the code \textbf{uniform}. Uniform codability follows
easily from codability since we can always construct products
$G \times G'$ of grammars$^{\ast}$ so that
$G = \auf \Sigma, N, A, R\zu$ and $L_B(G \times G') \vDash
\uli{X}(\auf x,y\zu)$ iff $L_B(G) \vDash
\uli{X}(x)$. The map $h$ is nothing but the projection onto
the first component.
%%
\begin{thm}
\label{thm:codeb}
Every constant $\mathsf{QML}^b$--formula is uniformly codable.
\end{thm}
%%
\proofbeg We only deliver a sketch of the proof. We choose an $n$
and show the uniform $n$--codability. For ease of exposition we
illustrate the proof for $n = 2$. For the formulae $\uli{a}(x)$, $a
\in A$, and $\uli{Y}(x)$, $Y \in N$, nothing special has to be
done. Again, the booleans are easy. There remain the modal
operators and the quantifiers. Before we begin we shall introduce
a somewhat more convenient notation. As usual we assume that we
have a grammar$^{\ast}$ $G = \auf \Sigma, N, A, R\zu$ as well as
some sets $H_{\eta}$ for certain formulae. Now we take the product
with a new grammar$^{\ast}$ and define $H_{\varphi}$. In place of
explicit labels we now use the formulae themselves, where $\eta$
stands for the set of labels from $H_{\eta}$.

The basic modalities are as follows. Put
%%
\begin{equation}
\boldmath{2} := \auf \{0,1\}, \{0,1\}, A, R_2\zu
\end{equation}
%%
where $R_2$ consists of all possible $n$--branching rules of a 
grammar in standard form. To code $\unten \eta$, we form the 
product of $G$ with \textbf{2}. However, we only choose a subset 
of rules and of the start symbols. Namely, we put $\Sigma' := 
\Sigma \times \{0,1\}$ and $H'_{\eta} := H_{\eta} \times \{0,1\}$, 
$H'_{\unten \eta} := N \times \{1\}$. The rules are all rules 
of the form
%%
\begin{equation}
\begin{array}{l}
\binbaum{\unten \eta}{\top}{\eta}
\binbaum{\unten \eta}{\eta}{\top}
\binbaum{\nicht\unten \eta}{\nicht\eta}{\nicht\eta}
\end{array}
\end{equation}
%%
Now we proceed to $\oben\eta$. Here $\Sigma'_{\oben\eta}
:= N \times \{0\}$.
%%
\begin{equation}
\begin{array}{l}
\binbaum{\eta}{\oben \eta}{\oben \eta}
\binbaum{\nicht\eta}{\nicht\oben\eta}{\nicht\oben\eta}
\end{array}
\end{equation}
%%
With $\rechts\eta$ we choose $\Sigma'_{\rechts\eta} := \Sigma \times
\{0\}$.
%%
\begin{equation}
\begin{array}{l}
\binbaum{\top}{\rechts\eta}{\eta}
\binbaum{\top}{\nicht\rechts \eta}{\nicht \eta}
\end{array}
\end{equation}
%%
Likewise, $\Sigma'_{\links\eta}$ is the start symbol of
$G'$ in the case of $\links\eta$.
%%
\begin{equation}
\begin{array}{l}
\binbaum{\top}{\eta}{\links\eta}
\binbaum{\top}{\nicht\eta}{\nicht \links \eta}
\end{array}
\end{equation}
%%
We proceed to the transitive relations. Notice that on binary 
branching trees $\rechts^+ \eta \dpf \rechts \eta$ and 
$\links^+ \eta \dpf \links \eta$. Now let us look at the 
relation $\unten^+\eta$.
%%
\begin{equation}
\begin{array}{l}
\binbaum{\unten^+ \eta}{\eta \oder \unten^+ \eta}{\top}
\binbaum{\unten^+ \eta}{\top}{\eta \oder \unten^+ \eta} \\
\binbaum{\quad\nicht\unten^+ \eta}{\nicht (\eta \oder \unten^+ \eta)\quad}%
{\nicht (\eta \oder \unten^+ \eta)}
\end{array}
\end{equation}
%%
The set of start symbols is $\Sigma \times \{0,1\}$.
Next we look at $\oben^+ \eta$.
%%
\begin{equation}
\begin{array}{l}
\binbaum{\eta\oder\oben^+ \eta}{\oben^+ \eta}{\oben^+\eta}
\binbaum{\nicht (\eta \oder \oben^+ \eta)}{\nicht\oben^+ %
\eta}{\nicht \oben^+\eta}
\end{array}
\end{equation}
%%
The set of start symbols is $\Sigma' := \Sigma \times \{0\}$.

Finally we study the quantifier $(\exists p)\eta$.
Let $\eta' := \eta[\mathsf{c}/p]$, where $\mathsf{c}$ is a new constant.
Our terminal alphabet is now $A \times \{0,1\}$, the
nonterminal alphabet $N \times \{0,1\}$.  We assume that
$\auf G^1, H^1_{\theta}\zu$ is a uniform code for $\theta$, $\theta$
an arbitrary subformula of $\eta'$. For every subset $\Sigma$ of
the set $\Delta$ of all subformulae of $\eta'$ we put
%%
\begin{equation}
L_{\Sigma} := \gund_{\theta \in \Sigma} \theta \und
    \gund_{\theta \in \Delta - \Sigma} \nicht\theta
\end{equation}
%%
Then $\auf G^1, H^1_{\Sigma}\zu$ is a code for $L_{\Sigma}$
where
%%
\begin{equation}
H^1_{\Sigma} := \bigcap_{\theta \in \Sigma} H^1_{\theta} \cap
    \bigcap_{\theta \in \Delta - \Sigma} (N - H^1_{\theta})
\end{equation}
%%
Now we build a new CFG$^{\ast}$, $G^2$. Put $N^2 := N \times
\{0,1\} \times \wp(N^1)$. The rules of $G^2$ are
all rules of the form
%%
\begin{equation}
\begin{array}{l}
\binbaum{\auf X, i, \Sigma\zu}{\auf Y_0, j_0, \Theta_0\zu}%
{\auf Y_1, j_1, \Theta_1\zu}
\end{array}
\end{equation}
%%
where $\auf X,i\zu \in H^1_{\Sigma}$, $\auf Y_0, j_0\zu \in %
H^1_{\Theta_0}$,  $\auf Y_1, j_1\zu \in H^1_{\Theta_1}$ and
$\Sigma \pf \Theta_0\Theta_1$ is a rule of $G^1$. (This in turn
is the case if there are $X$, $Y_0$ and $Y_1$ as well as $i$, $j_0$
and $j_1$ such that $\auf X,i\zu \pf \auf Y_0, j_0\zu\;
\auf Y_1,j_1\zu \in R$.) Likewise for unary rules. Now
we go over to the grammar$^{\ast}$ $G^3$, with $N^3 := N \times
\wp(\wp(N^1))$. Here we take all rules of the form
%%
\begin{equation}
\begin{array}{l}
\binbaum{\auf X, \BA\zu}{\auf Y_0, \BB_0\zu}{\auf Y_1,\BB_1\zu}
\end{array}
\end{equation}
%%
where $\BA$ is the set of all $\Sigma$ for which there are
$\Theta_0$, $\Theta_1$ and $i$, $j_0$, $j_1$ such that
%%
\begin{equation}
\begin{array}{l}
\binbaum{\auf X, i, \Sigma\zu}{\auf Y_0, j_0, \Theta_0\zu}%
{\auf Y_1, j_1, \Theta_1\zu}
\end{array}
\end{equation}
%%
is a rule of $G^2$.
\proofend
%%
\\
{\it Notes on this section.} From complexity theory we know that
CFLs, being in \textbf{PTIME}, actually possess a description using 
first order logic plus inflationary fixed point operator. This means 
that we can describe the set of strings in $L(G)$ for a CFG by means 
of a formula that uses first order logic plus inflationary fixed points. 
Since we can assume $G$ to be binary branching and invertible, it
suffices to find a constituent analysis of the string. This is a
set of subsets of the string, and so of too high complexity. What
we need is a first order description of the constituency in terms
of the string alone. The exercises describe a way to do this.
%%
\vplatz
\exercise
Show the following: $<$ is definable from $\prec$, likewise $>$.
Also, trees can be axiomatized alternatively with $\prec$ (or
$\succ$). Show furthermore that in ordered trees $\prec$
is uniquely determined from $<$. Give an explicit definition.
%%
\vplatz
\exercise
Let $x\, L\; y$ if $x$ and $y$ are sisters and $x \sqsubset y$.
Show that in ordered trees $L$ can be defined with $\sqsubset$
and conversely.
%%%
\vplatz
\exercise
Let $\GT$ be a tree over $A$ and $N$ such that every node that
is not preterminal is at least 2--branching. Let $\vec{x} =
x_0\dotsb x_{n-1}$ be the associated string. Define a set
$C \subseteq n^3$ as follows. $\auf i,j,k\zu \in C$ iff
the least node above $x_i$ and $x_j$ is lower
than the least node above $x_i$ and $x_k$. Further, for $X \in N$,
define $L_X \subseteq n^2$ by $\auf i,j\zu \in L_X$ iff
the least node above $x_i$ and $x_j$ has label $X$. Show that
$C$ uniquely codes the tree structure $\GT$ and $L_X$, $X \in N$,
the labelling. Finally, for every $a \in A$ we have a unary
relation $T_a \subseteq n$ to code the nodes of category $a$.
Axiomatize the trees in terms of the relations $C$, $L_X$, $X \in N$,
and $T_a$, $a \in A$.
%%%
\vplatz
\exercise
Show that a string of length $n$ possesses at most $2^{c n^3}$
different constituent structures for some constant $c$.
