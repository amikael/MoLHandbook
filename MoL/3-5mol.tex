\section{The AB--Calculus}
\label{kap3-3}
%
%
%
We shall now present a calculus to derive all the valid
derivability statements for AB--grammars. 
Notice that the only variable element is the elementary category 
assignment. We choose an alphabet $A$ and an elementary category 
assignment $\zeta$. We write 
%%
\index{$[\alpha]_{\zeta}$, $[\alpha]$}%%
%%%
$[\alpha]_{\zeta}$ for the set of all
unlabelled binary constituent structures over $A$ that have
root category $\alpha$ under some correct $\zeta$--labelling.
As $\zeta$ is completely arbitrary, we shall deal here only 
with the constituent structures obtained by taking away the 
terminal nodes. This eliminates $\zeta$ and $A$, and leaves a class of
purely categorial structures, denoted by $[\alpha]$. Since 
AB--grammars are invertible, for any given constituent structure 
there exists at most one labelling function (with the exception of 
the terminal labels). Now we introduce a binary symbol $\circ$, 
%%%
\index{$\circ$}%%%
%%%
which takes as arguments correctly
$\zeta$--labelled constituent structures. Let $\auf X, \GX, \ell\zu$
and $\auf Y, \GY,m\zu$ such constituent structures
and $X \cap Y = \varnothing$. Then let
%%
\begin{align}
\auf X, \GX,\ell\zu \circ \auf Y, \GY,m\zu := &
        \auf X \cup Y, \GX \cup \GY \cup \{X \cup Y\},n\zu \\
n(z) := & \begin{cases}
\ell(z) & \text{if $z \in \GX$}, \\
m(z) & \text{if $z \in \GY$}, \\
\ell(X) \cdot m(Y) & \text{if $z = X \cup Y$.}
\end{cases}
\end{align}
%%
(In principle, $\circ$ is well defined also if the constituent
structures are not binary branching.) In case where
$X \cap Y \neq \varnothing$ one has to proceed to the disjoint
sum. We shall not spell out the details. With the help of
$\circ$ we shall form terms over $A$, that is, we form the
algebra freely generated by $A$ by means of $\circ$.
To every term we inductively associate a constituent
structure in the following way.
%%
\begin{subequations}
\begin{align}
\alpha^k & := \auf \{0\}, \{\{0\}\}, \auf \{0\}, \alpha\zu\zu \\
(s \circ t)^k & := s^k \circ t^k
\end{align}
\end{subequations}
%%
Notice that $\circ$ has been used with two meanings. Finally, 
we take a look at $[\alpha]$. It denotes classes of binary 
branching constituent structures over $A$. The following holds.
%%%
\begin{subequations}
\begin{align}
[\alpha/\beta] \circ [\beta] & 
	\subseteq [\alpha] \\
[\beta] \circ [\beta\backslash \alpha] &
	\subseteq [\alpha]
\end{align}
\end{subequations}
%%
We abstract now from $A$ and $\zeta$. In place of interpreting $\circ$
as a constructor for constituent structures over $A$ we now
interpret it as a constructor to form constituent structures
over $\Cat_{\mbox{\smtt\tb}, \mbox{\smtt\tf}}(C)$ for some given $C$.
We call a term from categories with the help of $\circ$ a 
\textbf{category complex}.
%%%%%
\index{category complex}%%
%%%%%
Categories are denoted by lower case Greek letters, category
complexes by upper case Greek letters. Inductively, we extend
the interpretation $[-]$ to structures as follows.
%%
\begin{equation}
[\Gamma \circ \Delta] := [\Gamma] \circ
[\Delta] 
\end{equation}
%%
Next we introduce yet another symbol, $\bvdash$. This is a
relation between structures and categories. If $\Gamma$ is
a structure and $\alpha$ a category then $\Gamma \bvdash \alpha$
denotes the fact that for every interpretation in some alphabet
$A$ with category assignment $\zeta$ $[\Gamma] \subseteq [\alpha]$. 
We call the object $\Gamma \bvdash \alpha$ a \textbf{sequent}. %%
%%%
\index{sequent}%%
%%%
The interpretation that we get in this way we call the
\textbf{cancellation interpretation}. %%%
%%%
\index{cancellation interpretation}%%
%%%
Here, categories are inserted as concrete labels which are assigned
to nodes and which are subject to the cancellation interpretation.

We shall now introduce two different calculi, one of which will
be shown to be adequate for the cancellation interpretation. In 
formulating the rules we use the following convention. $\Gamma[\alpha]$ 
above the line means in this connection that $\Gamma$ is a
category complex in which we have fixed a single occurrence of 
$\alpha$. When we write, for example, $\Gamma[\Delta]$ below the line,
then this denotes the result of replacing that occurrence
of $\alpha$ by $\Delta$.
%%
\begin{equation}
\begin{array}{l@{\quad}l@{\qquad}l@{\quad}l}
\mbox{\rm (ax)} & \alpha \bvdash \alpha &
\mbox{\rm (cut)} & \begin{array}{c}
                \Gamma \bvdash \alpha \qquad \Delta[\alpha] \bvdash \beta
                        \\\hline
                    \Delta[\Gamma]  \bvdash \beta
                \end{array} \\
\mbox{\rm (\textbf{I}--{\mtt{\tf}})} & \begin{array}{c}
                \Gamma \circ \alpha \bvdash \beta \\\hline
                \Gamma \bvdash \beta/\alpha
              \end{array} 
	      	&
\mbox{\rm ({\mtt{\tf}}--\textbf{I})} & \begin{array}{c}
        \Gamma \bvdash \alpha \qquad \Delta[\beta] \bvdash \gamma \\\hline
            \Delta[\beta/\alpha \circ \Gamma] \bvdash \gamma
              \end{array} \\
\mbox{\rm (\textbf{I}--{\mtt{\tb}})} & \begin{array}{c}
                \alpha \circ \Gamma \bvdash \beta \\\hline
                \Gamma \bvdash \alpha \backslash \beta
                \end{array} 
		& 
\mbox{\rm ({\mtt{\tb}}--\textbf{I})} & \begin{array}{c}
        \Gamma \bvdash \alpha \qquad \Delta[\beta] \bvdash \gamma \\\hline
            \Delta[\Gamma \circ \alpha \backslash \beta] \bvdash \gamma
                \end{array}
\end{array}
\end{equation}
%%
%%%
\index{$\mathsf{AB}$, $\mathsf{AB} + \mbox{\rm (cut)}$, $\mathsf{AB}^{-}$}%%%
%%%%
We denote the above calculus by $\mathsf{AB} + \mbox{\rm (cut)}$, 
and by $\mathsf{AB}$ the calculus without (cut). Further, the 
calculus consisting of (ax) and the rules ({\mtt{\tb}}--\textbf{I}) and 
({\mtt{\tf}}--\textbf{I}) is called $\mathsf{AB}^{-}$.
%%%
\begin{defn}
%%%
\index{category!basic}%%
\index{category!distinguished}%%
\index{categorial sequent grammar}%%
%%%
Let $M$ be a set of category constructors.
A \textbf{categorial sequent grammar} is a quintuple
%%%%
\begin{equation}
G = \auf \mbox{\tt S}, C, \zeta, A, \CS\zu
\end{equation}
%%%
where $C$ is a finite
set, the set of \textbf{basic categories}, $\mbox{\tt S} \in C$ the
so called \textbf{distinguished category}, $A$ a finite set,
the alphabet, $\zeta \colon A \pf \wp(\Cat_M(C))$ a category 
assignment, and $\CS$ a sequent calculus. We write $\vdash_G \vec{x}$ 
if for some category complex 
$\Gamma$ whose associated string via $\zeta$ is $\vec{x}$ we have
$\stackrel{\CS}{\rightsquigarrow} \Gamma \bvdash \mbox{\tt S}$.
\end{defn}
%%
We stress here that the sequent calculi are calculi to derive sequents. 
A sequent corresponds to a grammatical rule, or, more precisely, 
the sequent $\Gamma \bvdash \alpha$ expresses the fact that a 
category complex of type $\Gamma$ is a constituent that has the 
category $\alpha$ by the rules of the grammar. The rules of the 
sequent calculus can then be seen as {\it metarules}, which allow 
to pass from one valid statement 
concerning the grammar to another. 
%%
\begin{prop}[Correctness]
\label{korrektheit}
If $\Gamma \bvdash \alpha$ is derivable in $\mathsf{AB}^{-}$
then $[\Gamma] \subseteq [\alpha]$.
\end{prop}
%%
$\mathsf{AB}$ is strictly stronger than $\mathsf{AB}^-$. Notice namely
that the following sequent is derivable in $\mathsf{AB}$:
%%
\begin{equation}
\alpha \bvdash (\beta/\alpha)\backslash\beta
\end{equation}
%%
In natural deduction style calculi this corresponds to the
following unary rule:
%%
\begin{equation}
\begin{array}{c}
\alpha \\\hline
(\beta/\alpha)\backslash\beta
\end{array}
\end{equation}
%%
This rule is known as \textbf{type raising},
%%%
\index{type raising}%%
%%%
since it allows to proceed from the category $\alpha$ to the
``raised'' category $(\beta/\alpha)\backslash\beta$. Perhaps 
one should better call it {\it category raising}, but the other name 
is standard. To see that it is not derivable in $\mathsf{AB}^-$
we simply note that it is not correct for the cancellation
interpretation. We shall return to the question of interpretation
of the calculus $\mathsf{AB}$ in the next section.

An important property of these calculi is their decidability.
Given $\Gamma$ and $\alpha$ we can decide in finite time whether or
not $\Gamma \bvdash \alpha$ is derivable.
%%
\begin{thm}[Cut Elimination]
%%%
\index{Cut Elimination}%%%
%%%
There exists an algorithm to construct a proof of a sequent 
$\Gamma \bvdash \alpha$ in $\mathsf{AB}$ from a proof of 
$\Gamma \bvdash \alpha$ in $\mathsf{AB} + \mbox{\rm (cut)}$. 
Hence (cut) is admissible for $\mathsf{AB}$.
\end{thm}
%%%
\proofbeg
We presented a rather careful proof of Theorem~\ref{thm:cutelimination}, 
so that here we just give a sketch to be filled in accordingly.
We leave it to the reader to verify that each of the operations 
reduces the cut--weight.
We turn immediately to the case where the cut is on a main formula 
of a premiss. The first case is that the formula is introduced by
(\textbf{I}--{\mtt{\tf}}).
%%
\begin{equation}
\begin{array}{ccc}
\Gamma \circ \alpha \bvdash \beta & \qquad & \\\cline{1-1}
\Gamma \bvdash \beta/\alpha & & \Delta[\beta/\alpha] \bvdash \gamma
\\\hline
\multicolumn{3}{c}{\Delta[\Gamma] \bvdash \gamma}
\end{array}
\end{equation}
%%
Now look at the rule instance that is immediately above
$\Delta[\beta/\alpha] \bvdash \gamma$. There are several cases.
Case (0). The premiss is an axiom. Then $\gamma = \beta/\alpha$, and
the cut is superfluous. Case (1). $\beta/\alpha$ is a main formula
of the right hand premiss. Then $\Delta[\beta/\alpha]
= \Theta[\beta/\alpha \circ \Xi]$ for some $\Theta$ and $\Xi$,
and the instance of the rule was as follows.
%%
\begin{equation}
\label{eq:schn1}
\begin{array}{c}
\Xi \bvdash \alpha \qquad \Theta[\beta] \bvdash \gamma \\\hline
\Theta[\beta/\alpha \circ \Xi] \bvdash \gamma
\end{array}
\end{equation}
%%
Now we can restructure \eqref{eq:schn1} as follows.
%%
\begin{equation}
\begin{array}{ccc}
\Gamma \circ \alpha \bvdash \beta \qquad \Theta[\beta] \bvdash \gamma &
\qquad & \\\cline{1-1}
\Theta[\Gamma \circ \alpha] \bvdash \gamma & & \Xi \bvdash \alpha \\\hline
\multicolumn{3}{c}{\Theta[\Gamma \circ \Xi] \bvdash \gamma}
\end{array}
\end{equation}
%%
Now we assume that the formula is not a main formula of the right hand 
premiss. Case (2).  $\gamma = \zeta/\varepsilon$ and the premiss is 
obtained by application of (\textbf{I}--{\mtt{\tf}}).
%%
\begin{equation}
\label{eq:schn2}
\begin{array}{rl}
\Delta[\beta/\alpha] \circ \varepsilon & \bvdash \zeta \\\hline
\Delta[\beta/\alpha] & \bvdash \zeta/\varepsilon
\end{array}
\end{equation}
%%
We replace \eqref{eq:schn2} by
%%
\begin{equation}
\begin{array}{ccc}
\Gamma \circ \alpha \bvdash \beta & \qquad & \\\cline{1-1}
\Gamma \bvdash \beta/\alpha & & \Delta[\beta/\alpha] \circ \varepsilon
        \bvdash \zeta \\\hline
\multicolumn{3}{c}{%
\begin{array}{c}
\Delta[\Gamma] \circ \varepsilon \bvdash \zeta \\\hline
\Delta[\Gamma] \bvdash \varepsilon/\zeta
\end{array}}
\end{array}
\end{equation}
%%
Case (3). $\gamma = \varepsilon\backslash \zeta$ and has been obtained
by applying the rule (\textbf{I}--{\mtt{\tb}}). Then proceed as in
Case (2). Case (4). The application of the rule introduces a formula
which occurs in $\Delta$. This case is left to the reader.

Now if the left hand premiss has been obtained by
({\mtt{\tb}}--\textbf{I}), then one proceeds quite analogously.
So, we assume that the left hand premiss is created by an
application of ({\mtt{\tb}}--\textbf{I}).
%%
\begin{equation}
\label{eq:schn3}
\begin{array}{ccc}
\Gamma \bvdash \alpha & \Delta[\beta] \bvdash \gamma & \\\cline{1-2}
\multicolumn{2}{c}{%
\Delta[\beta/\alpha \circ \Gamma] \bvdash \gamma}
 & \Theta[\gamma] \bvdash \delta \\\hline
\multicolumn{3}{c}{\Theta[\Delta[\beta/\alpha \circ \Gamma]] %
\bvdash \delta}
\end{array}
\end{equation}
%%
We can restructure \eqref{eq:schn3} as follows.
%%
\begin{equation}
\begin{array}{ccc}
   & \Delta[\beta] \bvdash \gamma & 
	\Theta[\gamma] \bvdash \delta \\\cline{2-3}
\Gamma \bvdash \alpha & 
	\multicolumn{2}{c}{\Theta[\Delta[\beta]] \bvdash \delta}
\\\hline
\multicolumn{3}{c}{\Theta[\Delta[\beta/\alpha \circ \Gamma]] \bvdash \delta}
\end{array}
\end{equation}
%%
Also here one calculates that the degree of the new cut is less than
the degree of the old cut. The case where the left hand premiss is
created by ({\mtt{\tb}}--\textbf{I}) is analogous. All cases have
now been looked at.
\proofend
%%
\begin{cor}
$\mathsf{AB} + \mbox{\rm (cut)}$ is decidable.
\proofend
\end{cor}
%%
$\mathsf{AB}$ gives a method to test category complexes for their syntactic 
category. We expect that the meanings of the terms are likewise 
systematically connected with a term and that we can determine the 
meaning of a certain string once we have found a derivation for it. 
We now look at the rules of $\mathsf{AB}$ to see how they can be used 
as rules for deducing sign--sequents. Before we start we shall 
distinguish two interpretations of the calculus. The first is the 
intrinsic interpretation: every sequent we derive should be 
correct, with all basic parts of it belonging to the original 
lexicon. The second is the global interpretation: the sequents 
we derive should be correct if the lexicon was suitably expanded. 
This only makes a difference with respect to signs with empty 
exponent. If a lexicon has no such signs the intrinsic interpretation 
bans their use altogether, but the global interpretation leaves 
room for their addition. Adding them, however, will make more 
sequents derivable that are based on the original lexicon only.

We also write $\vec{x} : \alpha : M$ for the sign 
$\auf \vec{x}, \alpha, M\zu$. If $\vec{x}$, $\alpha$ or 
$M$ is irrelevant in the context it is omitted. For the meanings we 
use $\lambda$--terms, which are however only proxy for the `real' 
meanings (see the discussion at the end of the preceding section). 
Therefore we now write $(\lambda x.xy)$ in place of 
{\tt ($\lambda$x$_{\snull}$.(x$_{\snull}$x$_{\seins}$))}. 
A \textbf{sign complex} 
%%%
\index{sign complex}%%%
%%%
is a term made of signs with the help of $\circ$. 
Sequents are pairs $\Gamma \bvdash \tau$ where $\Gamma$ is a sign 
complex and $\tau$ a sign. $\sigma$ maps categories to types, as in 
Section~\ref{kap3}.\ref{kap3-2}. If $\vec{x} : \alpha : M$ is derivable, we 
want that $M$ is of type $\sigma(\alpha)$. Hence, the rules of 
$\mathsf{AB}$ should preserve this property. We define first a relation 
%%%
\index{$\succ$, $\Vdash$}%%
%%%
$\Vdash$ between sign complexes. It proceeds by means of the following 
rules. 
%%%
\begin{align}
\label{eq:leftcancel}
\Gamma[\vec{x} : \alpha/\beta : M \circ \vec{y} : \beta : N] 
 & \succ \Gamma[\vec{x}\conc\vec{y} : \alpha : (MN)]
\\
\label{eq:rightcancel}
\Gamma[\vec{x} : \beta : M \circ \vec{y} : \beta\backslash\alpha : N] 
 & \succ \Gamma[\vec{x}\conc\vec{y} : \alpha : (NM)] 
\end{align}
%%%
Since $M$ and $N$ are actually functions and not $\lambda$--terms, 
one may exchange any two $\lambda$--terms that denote the same 
function. However, if one considers them being actual 
$\lambda$--terms, then the following rule has to be added:
%%%
\begin{equation}
	\Gamma[\vec{x} : \alpha : M] \succ
	\Gamma[\vec{x} : \alpha : N] 
	\qquad \text{if $M \equiv N$}
\end{equation}
%%%
For $\Gamma$ a sign complex and $\sigma$ a sign put 
$\Gamma \Vdash \sigma$ iff $\Gamma \succ^{\ast} \sigma$.
We want to design a calculus that generates all and only 
the sequents $\Gamma \bvdash \sigma$ such that 
$\Gamma \Vdash \sigma$.

To begin, we shall omit the strings and deal with the 
meanings. Later we shall turn to the strings, which pose 
independent problems. The axioms are as follows.
%%
\begin{equation}
\mbox{\rm (ax)} \qquad 
\alpha  : M \bvdash \alpha : M
\end{equation}
%%
where $M$ is a term of type $\sigma(\alpha)$. (cut) looks like this.
%%
\begin{equation}
{\rm (cut)}\quad\begin{array}{c}
\Gamma \bvdash \alpha : N \qquad
    \Delta[\alpha : x_{\eta}]
    \bvdash \beta : M
\\\hline
\Delta[\Gamma] \bvdash \beta : [N/x_{\eta}]M
\end{array}
\end{equation}
%%
So, if $\Delta[\alpha : x_{\eta}]$ is a sign complex containing 
an occurrence of $\alpha : x_{\eta}$, then the occurrence of this 
sign complex is replaced and with it the variable $x_{\eta}$ in 
$M$. So, semantically speaking cut is substitution. Notice that 
since we cannot tell which occurrence in $M$ is to be replaced, 
we have to replace all of them. We will see that there are reasons 
to require that every variable has exactly one occurrence, so that 
this problem will never arise. (We could make this a condition on (cut). 
But see below for the fate of (cut).) The other rules are more 
complex. 
%%
\begin{equation}
\mbox{\rm ({\mtt{\tf}}--\textbf{I})}\quad
\begin{array}{c}
\Gamma \bvdash \alpha : M \qquad \Delta[\beta : x_{\zeta}] \bvdash \gamma :
    N \\\hline
\Delta[\beta/\alpha : x_{\eta \pf \zeta} \circ \Gamma] \bvdash \gamma :
    [(x_{\eta \pf \zeta}M)/x_{\zeta}]N
\end{array}
\end{equation}
%%
This corresponds to the replacement of a primitive constituent by
a complex constituent or the replacement of a value $M(x)$ by the
pair $\auf M,x\zu$. Here, the variable $x_{\eta \pf \zeta}$ is
introduced, which stands for a function from objects of type 
$\eta$ to objects of type $\zeta$. The variable $x_{\zeta}$ has, 
however, disappeared. This is a serious deficit of the calculus 
(which has other advantages, however). We shall below develop a 
different calculus. Analogously for the rule ({\mtt{\tb}}--\textbf{I}). 
%%%
\begin{equation}
\begin{array}{ll}
\mbox{\rm (\textbf{E}--{\mtt{\tf}})} &
\quad\begin{array}{c}
        \Gamma \bvdash \alpha : M
    \quad \Delta \bvdash \beta/\alpha : N \\\hline
            \Delta \circ \Gamma \bvdash \beta : (NM)
              \end{array} \\
\mbox{\rm (\textbf{E}--{\mtt{\tb}})} &
\quad\begin{array}{c}
        \Gamma \bvdash \alpha : M \quad
    \Delta \bvdash \alpha\backslash \beta : N \\\hline
            \Gamma \circ \Delta \bvdash \beta : (NM)
                \end{array}
\end{array}
\end{equation}
%%%
\begin{lem}
\label{lem:e-der}
The rule \mbox{\rm (\textbf{E}--{\mtt{\tf}})} is derivable from 
(cut) and \mbox{\rm ({\mtt{\tf}}--\textbf{I})}. Likewise, the rule 
\mbox{\rm (\textbf{E}--{\mtt{\tf}})} is derivable from 
(cut) and \mbox{\rm ({\mtt{\tf}}--\textbf{I})}. 
\end{lem}
%%%
\proofbeg
The following is an instance of \mbox{\rm ({\mtt{\tf}}--\textbf{I})}. 
%%%
\begin{equation}
\begin{array}{c}
\beta : x_{\zeta} \bvdash \beta : x_{\zeta} \qquad
	\alpha : x_{\eta} \bvdash \alpha :x_{\eta} 
\\\hline
\alpha/\beta : x_{\zeta\pf\eta} \circ \beta : x_{\zeta} 
	\bvdash \alpha : x_{\eta} 
\end{array}
\end{equation}
%%%
Now two cuts, with $\alpha : M \bvdash \alpha : M$ and with 
$\beta : N \bvdash \beta :N$, give \mbox{\rm (\textbf{E}--{\mtt{\tf}})}.
\proofend

Thus, the rules \eqref{eq:leftcancel} and \eqref{eq:rightcancel} 
are accounted for.

The rules (\textbf{I}--{\mtt{\tf}}) and (\textbf{I}--{\mtt{\tb}}) 
can be interpreted as follows. Assume that $\Gamma$ is a constituent of
category $\alpha/\beta$. 
%%
\begin{equation}
\mbox{\rm (\textbf{I}--{\mtt{\tf}})}\qquad
\begin{array}{c}
\Gamma \circ \alpha : x_{\eta} \bvdash \beta : M
    \\\hline
\Gamma \bvdash \beta/\alpha :
    (\lambda x_{\eta}. M)
\end{array} 
\end{equation}
%%
Here, the meaning of $\Gamma$ is of the form $M$ and the
meaning of $\alpha$ is $N$. Notice that $\mathsf{AB}$ forces us 
in this way to view the meaning of a word of category $\alpha/\beta$ 
to be a function from $\eta$--objects to $\zeta$--objects.
For it is formally required that $\Gamma$ has to have the
meaning of a function. We call the rules (\textbf{I}--{\mtt{\tf}}) 
and (\textbf{I}--{\mtt{\tb}}) also \textbf{abstraction rules}.
These rules have to be restricted, however. Define for a variable 
$x$ and a term $M$, 
%%%
\index{$\focc(x,M)$}%%
%%%
$\focc(x,M)$ to be the number of free occurrences of $x$ in $M$.
In the applications of the introduction rules, we add a side 
condition: 
%%%
\begin{equation}
\text{In }\mbox{\rm (\textbf{I}--{\mtt{\tf}})} \text{ and }
\mbox{\rm (\textbf{I}--{\mtt{\tb}})} : 
\focc(x_{\eta},M) \leq 1 
\end{equation}
%%%
(In fact, one can show that from this condition already follows 
$\focc(x_{\eta},M) = 1$, by induction on the proof.) To see the 
need for this restriction, look at the following derivation.
%%%
%\begin{equation}
$$\begin{array}{r@{\;\bvdash\;}l}
(\beta/\alpha/\alpha : M \circ \alpha : x_{\eta}) \circ 
	\alpha : x_{\eta} & \beta : (Mx_{\eta})x_{\eta} \\\hline
\beta/\alpha/\alpha : M \circ \alpha : x_{\eta} & 
	\beta/\alpha : \lambda x_{\eta}.(Mx_{\eta})x_{\eta} \\\hline
\beta/\alpha/\alpha : M & 
	\beta/\alpha/\alpha : \lambda x_{\eta}.\lambda x_{\eta}.%
((Mx_{\eta})x_{\eta})
\end{array}$$
%\end{equation}
%%%
The first is obtained using two applications of the derivable 
(\textbf{E}--{\mtt{\tf}}). 

This rule must be further restricted, however, as the next 
example shows. In the rule ({\mtt{\tb}}--\textbf{I}) put 
$\Delta := \gamma := \beta$.
%%
\begin{equation}
\begin{array}{c}
\alpha : x_{\eta} \bvdash \alpha : x_{\eta} \qquad
    \beta : x_{\zeta} \bvdash \beta : x_{\zeta} \\\hline
\alpha/\beta : x_{\zeta \pf \eta} \circ
    \beta : x_{\zeta} \bvdash \alpha :
    (x_{\zeta \pf \eta}x_{\zeta})
\end{array}
\end{equation}
%%
Using (\textbf{I}--{\mtt{\tf}}), we get 
%%
\begin{equation}
\begin{array}{c}
\alpha/\beta : x_{\zeta \pf \eta} \circ \beta : x_{\zeta}
    \bvdash \alpha : (x_{\zeta \pf \eta}x_{\zeta})
\\\hline
\alpha/\beta : x_{\zeta \pf \eta} \bvdash \alpha/\beta :
    (\lambda x_{\zeta}.(x_{\zeta \pf \eta}x_{\zeta}))
\end{array}
\end{equation}
%%
Now $\lambda x_{\zeta}.x_{\zeta \pf \eta} x_{\zeta}$
is the same function as $x_{\zeta \pf \eta}$. On the other hand, by applying
(\textbf{I}--{\mtt{\tb}}) we get
%%
\begin{equation}
\begin{array}{c}
\alpha/\beta : x_{\zeta \pf \eta} \circ \beta : x_{\zeta}
    \bvdash \alpha : (x_{\zeta \pf \eta}x_{\zeta})
\\\hline
\beta : x_{\zeta}\bvdash (\alpha/\beta)\backslash \alpha :
    (\lambda x_{\zeta \pf \eta}.(%
    x_{\zeta \pf \eta}x_{\zeta}))
\end{array}
\end{equation}
%%
This is the type raising rule which we have discussed above.
A variable $x_{\zeta}$ can also be regarded as a function, which 
for given function $f$ taking arguments of type $\zeta$ returns 
the value $f(x_{\zeta})$. However, $x_{\zeta}$ is not the same 
function as $(\lambda x_{\zeta \pf \eta}.(x_{\zeta \pf\eta}x_{\zeta}))$.
(The latter has the type $(\beta \pf \alpha) \pf \alpha$.)
Therefore the application of the rule is incorrect in this
case. Moreover, in the typed $\lambda$--calculus the equation 
$x_{\zeta} \doteq (\lambda x_{\zeta \pf \eta}.(%
    x_{\zeta \pf \eta}x_{\zeta}))$ is invalid. 

To remedy the situation we must require that the variable
which we have abstracted over appears on the left hand side of
$\bvdash$ in the premiss as an argument variable and not as a
variable of a function that is being applied to something. So,
the final form of the right slash--introduction rule is as follows.
%%
\begin{equation}
\mbox{\rm (\textbf{I}--{\mtt{\tf}})}\quad
\begin{array}{c}
\Gamma \circ \alpha : x_{\eta} \bvdash \beta : M
    \\\hline
\Gamma \bvdash \beta/\alpha :
    (\lambda x_{\eta}. M)
\end{array} \quad
\begin{array}{l}
\mbox{$x_{\eta}$ an argument variable,} \\
\text{and $\focc(x_{\eta},M) \leq 1$}
\end{array}
\end{equation}
%%
How can one detect whether $x_{\eta}$ is an argument variable?
To this end we require that the sequent $\Gamma \bvdash \beta/\alpha$ 
be derivable in categorial $\mathsf{AB}^-$. This seems paradoxical. 
For with this restriction the calculus seems to be as weak as 
$\mathsf{AB}^-$. Why should one make use of the rule
(\textbf{I}--{\mtt{\tf}}) if the sequent is anyway derivable? To 
understand this one should take note of the difference between the
categorial calculus and the interpreted calculus. We allow the use of
the interpreted rule (\textbf{I}--{\mtt{\tf}}) if $\Gamma \bvdash
\beta/\alpha$ is derivable in the categorial calculus; or, to
be more prosaic, if $\Gamma$ has the category $\beta/\alpha$ and
hence the type $\alpha \pf \beta$. That this indeed strengthens
the calculus can be seen as follows. In the interpreted $\mathsf{AB}^-$ 
the following sequent is not derivable (though it is derivable in 
$\mathsf{AB}$). The proof of this claim is left as an exercise. 
%%
\begin{equation}
\alpha/\beta : x_{\zeta \pf \eta} \bvdash
\alpha/\beta : \lambda x_{\zeta}.x_{\zeta\pf\eta}x_{\zeta}
\end{equation}
%%
We assign to a sign complex a sign as follows. 
%%
%%%
\index{$\S(\Gamma)$}%%%
%%%%
\begin{align}
\notag
\S(\vec{x} : \alpha : M) & := \auf \vec{x}, \alpha, M\zu \\
\S(\vec{x} : \alpha/\beta : M \circ \vec{y} : \beta : N) 
        & := \mbox{\mtt A}_{\sgr}(\auf \vec{x}, \alpha/\beta, M\zu, 
	\auf \vec{y}, \beta, N\zu) \\
\notag
\S(\vec{x} : \beta : M \circ \vec{y} : \beta\backslash \alpha : N) 
	& := \mbox{\mtt A}_{\skl}(\auf \vec{x}, \beta, M\zu, 
	\auf \vec{y}, \beta\backslash\alpha, N\zu) 
\end{align}
%%
It is easy to see that if $\Gamma \bvdash \alpha : M$ is derivable 
in the interpreted $\mathsf{AB}$ then $\S(\Gamma) = \auf \vec{x}, \alpha, %
M'\zu$ for some $M' \equiv M$. (Of course, $M$ and $M'$ are just notational 
variants denoting the same object. Thus they are identical qua 
objects they represent.) 

The calculus that has just been defined has drawbacks. We will see 
below that (cut) cannot be formulated for strings. Thus, we have 
to do without it. But then we cannot derive the rules 
\mbox{\rm (\textbf{E}--{\mtt{\tf}})} and 
\mbox{\rm (\textbf{E}--{\mtt{\tb}})}. The calculus $\mathsf{N}$ 
obviates the need for that.
%%
\begin{defn}
%%%
\index{$\mathsf{N}$}%%
%%%
The calculus $\mathsf{N}$ has the rules \mbox{(ax)}, 
\mbox{\rm (\textbf{I}--{\mtt{\tf}})}, \mbox{\rm (\textbf{E}--{\mtt{\tf}})} 
\mbox{\rm (\textbf{I}--{\mtt{\tf}})} and 
\mbox{\rm (\textbf{E}--{\mtt{\tb}})}. 
\end{defn}
%%
In (the interpreted) $\mathsf{N}$, (cut) is admissible. (The 
proof of that is left as an exercise.) 

Now let us turn to strings. Now we omit the interpretation, since
it has been dealt with. Our objects are now written as $\vec{x} :
\alpha$ where $\vec{x}$ is a string and $\alpha$ a category. The
reader is reminded of the fact that $\vec{y}/\vec{x}$ denotes that
string which results from $\vec{y}$ by removing the postfix
$\vec{x}$. This is clearly defined only if $\vec{y} =
\vec{u}\conc\vec{x}$ for some $\vec{u}$, and then we have
$\vec{y}/\vec{x} = \vec{u}$. Analogously for $\vec{x}\backslash
\vec{y}$.
%%
\begin{equation}
$$\begin{array}{l@{\quad}l@{\quad}l@{\quad}l}
\mbox{(ax)} & \vec{x} : \alpha \bvdash \vec{x} : \alpha & \\
\mbox{\rm (\textbf{I}--{\mtt{\tf}})} &
        \begin{array}{c}
                \Gamma \circ \vec{x} : \alpha \bvdash \vec{y} : \beta
        \\\hline
                \Gamma \bvdash \vec{y}/ \vec{x} : \beta/\alpha
              \end{array} &
\mbox{\rm (\textbf{E}--{\mtt{\tf}})} & \begin{array}{c}
        \Gamma \bvdash \vec{x} : \alpha
   	 \quad 
    \Delta \bvdash \vec{y} : \beta/\alpha  \\\hline
            \Delta \circ \Gamma \bvdash
        \vec{y} \conc \vec{x} : \beta
              \end{array} \\
\mbox{\rm (\textbf{I}--{\mtt{\tb}})} &
        \begin{array}{c}
                \vec{x} : \alpha \circ \Gamma \bvdash \vec{y} : \beta
        \\\hline
                \Gamma \bvdash \vec{x}\backslash \vec{y} :
        \alpha \backslash \beta
                \end{array} &
\mbox{\rm (\textbf{E}--{\mtt{\tb}})} & \begin{array}{c}
        \Gamma \bvdash \vec{x} : \alpha \quad
    \Delta \bvdash \vec{y} : \alpha\backslash \beta \\\hline
            \Gamma \circ \Delta \bvdash \vec{x} \conc \vec{y} : \beta
        \end{array}
\end{array}$$
\end{equation}
%%
The cut rule is however no more a rule of the calculus. There
is no formulation of it at all. Suppose we try to formulate a 
cut rule. Then it would go as follows.
%%
\begin{equation}
\begin{array}{c}
                \Gamma \bvdash \vec{x} : \alpha \qquad
        \Delta[\vec{y} : \alpha] \bvdash \vec{z} : \beta
                        \\\hline
                    \Delta[\Gamma]  \bvdash [\vec{y}/\vec{x}]
            \vec{z} : \beta
                \end{array}
\end{equation}
%%
Here, $[\vec{y}/\vec{x}]\vec{z}$ denotes the result of replacing
$\vec{y}$ for $\vec{x}$ in $\vec{z}$. So, on the strings (cut) 
becomes constituent replacement. Notice that only one occurrence 
may be replaced, so if $\vec{x}$ occurs several times, the result 
of the operation $[\vec{y}/\vec{x}]\vec{z}$ is not uniquely defined. 
Moreover, $\vec{x}$ may occur accidentally in $\vec{z}$! Thus, it 
is not clear which of the occurrences is the right one to be replaced. 
So the rule of (cut) cannot even be properly formulated. On 
the other hand, semantically it is admissible, so for the semantics 
we can do without it anyway. However, the same problem of 
substitution arises with the rules ({\mtt{\tf}}--\textbf{I}) and 
({\mtt{\tb}}--\textbf{I}). Thus, they had to be eliminated as 
well.

This completes the definition of the sign calculus $\mathsf{N}$.
Call $\mathsf{E}$ the calculus consisting of just ({\mtt{\tf}}--\textbf{E}) 
%%%%
\index{$\mathsf{E}$}%%%
%%%%
and ({\mtt{\tb}}--\textbf{E}). Based on Lemma~\ref{lem:e-der} the 
completeness of $\mathsf{E}$ for $\Vdash$ is easily established.
%%%
\begin{thm}
$\stackrel{\mathsf{E}}{\leadsto} \Gamma \bvdash \sigma$ iff 
$\Gamma \Vdash \sigma$.
\end{thm}
%%
$\mathsf{N}$ is certainly correct for the global interpretation, 
but it is correct for the intrinsic interpretation? The answer 
is actually yes! The fact is that the introduction rules are 
toothless tigers: they can only eliminate a variable that has 
never had a chance to play a role. For assume that we have an 
$\mathsf{N}$--proof. If (\textbf{I}--{\mtt{\tf}}) is used, let 
the highest application be as follows. 
%%%
\begin{equation}
\begin{array}{c}
\Gamma \circ \varepsilon : \alpha : x_{\eta} \bvdash \vec{y} : \beta : M
    \\\hline
\Gamma \bvdash \vec{y} : \beta/\alpha : (\lambda x_{\eta}. M)
\end{array} 
\end{equation}
%%%
Then the sequent above the line has been introduced by 
(\textbf{E}--{\mtt{\tf}}): 
%%%
\begin{equation}
\begin{array}{c}
\Gamma \bvdash \vec{y} : \beta/\alpha : N 
	\qquad \varepsilon : \alpha : x_{\eta} 
	\bvdash \varepsilon : \alpha : x_{\eta} 
    \\\hline
\Gamma \circ \varepsilon : \alpha : x_{\eta} \bvdash 
	\vec{y} : \beta : Nx_{\eta}
\end{array} 
\end{equation}
%%%
Here, $N x_{\eta} = M$. Since $(\lambda x_{\eta}.M) = 
(\lambda x_{\eta}.Nx_{\eta}) = N$, this part of the proof can 
be eliminated.
%%%
\begin{thm}
The rules \mbox{\rm (\textbf{I}--{\mtt{\tf}})} and 
\mbox{\rm (\textbf{I}--{\mtt{\tb}})} are admissible in $\mathsf{E}$.
\end{thm}
%%%
In the next section, however, we shall study the effect of adding 
associativity. In presence of associativity the introduction 
rules actually do considerable work. In that case, to regain 
correctness, we can either ban the introduction rules, or we 
can restrict the axioms. Given an AB--sign grammar $\GA$ we can 
restrict the set of axioms to 
%%
\begin{equation}
\mbox{\rm (ax$_{\GA}$)} \quad \upsilon(f) \bvdash \upsilon(f)
    \qquad \text{where $f \in F$ and $\Omega(f) = 0$}
\end{equation}
%%
For an AB--grammar does not possess any modes of the form 
$\auf \varepsilon, \alpha, x_{\alpha}\zu$ where $x_{\alpha}$
is a variable. 
%%
\vplatz
\exercise
Prove the correctness theorem, Proposition~\ref{korrektheit}.
%%
\vplatz
\exercise
Define a function $p$ from category complexes to categories as follows.
%%
\begin{equation}
\begin{split}
p(\alpha) & := \alpha \\
p(\Gamma \circ \Delta) & := p(\Gamma) \cdot p(\Delta)
\end{split}
\end{equation}
%%
Show that $\Gamma \bvdash \alpha$ is derivable in $\mathsf{AB}^-$ 
iff $p(\Gamma) = \alpha$. Show that this also holds for $\mathsf{AB}^- + %
\mbox{(cut)}$. Conclude from this that (cut) is admissible 
in (categorial!) $\mathsf{AB}^-$. (This could in principle be extracted 
from the proof for $\mathsf{AB}$, but this proof here is quite 
simple.)
%%
%\vplatz
%\exercise
%Prove the completeness of \mathsf{AB} with respect to the 
%cancellation interpretation. This means that
%if for all $A$ and all $\zeta$ $[\Gamma]_{\zeta} \subseteq
%[\alpha]_{\zeta}$ then $\Gamma \bvdash \alpha$ is derivable
%in \textsf{AB}.
%%
\vplatz
\exercise
Show that every CFL can be generated by an AB--grammar 
using only two basic categories. 
%%
\vplatz
\exercise
Show the following claim: in the interpreted
$\mathsf{AB}^-$--calculus no sequents are derivable
which contain bound variables.
%%
\vplatz
\exercise
\label{ueb:nschnitt}
Show that (cut) is admissible for $\mathsf{N}$.
