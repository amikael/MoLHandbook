\newcommand{\strictif}{\supset}
\section{Propositional Logic}
\label{kap:prop}
%
%
%
Before we can enter a discussion of categorial grammar and type
systems, we shall have to introduce some techniques from
propositional logic. We seize the opportunity to present boolean
logic using our notions of the previous section. The alphabet is
defined to be $A_P := \{\mbox{\mtt p}, \mbox{\mtt 0}, \mbox{\mtt 1},
\mbox{\mtt (}, \mbox{\mtt )}, \sbot, \mbox{\mtt\symbol{25}}\}$. 
Further, let $T := \{P\}$, and $M := \{0,1\}$. Next, we define 
the following modes. The zeroary modes are
%%%%
\begin{equation}
\mbox{\mtt X}_{\vec{\alpha}} := \auf \mbox{\mtt p}\vec{\alpha},
    P, 0\zu, \quad 
\mbox{\mtt Y}_{\vec{\alpha}} := \auf \mbox{\mtt p}\vec{\alpha},
    P, 1\zu, \quad 
\mbox{\mtt M}_{\bot} := \auf \sbot, P, 0\zu 
\end{equation}
%%%
Here, $\vec{\alpha}$ ranges over (possibly empty) sequences of
{\mtt 0} and {\mtt 1}. (So, the signature is infinite.) Further, 
let $\strictif$ be the following function:
%%
\begin{equation}
\begin{array}{l|ll}
\strictif & 0 & 1 \\\hline
0       & 1 & 1 \\
1       & 0 & 1
\end{array}
\end{equation}
%%
The binary mode $\mbox{\tt M}_{\mbox{\smtt\symbol{25}}}$ of implication 
formation is spelled out as follows.
%%
\begin{equation}
\mbox{\tt M}_{\mbox{\smtt\symbol{25}}}(\auf \vec{x}, P, \eta\zu,
    \auf \vec{y}, P, \theta\zu)
:= 
    \auf \mbox{\mtt ($\vec{x}$\symbol{25}$\vec{y}$)}, P,
    \eta \strictif \theta\zu
\end{equation}
%%%
The system of signs generated by these modes is called
\textbf{boolean logic}
%%%
\index{logic!boolean}%%
%%%
and is denoted by $\Sigma_{\mathsf{B}}$. To see that this is
indeed so, let us explain in more conventional terms what these
definitions amount to. First, the string language $L$ we have
defined is a subset of $A_P^{\ast}$, which is generated as follows.
%%
\begin{dingautolist}{192}
\item If $\vec{\alpha} \in \{\mbox{\mtt 0}, \mbox{\mtt 1}\}^{\ast}$,
    then $\mbox{\mtt p}\vec{\alpha} \in L$. These sequences are
    called \textbf{propositional variables}.
%%%
\index{variable!propositional}%%
%%%
\item $\sbot\in L$.
\item If $\vec{x}, \vec{y}\in L$ then $\mbox{\mtt ($\vec{x}$\symbol{25}$%
\vec{y}$)}\in L$.
\end{dingautolist}
%%
$\vec{x}$ is also called a \textbf{well--formed formula} (\textbf{wff}) 
or simply a \textbf{formula} 
%%%
\index{formula!well--formed}%%
\index{wff}%%
%%%
iff it belongs to $L$. There are three kinds of wffs.
%%%
\begin{defn}
\index{tautology}%%
\index{contradiction}%%
\index{formula!contingent}%%
%%%
Let $\vec{x}$ be a well--formed formula. $\vec{x}$ is a 
\textbf{tautology} if $\auf \vec{x}, P, 0\zu \not\in \Sigma_{\mathsf{B}}$. 
$\vec{x}$ is a \textbf{contradiction} if $\auf\vec{x}, P, 1\zu
\not\in \Sigma_{\mathsf{B}}$. If $\vec{x}$ is neither a tautology
nor a contradiction, it is called \textbf{contingent}.
\end{defn}
%%%
The set of tautologies is denoted by 
%%%%
\index{$\Taut_{\mathsf{B}}(\mbox{\mtt\symbol{25}},\sbot)$}%% 
%%%
$\Taut_{\mathsf{B}}(\mbox{\mtt\symbol{25}},\sbot)$, 
or simply by $\Taut_{\mathsf{B}}$ if the
language is clear from the context. It is easy to see that
$\vec{x}$ is a tautology iff {\mtt ($\vec{x}$\symbol{25}$\bot)$} 
is a contradiction. Likewise, $\vec{x}$ is a contradiction iff 
{\mtt ($\vec{x}$\symbol{25}$\bot)$} is a tautology. We now agree 
on the following convention. Lower case Greek letters are proxy 
for well--formed formulae, upper case Greek letters are proxy for 
sets of formulae. Further, we write $\Delta; \varphi$ instead of
$\Delta\cup \{\varphi\}$ and $\varphi;\chi$ in place of
$\{\varphi, \chi\}$.

Our first task will be to present a calculus with which we can
generate all the tautologies of $\Sigma_{\mathsf{B}}$. For this
aim we use a so--called {\it Hilbert style calculus}. Define the
following sets of formulae.
%%
\index{Hilbert (style) calculus}
%%
\begin{equation}
\begin{array}{ll}
\mbox{\rm (a0)} & \mbox{\mtt ($\varphi$\symbol{25}($\psi$\symbol{25}%
$\varphi$))} \\
\mbox{\rm (a1)} & \mbox{\mtt (($\varphi$\symbol{25}($\psi$\symbol{25}%
$\chi$))\symbol{25}(($\varphi$\symbol{25}$\psi$)\symbol{25}(%
$\varphi$\symbol{25}$\chi$)))} \\
\mbox{\rm (a2)} & \mbox{\mtt ($\sbot$\symbol{25}$\varphi$)} \\
\mbox{\rm (a3)} & \mbox{\mtt ((($\varphi$\symbol{25}$\sbot$)\symbol{25}%
$\sbot$)\symbol{25}$\varphi$)}
\end{array}
\end{equation}
%%%
The logic axiomatized by (a0) -- (a3) is known as \textbf{classical} 
or \textbf{boolean logic},
%%%
\index{logic!classical}\index{logic!boolean}%%
%%%
the logic axiomatized by (a0) -- (a2) as \textbf{intuitionistic logic}.
%%%%
\index{intuitionistic logic}%%
\index{logic!intuitionistic}%%%
%%%%
To be more precise, (a0) -- (a3) each are sets of formulae. For example:
\begin{equation}
\mbox{\rm (a0)} = \{\mbox{\mtt ($\varphi$\symbol{25}($\psi%
$\symbol{25}$\varphi$))} : \varphi, \psi \in L\}
\end{equation}
%%%
We call (a0) an \textbf{axiom schema} and
%%%
\index{axiom schema}%%%
\index{axiom schema!instance}%%
%%%
its elements \textbf{instances of} (a0). Likewise with (a1) -- (a3).
%%%
\begin{defn}
%%%
\index{proof}
\index{proof!length of a \faul}%%%
\index{$\vdash^{\mathsf{B}} \varphi$}%%%
%%%%
A finite sequence $\Pi = \auf \delta_i : i < n\zu$ of formulae 
is a $\mathsf{B}$--\textbf{proof of} $\varphi$ if (a) $\delta_{n-1}
= \varphi$ and (b) for all $i < n$ either (b1) $\delta_i$ is an
instance of (a0) -- (a3) or (b2) there are $j, k < i$ such that
$\delta_k = \mbox{\mtt ($\delta_j$\symbol{25}$\delta_i$)}$. The 
number $n$ is called the \textbf{length of} $\Pi$. We write 
$\vdash^{\mathsf{B}} \varphi$ if there is a $\mathsf{B}$--proof of 
$\varphi$.
\end{defn}
%%%
The formulae (a0) -- (a3) are called the \textbf{axioms} of this
%%%
\index{axiom}%%
\index{Modus Ponens (MP)}%%
%%%
calculus. Moreover, this calculus uses a single inference rule,
which is known as Modus Ponens. It is the inference from 
{\mtt ($\varphi$\symbol{25}$\chi$)} and $\varphi$ to
$\chi$. The easiest part is to show that the calculus generates
only tautologies.
%%
\begin{lem}
If $\vdash^{\mathsf{B}} \varphi$ then $\varphi$ is a tautology.
\end{lem}
%%
The proof is by induction on the length of the proof. The completeness
part is somewhat harder and requires a little detour. We shall extend 
the notion of proof somewhat to cover proofs from assumptions.
%%%
\begin{defn}
%%%
\index{proof}%%
%%%
A $\mathsf{B}$--\textbf{proof of} $\varphi$ \textbf{from} $\Delta$ is a
finite sequence $\Pi = \auf \delta_i : i < n\zu$ of formulae such
that (a) $\delta_{n-1} = \varphi$ and (b) for all $i < n$ either
(b1) $\delta_i$ is an instance of (a0) -- (a3) or (b2) there are
$j, k < i$ such that $\delta_k = \mbox{\mtt ($\delta_j$\symbol{25}%
$\delta_i$)}$ or (b3) $\delta_i \in \Delta$. We write
$\Delta \vdash^{\mathsf{B}} \varphi$ if there is a 
$\mathsf{B}$--proof of $\varphi$ from $\Delta$.
\end{defn}
%%%
To understand this notion of a hypothetical proof, we shall introduce
the notion of an \textbf{assignment}. It is common to define an assignment
to be a function from variables to the set $\{0,1\}$. Here, we
shall give an effectively equivalent definition.
%%%
\begin{defn}
%%%
\index{assignment}%%
%%%
An \textbf{assignment} is a maximal subset $A$ of
%%
\begin{equation}
\{\mbox{\mtt X}_{\vec{\alpha}}
: \vec{\alpha} \in (\mbox{\mtt 0} \cup \mbox{\mtt 1})^{\ast}\}
\cup
\{\mbox{\mtt Y}_{\vec{\alpha}}
: \vec{\alpha} \in (\mbox{\mtt 0} \cup \mbox{\mtt 1})^{\ast}\}
\end{equation}
%%
such that for no $\vec{\alpha}$ both
$\mbox{\mtt X}_{\vec{\alpha}}, \mbox{\mtt Y}_{\vec{\alpha}}
\in A$.
\end{defn}
%%%
(So, an assignment is a set of zeroary modes.) Each assignment
defines a closure under the modes $\mbox{\mtt M}_{\bot}$ and
$\mbox{\mtt M}_{\mbox{\smtt\symbol{25}}}$, which we denote by 
%%%%
\index{$\Sigma_{\mathsf{B}}(A)$}%%%
%%%%
$\Sigma_{\mathsf{B}}(A)$.
%%%
\begin{lem}
Let $A$ be an assignment and $\varphi$ a well--formed formula. Then
either $\auf\varphi, P,0\zu \in \Sigma_{\mathsf{B}}(A)$ or
$\auf\varphi, P, 1\zu \in \Sigma_{\mathsf{B}}(A)$, but not both.
\end{lem}
%%
The proof is by induction on the length of $\vec{x}$. We say that
an assignment $A$ makes a formula $\varphi$ \textbf{true} if $\auf
\varphi, P, 1\zu \in \Sigma_{\mathsf{B}}(A)$.
%%%
\index{truth}%%
%%%
\begin{defn}
Let $\Delta$ be a set of formulae and $\varphi$ a formula. We say
that $\varphi$ \textbf{follows from} (or \textbf{is a consequence of})
%%%%
\index{consequence}%%
\index{$\vDash$}%%
%%%%
$\Delta$ if for all assignments $A$: if $A$ makes all formulae of
$\Delta$ true then it makes $\varphi$ true as well.
In that case we write $\Delta \vDash \varphi$.
\end{defn}
%%%
Our aim is to show that the Hilbert calculus characterizes this
notion of consequence:
%%%
\begin{thm}
\label{thm:hilbertmain}%%
$\Delta \vdash^{\mathsf{B}} \varphi$ iff $\Delta
\vDash \varphi$.
\end{thm}
%%%
Again, the proof has to be deferred until the matter is
sufficiently simplified. Let us first show the following
fact, known as the \textbf{Deduction Theorem} (\textbf{DT}).
%%%
%%%
\begin{lem}[Deduction Theorem]
\index{Deduction Theorem}
$\Delta; \varphi \vdash^{\mathsf{B}} \chi$ iff $\Delta
\vdash^{\mathsf{B}}
    \mbox{\mtt ($\varphi$\symbol{25}$\chi$)}$.
\end{lem}
%%%
\proofbeg 
The direction from right to left is immediate and left
to the reader. Now, for the other direction suppose that $\Delta;
\varphi \vdash^{\mathsf{B}} \chi$. Then there exists a proof $\Pi
= \auf \delta_i : i < n\zu$ of $\chi$ from $\Delta;\varphi$.  We
shall inductively construct a proof $\Pi' = \auf \delta'_j : j <
m\zu$ of {\mtt ($\varphi$\symbol{25}$\chi$)} from $\Delta$.
The construction is as follows. We define $\Pi_i$ inductively. 
%%
\begin{equation}
\Pi_0 := \varepsilon, \qquad \Pi_{i+1} := \Pi_i^{\smallfrown}\Sigma_i,
\end{equation}
%%
where $\Sigma_i$, $i < n$, is defined as given below. Furthermore,
we will verify inductively that $\Pi_{i+1}$ is a proof of its last
formula, which is {\mtt ($\varphi$\symbol{25}$\delta_i$)}.
Then $\Pi' := \Pi_n$ will be the desired proof, since $\delta_{n-1}
= \chi$.  Choose $i < n$. Then either (1) $\delta_i \in \Delta$ or
(2) $\delta$ is an instance of (a0) -- (a3) or (3) $\delta_i = \varphi$
or (4) there are $j, k < i$ such that $\delta_k = \mbox{\mtt ($\delta_j$%
\symbol{25}$\delta_i$)}$. In the first two cases we put
$\Sigma_i := \auf\delta_i, \mbox{\mtt ($\delta_i$\symbol{25}(%
$\varphi$\symbol{25}$\delta_i$))}, \mbox{\mtt ($\varphi$%
\symbol{25}$\delta_i$)}\zu$. In Case (3) we put
%%
\begin{align}
%\begin{array}{l@{}l}
\Sigma_i := \quad \auf &
\mbox{\mtt (($\varphi$\symbol{25}(($\varphi$\symbol{25}$\varphi$%
)\symbol{25}$\varphi$))\symbol{25}(($\varphi$\symbol{25}(%
$\varphi$\symbol{25}$\varphi$))\symbol{25}($\varphi$\symbol{25}%
$\varphi$)))}, \\\notag
&
    \mbox{\mtt ($\varphi$\symbol{25}(($\varphi$\symbol{25}$\varphi$%
)\symbol{25}$\varphi$))}, \\\notag
&
    \mbox{\mtt (($\varphi$\symbol{25}($\varphi$\symbol{25}$\varphi$))%
\symbol{25}($\varphi$\symbol{25}$\varphi$))}, \\\notag
&
    \mbox{\mtt ($\varphi$\symbol{25}($\varphi$\symbol{25}%
$\varphi$))}, \\\notag
&
    \mbox{\mtt ($\varphi$\symbol{25}$\varphi$)}\zu
\end{align}
%%
$\Sigma_i$ is a proof of {\mtt ($\varphi$\symbol{25}$\varphi$)},
as is readily checked. Finally, Case (4). There are $j, k < i$ such
that $\delta_k = \mbox{\mtt ($\delta_j$\symbol{25}$\delta_i$)}$.
Then, by induction hypothesis, {\mtt ($\varphi$\symbol{25}$\delta_j$)}
and $\mbox{\mtt ($\varphi$\symbol{25}$\delta_k$)} =
\mbox{\mtt ($\varphi$\symbol{25}($\delta_j$\symbol{25}$\delta_i$))}$
already occur in the proof. Then put
%%
\begin{align}
%\begin{array}{l@{}l}
\Sigma_i := \quad \auf &
\mbox{\mtt (($\varphi$\symbol{25}($\delta_j$\symbol{25}$\delta_i$))%
\symbol{25}(($\varphi$\symbol{25}$\delta_j$)\symbol{25}($\varphi$%
\symbol{25}$\delta_i$)))}, \\\notag
&
    \mbox{\mtt (($\varphi$\symbol{25}$\delta_j$)\symbol{25}(%
$\varphi$\symbol{25}$\delta_i$))}, \\\notag
&
    \mbox{\mtt ($\varphi$\symbol{25}$\delta_i$)}
    \zu
\end{align}
%%
It is verified that $\Pi_{i+1}$ is a proof of {\mtt ($\varphi$\symbol{25}%
$\delta_i$)}.
\proofend

A special variant is the following.
%%%
\begin{lem}[Little Deduction Theorem]
%%%
\index{Little Deduction Theorem}%%%
%%%
For all $\Delta$ and $\varphi$: $\Delta\vdash^{\mathsf{B}}\varphi$ 
if and only if $\Delta;\mbox{\mtt ($\varphi$\symbol{25}$\sbot$)}%
\vdash^{\mathsf{B}}\sbot$.
\end{lem}
%%%
\proofbeg 
Assume that $\Delta\vdash^{\mathsf{B}}\varphi$. Then
there is a proof $\Pi$ of $\varphi$ from $\Delta$. It follows that 
$\Pi^{\smallfrown}\auf \mbox{\mtt ($\varphi$\symbol{25}$\sbot$)},
\sbot\zu$ is a proof of $\sbot$ from $\Delta;\mbox{\mtt ($\varphi$
\symbol{25}$\sbot$)}$. Conversely, assume that
$\Delta;\mbox{\mtt ($\varphi$\symbol{25}$\sbot$)} \vdash^{\mathsf{B}}
\sbot$. Applying DT we get $\Delta\vdash^{\mathsf{B}}\mbox{\mtt
(($\varphi$\symbol{25}$\sbot$)\symbol{25}$\sbot$)}$. Using (a3) we
get $\Delta\vdash^{\mathsf{B}}\varphi$. 
\proofend
%%%
\begin{prop}
\label{prop:deduce}
The following holds.
\begin{dingautolist}{192}
\item $\varphi \vdash^{\mathsf{B}} \varphi$.
\item If $\Delta \subseteq  \Delta'$ and
    $\Delta \vdash^{\mathsf{B}}\varphi$
    then also $\Delta' \vdash^{\mathsf{B}} \varphi$.
\item If $\Delta \vdash^{\mathsf{B}} \varphi$ and
    $\Gamma; \varphi \vdash^{\mathsf{B}} \chi$
    then $\Gamma;\Delta \vdash^{\mathsf{B}} \chi$.
\end{dingautolist}
\end{prop}
%%%
This is easily verified. Now we are ready for the proof of
Theorem~\ref{thm:hilbertmain}. An easy induction on the length of
a proof establishes that if $\Delta \vdash^{\mathsf{B}} \varphi$
then also $\Delta \vDash \varphi$. (This is called the {\it
correctness\/} of the calculus.) So, the converse implication,
which is the {\it completeness\/} part needs proof. Assume that
$\Delta \nvdash^{\mathsf{B}} \varphi$. We shall show that also
$\Delta \nvDash \varphi$. Call a set $\Sigma$ \textbf{consistent} 
(\textbf{in} $\vdash^{\mathsf{B}}$)  if
%%%
\index{set!consistent}%%
%%%
$\Sigma \nvdash^{\mathsf{B}} \sbot$.
%%%
\begin{lem}
\label{lem:tableau}
\begin{dingautolist}{192}
\item
Let $\Delta;\mbox{\mtt ($\varphi$\symbol{25}$\chi$)}$ be consistent.
Then either $\Delta;\mbox{\mtt ($\varphi$\symbol{25}$\sbot$)}$ is
consistent or $\Delta;\chi$ is consistent.
\item
Let $\Delta;\mbox{\mtt (($\varphi$\symbol{25}$\chi$)\symbol{25}%
$\sbot$)}$ be consistent. Then also $\Delta;\varphi;\mbox{\mtt 
($\chi$\symbol{25}$\sbot$)}$ is consistent.
\end{dingautolist}
\end{lem}
%%%
\proofbeg 
\ding{192}. Assume that both $\Delta;\mbox{\mtt ($\varphi$\symbol{25}%
$\sbot$)}$ and $\Delta;\chi$ are inconsistent. Then we have $\Delta;\mbox{\mtt
($\varphi$\symbol{25}$\sbot$)} \vdash^{\mathsf{B}} \sbot$ and
$\Delta;\chi\vdash^{\mathsf{B}} \sbot$. So
$\Delta\vdash^{\mathsf{B}} \mbox{\mtt (($\varphi$\symbol{25}%
$\sbot$)\symbol{25}$\sbot$)}$ by DT and, using (a3), 
$\Delta\vdash^{\mathsf{B}} \varphi$. Hence $\Delta;\mbox{\mtt %
($\varphi$\symbol{25}$\chi$)}%
\vdash^{\mathsf{B}} \varphi$ and so $\Delta;\mbox{\mtt
($\varphi$\symbol{25}$\chi$)} \vdash^{\mathsf{B}} \chi$. Because
$\Delta;\chi\vdash^{\mathsf{B}} \sbot$, we also have
$\Delta;\mbox{\mtt ($\varphi$\symbol{25}$\chi$)}\vdash^{\mathsf{B}}
\sbot$, showing that $\Delta;\mbox{\mtt ($\varphi$\symbol{25}$\chi$)}$ 
is inconsistent. \ding{193}. Assume $\Delta;\varphi;\mbox{\mtt ($\chi$%
\symbol{25}$\sbot$)}$ is inconsistent. Then 
$\Delta;\varphi;\mbox{\mtt ($\chi$\symbol{25}%
$\sbot$)}\vdash^{\mathsf{B}} \sbot$. So,
$\Delta;\varphi\vdash^{\mathsf{B}} \mbox{\mtt
(($\chi$\symbol{25}$\sbot$)\symbol{25}$\sbot$)}$, by applying DT.
So, $\Delta;\varphi\vdash^{\mathsf{B}}\chi$, using (a3). Applying
DT we get $\Delta \vdash^{\mathsf{B}} \mbox{\mtt
($\varphi$\symbol{25}$\chi$)}$. Using (a3) and DT once again it is
finally seen that $\Delta;\mbox{\mtt (($\varphi$\symbol{25}$\chi$%
)\symbol{25}$\sbot$)}$ is inconsistent. 
\proofend

Finally, let us return to our proof of the completeness theorem.
We assume that $\Delta \nvdash^{\mathsf{B}} \varphi$. We have to
find an assignment $A$ that makes $\Delta$ true but not
$\varphi$. We may also apply the Little DT and assume that
$\Delta;\mbox{\mtt ($\varphi$\symbol{25}$\sbot$)}$ is consistent and
find an assignment that makes this set true. The way to find such
an assignment is by applying the so--called downward closure of
the set.
%%%
\begin{defn}
\index{set!downward closed}%%
%%
A set $\Delta$ is \textbf{downward closed} iff (1) for
all $\mbox{\mtt ($\varphi$\symbol{25}$\chi$)} \in \Delta$ either
$\mbox{\mtt ($\varphi$\symbol{25}$\sbot$)} \in \Delta$ or $\chi \in
\Delta$ and (2) for all formulae $\mbox{\mtt (($\varphi$\symbol{25}$\chi$%
)\symbol{25}$\sbot$)} \in \Delta$ also $\varphi \in \Delta$ and 
$\mbox{\mtt ($\chi$\symbol{25}$\sbot$)} \in \Delta$.
\end{defn}
%%%
Now, by Lemma~\ref{lem:tableau} every consistent set has a consistent
closure $\Delta^{\ast}$. (It is an exercise for the diligent reader
to show this. In fact, for infinite sets a little work is needed here,
but we really need this only for finite sets.) Define the following
assignment.
%%
\begin{align}
%\begin{array}{l@{}l@{}l}
A \quad := & \phantom{\mbox{}\cup\mbox{}}
    \{\auf \mbox{\mtt p}\vec{\alpha}, P, 1\zu :
    \text{\mtt (p$\vec{\alpha}$\symbol{25}$\sbot$)}
    \text{ does not occur in }\Delta^{\ast}\} \\\notag
        & \cup
    \{\auf \mbox{\mtt p}\vec{\alpha}, P, 0\zu :
    \text{\mtt (p$\vec{\alpha}$\symbol{25}$\sbot$)}
    \text{ does occur in }\Delta^{\ast}\}
\end{align}
%%
It is shown by induction on the formulae of $\Delta^{\ast}$ that
the so--defined assignment makes every formula of $\Delta^{\ast}$
true. Using the correspondence between syntactic derivability
and semantic consequence we immediately derive the following.
%%%
\begin{thm}[Compactness Theorem]
%%%%
\index{Compactness Theorem}%%
%%%%
Let $\varphi$ be a formula and $\Delta$ a set of formulae such
that $\Delta \vDash \varphi$. Then there exists a finite set $\Delta'
\subseteq \Delta$ such that $\Delta' \vDash \varphi$.
\end{thm}
%%
\proofbeg 
Suppose that $\Delta \vDash \varphi$. Then $\Delta
\vdash^{\mathsf{B}} \varphi$. Hence there exists a proof of
$\varphi$ from $\Delta$. Let $\Delta'$ be the set of those
formulae in $\Delta$ that occur in that proof. $\Delta'$ is
finite. Clearly, this proof is a proof of $\varphi$ from
$\Delta'$, showing $\Delta' \vdash^{\mathsf{B}} \varphi$. Hence
$\Delta' \vDash \varphi$. 
\proofend

Usually, one has more connectives than just $\sbot$ and 
{\mtt\symbol{25}}. Now, two effectively equivalent strategies suggest 
themselves, and they are used whenever convenient. The first is to 
introduce a new connective as an abbreviation. So, we might define 
(for well--formed formulae)
%%
\begin{align}
\nicht\varphi & := \mbox{\mtt $\varphi$\symbol{25}$\sbot$} \\
\varphi\oder\chi & := \mbox{\mtt ($\varphi$\symbol{25}$\sbot$)%
\symbol{25}$\chi$} \\
\varphi\und\chi & := \mbox{\mtt ($\varphi$\symbol{25}($\chi$%
\symbol{25}$\sbot$))\symbol{25}$\sbot$} 
\end{align}
%%
After the introduction of these abbreviations, everything is
the same as before, because we have not changed the language,
only our way of referring to its strings. However, we may also
change the language by expanding the alphabet. In the cases at hand
we will add the following unary and binary modes (depending on which
symbol is to be added):
%%
\begin{align}
\mbox{\mtt M}_{\mbox{\smtt\symbol{5}}}(\auf \vec{x}, P, \eta\zu) &
    := \auf \mbox{\mtt (\symbol{5}$\vec{x}$)}, P, -\eta\zu \\
\mbox{\mtt M}_{\mbox{\smtt\symbol{31}}}(\auf \vec{x}, P, \eta\zu,
    \auf \vec{y}, P, \theta\zu)
    &
    := \auf \mbox{\mtt ($\vec{x}$\symbol{31}$\vec{y}$)}, P,
    \eta \cup \theta\zu \\
\mbox{\mtt M}_{\mbox{\smtt\symbol{4}}}(\auf \vec{x}, P, \eta\zu,
    \auf \vec{y}, P, \theta\zu)
    &
    := \auf \mbox{\mtt ($\vec{x}$\symbol{4}$\vec{y}$)}, P,
    \eta \cap \theta\zu
\end{align}
%%%
\begin{equation}
\begin{array}{l|ll}
\cup & 0 & 1 \\\hline
0    & 0 & 1 \\
1    & 1 & 1
\end{array}\qquad
\begin{array}{l|ll}
\cap & 0 & 1 \\\hline
0    & 0 & 0 \\
1    & 0 & 1
\end{array}\qquad
\begin{array}{l|l}
  & - \\\hline
0 & 1 \\
1 & 0
\end{array}
\end{equation}
%%
For {\mtt\symbol{4}}, {\mtt\symbol{25}} and {\mtt\symbol{5}} 
we need the postulates shown in \eqref{eq:p4}, \eqref{eq:p25} 
and \eqref{eq:p5}, respectively:
%%
\begin{align}
\label{eq:p4}
\mbox{\mtt ($\varphi$\symbol{25}($\psi$\symbol{25}($\varphi$%
\symbol{4}$\psi$)))}, &
\mbox{\mtt ($\varphi$\symbol{25}($\psi$\symbol{25}($\psi$%
\symbol{4}$\varphi$)))}, \\\notag
\mbox{\mtt (($\varphi$\symbol{4}$\psi$)\symbol{25}$\varphi$)},
&
\mbox{\mtt (($\varphi$\symbol{4}$\psi$)\symbol{25}$\psi$)} \\
\label{eq:p25}
\mbox{\mtt ($\varphi$\symbol{25}($\varphi$\symbol{31}$\psi$))},
& 
\mbox{\mtt ($\psi$\symbol{25}($\varphi$\symbol{31}$\psi$))}, \\\notag
\mbox{\mtt ((($\varphi$\symbol{31}$\psi$)\symbol{25}$\chi$)%
\symbol{25}($\varphi$\symbol{25}$\chi$))},
& 
\mbox{\mtt ((($\varphi$\symbol{31}$\psi$)\symbol{25}$\chi$)%
\symbol{25}($\psi$\symbol{25}$\chi$))} \\
\label{eq:p5}
\mbox{\mtt (($\varphi$\symbol{25}$\psi$)\symbol{25}%
((\symbol{5}$\psi$)\symbol{25}(\symbol{5}$\varphi$)))},
& 
\mbox{\mtt ($\varphi$\symbol{25}(\symbol{5}(\symbol{5}$\varphi$)))}
\end{align}
%%
Notice that in defining the axioms we have made use of {\mtt\symbol{25}}
alone. The formula \eqref{eq:classneg} is derivable. 
%%%
\begin{equation}
\label{eq:classneg}
\mbox{\mtt ((\symbol{5}(\symbol{5}$\varphi$))\symbol{25}$\varphi$)}
\end{equation}
%%%
If we eliminate the connective $\sbot$ and define $\Delta \vdash \varphi$
as before (eliminating the axioms (a2) and (a3), however) we get
once again intuitionistic logic, unless we add \eqref{eq:classneg}. 
The semantics of intuitionistic logic 
is too complicated to be explained here, so we just use the Hilbert 
calculus to introduce it. We claim that with only (a0) and (a1) it 
is not possible to prove all formulae of $\Taut_{\mathsf{B}}$ 
that use only {\mtt\symbol{25}}. A case in point is the formula
%%
\begin{equation}
\mbox{\mtt ((($\varphi$\symbol{25}$\chi$)\symbol{25}$\varphi$)%
\symbol{25}$\varphi$)}
\end{equation}
%%
which is known as {\it Peirce's Formula}. Together with Peirce's
Formula, (a0) and (a1) axiomatize the full set of tautologies of
boolean logic in {\mtt\symbol{25}}. The calculus based on (a0) and 
(a1) is called \textsf{H} and we write $\Delta \vdash^{\mathsf{H}} \chi$ 
to say that there is a proof in the Hilbert calculus of $\chi$ from
$\Delta$ using (a0) and (a1).

Rather than axiomatizing the set of tautologies we can also axiomatize 
the deducibility relation itself. This idea goes back to Gerhard 
Gentzen, who used it among other to show the consistency of arithmetic 
(which is of no concern here). For simplicity, we stay with the language 
with only the arrow. We shall axiomatize the derivability of intuitionistic 
logic. The statements that we are deriving now have the form 
`$\Delta  \boldsymbol{\vdash} \varphi$' and are called \textbf{sequents}. 
$\Delta$ is called the \textbf{antecedent} and $\varphi$ the 
\textbf{succedent} of that sequent.
%%%
\index{sequent}%%
\index{antecedent}%%
\index{succedent}%%
%%%
The axioms are
%%
\begin{equation}
\mbox{\rm (ax)} \quad \varphi \bvdash \varphi
\end{equation}
%%
Then there are the following rules of introduction of connectives:
%%
\begin{equation}
\mbox{\rm (\textbf{I}{\mtt\symbol{25}})}\quad
\begin{array}{c}
    \Delta;\varphi\bvdash\chi\\\hline
    \Delta\bvdash\mbox{\mtt ($\varphi$\symbol{25}$\chi$)}
    \end{array}
\qquad
\mbox{\rm ({\mtt\symbol{25}}\textbf{I})}\quad
\begin{array}{c}
    \Delta\bvdash\varphi \qquad \Delta;\psi\bvdash\chi\\\hline
    \Delta;\mbox{\mtt ($\varphi$\symbol{25}$\psi$)}\bvdash
        \chi
    \end{array}
\end{equation}
%%
Notice that these rules introduce occurrences of the arrow. The
rule (\textbf{I}{\mtt\symbol{25}}) introduces an occurrence on the 
right hand side of $\bvdash$, while ({\mtt\symbol{25}}\textbf{I}) 
puts an occurrence on the left hand side. (The names of the rules 
are chosen accordingly.) Further, there are the following so--called 
rules of inference:
%%
\begin{equation}
\mbox{\rm (cut)}\quad
\begin{array}{c}
\Delta\bvdash\varphi \qquad \Theta;\varphi \bvdash \chi \\\hline
\Delta;\Theta \bvdash\chi
\end{array}
\qquad
\mbox{\rm (mon)}\quad
\begin{array}{c}
\Delta\bvdash\varphi \\\hline
\Delta;\Theta\bvdash\varphi
\end{array}
\end{equation}
%%%
\index{premiss}%%
\index{conclusion}%%
\index{formula!main}%%
\index{formula!cut--\faul}%%
\index{Gentzen calculus}%%
%%%
The sequents above the line are called the \textbf{premisses}, the sequent
below the lines the \textbf{conclusion} of the rule. Further, the
formulae that are introduced by the rules ({\mtt\symbol{25}}\textbf{I}) 
and (\textbf{I}{\mtt\symbol{25}}) are called \textbf{main formulae}, 
and the formula $\varphi$ in (cut) the \textbf{cut--formula}. Let 
us call this the \textbf{Gentzen calculus}. It is denoted by $\CH$.
%%%
\begin{defn}
%%%
\index{sequent proof}%%
%%%
Let $\Delta \bvdash \varphi$ be a sequent. A (\textbf{sequent}) 
\textbf{proof of length} $n$ of $\Delta \bvdash \varphi$ in $\CH$ is a
sequence $\Pi = \auf \Sigma_i \bvdash \chi_i : i < n+1\zu$ such
that (a) $\Sigma_n = \Delta$, $\chi_n = \varphi$, (b) for all $i <
n+1$ either (ba) $\Sigma_i \bvdash \chi_i$ is an axiom or (bb)
$\Sigma_i \bvdash \chi_i$ follows from some earlier sequents by
application of a rule of $\CH$.
\end{defn}
%%%
It remains to say what it means that a sequent follows from some
other sequents by application of a rule. This, however, is straightforward.
For example, $\Delta \bvdash \mbox{\mtt ($\varphi$\symbol{25}$\chi$%
)}$ follows from the earlier sequents by application of the rule
(\textbf{I}{\mtt\symbol{25}}) if among the earlier sequents we find 
the sequent $\Delta;\varphi \bvdash \chi$. We shall define also a different
notion of proof, which is based on trees rather than sequences.
In doing so, we shall also formulate a somewhat more abstract
notion of a calculus.
%%%
\begin{defn}
%%%
\index{rule}%%
\index{rule!finitary}%%%
\index{sequent calculus}%%
\index{proof tree}%%
%%%
A \textbf{finitary rule} is a pair $\rho = \auf M, \GS\zu$, where
$M$ is a finite set of sequents and $\GS$ a single sequent. (These
rules are written down using lower case Greek letters as schematic
variables for formulae and upper case Greek letters as schematic
variables for sets of formulae.) A \textbf{sequent calculus} $\CS$
is a set of finitary rules. An $\CS$--\textbf{proof tree} is a
triple $\BT = \auf T, \succ, \ell\zu$ such that $\auf T, \prec\zu$
is a tree and for all $x$: if $\{y_i : i < n\}$ are the daughters
of $T$, $\auf \{\ell(y_i) : i < n\}, \ell(x)\zu$ is an instance of
a rule of $\CS$. If $r$ is the root of $\BT$, we say that $\BT$
\textbf{proves} $\ell(r)$ \textbf{in} $\CS$. We write
%%%
\begin{equation}
\stackrel{\CS}{\rightsquigarrow} \Delta \bvdash \varphi
\end{equation}
%%%
to say that the sequent $\Delta\bvdash \varphi$ has a proof in $\CS$.
\end{defn}
%%%
We start with the only rule for $\sbot$, which actually is an axiom.
%%%
\begin{equation}
\mbox{\rm ($\sbot$\textbf{I})}\quad \sbot \bvdash \varphi
\end{equation}
%%
For negation we have these rules.
%%
\begin{equation}
\mbox{\rm ({\mtt\symbol{5}}\textbf{I})}\quad
\begin{array}{c}
\Delta \bvdash \varphi \\\hline
\Delta; \mbox{\mtt (\symbol{5}$\varphi$)} \bvdash \sbot
\end{array}
\qquad
\mbox{\rm (\textbf{I}{\mtt\symbol{5}})}\quad
\begin{array}{c}
\Delta; \varphi \bvdash \sbot \\\hline
\Delta \bvdash \mbox{\mtt (\symbol{5}$\varphi$)}
\end{array}
\end{equation}
%%
The following are the rules for conjunction.
%%
\begin{equation}
\mbox{\rm ({\mtt\symbol{4}}\textbf{I})}\quad
\begin{array}{c}
\Delta;\varphi;\psi \bvdash \chi \\\hline
\Delta;\mbox{\mtt ($\varphi$\symbol{4}$\psi$)} \bvdash \chi
\end{array}
\qquad
\mbox{\rm (\textbf{I}{\mtt\symbol{4}})}\quad
\begin{array}{c}
\Delta \bvdash \varphi \qquad \Delta \bvdash \psi\\\hline
\Delta \bvdash \mbox{\mtt ($\varphi$\symbol{4}$\psi$)}
\end{array}
\end{equation}
%%
Finally, these are the rules for {\mtt\symbol{31}}.
%%
\begin{equation}
\begin{array}{l}
\mbox{\rm ({\mtt\symbol{31}}\textbf{I})}\quad
\begin{array}{c}
\Delta; \varphi \bvdash \chi \qquad \Delta; \psi \bvdash \chi \\\hline
\Delta; \mbox{\mtt ($\varphi$\symbol{31}$\psi$)} \bvdash \chi
\end{array} \\
\mbox{\rm (\textbf{I}$_1${\mtt\symbol{31}})}\quad
\begin{array}{c}
\Delta \bvdash \varphi \\\hline
\Delta \bvdash \mbox{\mtt ($\varphi$\symbol{31}$\psi$)}
\end{array}
\qquad
\mbox{\rm (\textbf{I}$_2${\mtt\symbol{31}})}\quad
\begin{array}{c}
\Delta \bvdash \psi \\\hline
\Delta \bvdash \mbox{\mtt ($\varphi$\symbol{31}$\psi$)}
\end{array}
\end{array}
\end{equation}
%%
Let us return to the calculus $\CH$. We shall first of all show
that we can weaken the rule system without changing the set of
derivable sequents. Notice that the following is a proof tree.
%%%
\begin{equation}
\begin{array}{c}
\varphi \bvdash \varphi \qquad \psi \bvdash \psi \\\hline
\mbox{\mtt ($\varphi$\symbol{25}$\psi$)};\varphi \bvdash \psi \\\hline
\mbox{\mtt ($\varphi$\symbol{25}$\psi$)}
    \bvdash\mbox{\mtt ($\varphi$\symbol{25}$\psi$)}
\end{array}
\end{equation}
%%%
This shows us that in place of the rule (ax) we may actually use a
restricted rule, where we have only $\mbox{\mtt p}_i \bvdash %
\mbox{\mtt p}_i$.
%%%
\index{axiom!primitive}%%
%%%
Call such an instance of (ax) \textbf{primitive}. This fact may be
used for the following theorem.
%%%
\begin{lem}
$\stackrel{\CH}{\rightsquigarrow} \Delta \bvdash 
\mbox{\mtt ($\varphi$\symbol{25}$\chi$)}$ iff 
$\stackrel{\CH}{\rightsquigarrow} \Delta;\varphi \bvdash\chi$.
\end{lem}
%%%
\proofbeg
From right to left follows using the rule (\textbf{I}{\mtt\symbol{25}}).
Let us prove the other direction. We know that there exists a proof
tree for $\Delta \bvdash \mbox{\mtt ($\varphi$\symbol{25}$\chi$)}$
from primitive axioms. Now we trace backwards the occurrence of
{\mtt ($\varphi$\symbol{25}$\chi$)} in the tree from the root
upwards. Obviously, since the formula has not been introduced
by (ax), it must have been introduced by the rule 
(\textbf{I}{\mtt\symbol{25}}). Let $x$ be the node where the 
formula is introduced. Then we remove $x$ from the tree, thereby 
also removing that instance of (\textbf{I}{\mtt\symbol{25}}). 
Going down from $x$, we have to repair our proof as follows. 
Suppose that at $y < x$ we have an instance of
(mon). Then instead of the proof part to the left we use the one
to the right.
%%%
\begin{equation}
\begin{array}{c}
    \Sigma\bvdash\mbox{\mtt ($\varphi$\symbol{25}$\chi$)} \\\hline
    \Sigma;\Theta\bvdash\mbox{\mtt ($\varphi$\symbol{25}$\chi$)}
\end{array}
\quad
\begin{array}{c}
    \Sigma;\varphi\bvdash\chi \\\hline
    \Sigma;\Theta;\varphi\bvdash\chi
\end{array}
\end{equation}
%%%
Suppose that we have an instance of (cut). Then our specified 
occurrence of {\mtt ($\varphi$\symbol{25}$\chi$)} is the one 
that is on the right of the target sequent. So, in place of the 
proof part on the left we use the one on the right.
%%%
\begin{equation}
\begin{array}{c}
\Delta\bvdash\psi \quad \Theta;\psi \bvdash 
	\mbox{\mtt ($\varphi$\symbol{25}$\chi$)} \\\hline
\Delta;\Theta \bvdash\mbox{\mtt ($\varphi$\symbol{25}$\chi$)}
\end{array}
\qquad
\begin{array}{c}
\Delta\bvdash\psi \quad \Theta;\varphi;\psi \bvdash \chi\\\hline
\Delta;\Theta;\varphi \bvdash\chi
\end{array}
\end{equation}
%%%
Now suppose that we have an instance of ({\mtt\symbol{25}}\textbf{I}). 
Then this instance must be as shown to the left. We replace it by 
the one on the right.
%%%
\begin{equation}
\begin{array}{c}
    \Delta\bvdash\tau \quad \Delta;\psi\bvdash%
    \mbox{\mtt ($\varphi$\symbol{25}$\chi$)} \\\hline
    \Delta;\mbox{\mtt ($\tau$\symbol{25}$\psi$)}\bvdash
        \mbox{\mtt ($\varphi$\symbol{25}$\chi$)}
    \end{array}
\qquad
\begin{array}{c}
\Delta\bvdash\tau \quad \Delta;\phi;\psi\bvdash \chi\\\hline
\Delta;\mbox{\mtt ($\tau$\symbol{25}$\psi$)};\varphi \bvdash \chi
\end{array}
\end{equation}
%%%
The rule ({\mtt\symbol{25}}\textbf{I}) does not occur below $x$, as is 
easily seen. This concludes the replacement. It is verified that after 
performing these replacements, we obtain a proof tree for
$\Delta;\varphi\bvdash\chi$.
\proofend
%%%
\begin{thm}
$\Delta \vdash^{\mathsf{H}} \varphi$ iff
 $\stackrel{\CH}{\rightsquigarrow} \Delta \bvdash \varphi$.
\end{thm}
%%%
\proofbeg 
Suppose that $\Delta \vdash^{\mathsf{H}} \varphi$. By induction on the 
length of the proof we shall show that 
$\stackrel{\CH}{\rightsquigarrow} \Delta \bvdash \varphi$. Using DT we may
restrict ourselves to $\Delta = \varnothing$. First, we shall show
that (a0) and (a1) can be derived. (a0) is derived as follows.
%%
\begin{equation}
$$\begin{array}{r@{\;\bvdash\;}l}
\varphi      & \varphi \\\hline
\varphi;\psi & \varphi \\\hline
\varphi      & \mbox{\mtt ($\psi$\symbol{25}$\varphi$)} \\\hline
             &
    \mbox{\mtt ($\varphi$\symbol{25}($\psi$\symbol{25}$\varphi$))}
\end{array}
\end{equation}
%%
For (a1) we need a little more work.
%%
\begin{equation}
\begin{array}{cccc}
& & \psi \bvdash \psi & \chi \bvdash \chi \\\cline{3-4}
& \varphi \bvdash \varphi &
    \multicolumn{2}{c}{\psi;\mbox{\mtt ($\psi$\symbol{25}$\chi$)}
        \bvdash \chi} \\\cline{2-4}
\varphi \bvdash \varphi &
    \multicolumn{3}{c}{\varphi;\mbox{\mtt ($\varphi$\symbol{25}$\psi$)};
    \mbox{\mtt ($\psi$\symbol{25}$\chi$)} \bvdash \chi}
\\\cline{1-4}
\multicolumn{4}{c}{
    \mbox{\mtt ($\varphi$\symbol{25}($\psi$\symbol{25}$\chi$))};%
\mbox{\mtt ($\varphi$\symbol{25}$\psi$)};
    \varphi \bvdash \chi}
\end{array}
\end{equation}
%%%
If we apply (\textbf{I}{\mtt\symbol{25}}) three times we get (a1). Next 
we have to show that if we have $\stackrel{\CH}{\rightsquigarrow} \varnothing 
\bvdash \phi$ and $\stackrel{\CH}{\rightsquigarrow} \varnothing \bvdash 
\mbox{\mtt ($\chi$\symbol{25}$\phi$)}$ then $\stackrel{\CH}{\rightsquigarrow}
\varnothing \bvdash \chi$. By DT, we also have
$\stackrel{\CH}{\rightsquigarrow} \varphi \bvdash \chi$ and then a single
application of (cut) yields the desired conclusion. This proves
that $\stackrel{\CH}{\rightsquigarrow}\varnothing \bvdash \varphi$. Now,
conversely, we have to show that $\stackrel{\CH}{\rightsquigarrow}
\Delta\bvdash\varphi$ implies that $\Delta \vdash^{\mathsf{H}}
\varphi$. This is shown by induction on the height of the nodes in
the proof tree. If it is 1, we have an axiom: however,
$\varphi\vdash^{\mathsf{H}} \varphi$ clearly holds. Now suppose
the claim is true for all nodes of depth $< i$ and let $x$ be of
depth $i$. Then $x$ is the result of applying one of the four
rules. ({\mtt\symbol{25}}\textbf{I}). By induction hypothesis, $\Delta
\vdash^{\mathsf{H}} \varphi$ and $\Delta;\psi \vdash^{\mathsf{H}}
\chi$. We need to show that 
$\Delta;\mbox{\mtt ($\varphi$\symbol{25}$\psi$)} \vdash^{\mathsf{H}} 
\chi$. Simply let $\Pi_1$ be a proof of $\varphi$ from $\Delta$, $\Pi_2$ 
a proof of $\chi$ from $\Delta;\psi$. Then $\Pi_3$ is a proof of $\chi$
from $\Delta;\mbox{\mtt ($\varphi$\symbol{25}$\psi$)}$.
%%
\begin{equation}
\Pi_3 := \Pi_1^{\smallfrown}\auf\mbox{\mtt ($\varphi$\symbol{25}%
$\psi$)},\psi\zu^{\smallfrown}\Pi_2
\end{equation}
%%
(\textbf{I}{\mtt\symbol{25}}). This is straightforward from DT. (cut). 
Suppose that $\Pi_1$ is a proof of $\varphi$ from $\Delta$ and $\Pi_2$ a
proof of $\chi$ from $\Theta;\varphi$. Then
$\Pi_1^{\smallfrown}\Pi_2$ is a proof of $\chi$ from $\Delta;
\varphi$, as is easily seen. (mon). This follows from
Proposition~\ref{prop:deduce}. 
\proofend

Call a rule $\rho$ \textbf{admissible} 
%%%
\index{rule!admissible}%%
%%%
for a calculus $\CS$ if any
sequent $\Delta \bvdash \varphi$ that is derivable in $\CS + \rho$
is also derivable in $\CS$. Conversely, if $\rho$ is admissible
in $\CS$, we say that $\rho$ is \textbf{eliminable from} 
%%%
\index{rule!eliminable}%%
%%%
$\CS + \rho$. We shall show that (cut) is eliminable from $\CH$, so that
it can be omitted without losing derivable sequents. As
cut--elimination will play a big role in the sequel, the reader is
asked to watch the procedure carefully.
%%%
\begin{thm}[Cut Elimination]
\index{Cut Elimination}%%%
\label{thm:cutelimination}
\mbox{\rm (cut)} is eliminable from $\CH$.
\end{thm}
%%%
\proofbeg
Recall that (cut) is the following rule.
%%
\begin{equation}
\label{eq:dgcut}
\mbox{\rm (cut)}\quad
\begin{array}{c}
\Delta\bvdash\varphi \quad \Theta;\varphi \bvdash \chi \\\hline
\Delta;\Theta \bvdash\chi
\end{array}
\end{equation}
%%
\index{cut!degree of a \faul}%%
\index{cut!weight of a \faul}%%
\index{cut--weight}%%
%%%
Two measures are introduced. The \textbf{degree} of \eqref{eq:dgcut}
is
%%%
\begin{equation}
d := |\Delta| + |\Theta| + |\varphi| + |\chi|
\end{equation}
%%%
The \textbf{weight} of \eqref{eq:dgcut} is $2^d$.  The \textbf{cut--weight} 
of a proof tree $\BT$ is the sum over all weights of occurrences of cuts 
(= instances of (cut)) in it. Obviously,
the cut--weight of a proof tree is zero iff there are no
cuts in it. We shall now present a procedure that operates on proof
trees in such a way that it reduces the cut--weight of every given
tree if it is nonzero. This procedure is as follows. Let $\BT$ be 
given, and let $x$ be a node carrying the conclusion of an instance 
of (cut). We shall assume that above $x$ no instances of (cut) exist. 
(Obviously, $x$ exists if there are cuts in $\BT$.) $x$ has two mothers,
$y_1$ and $y_2$. Case (1). Suppose that $y_1$ is a leaf. Then
we have $\ell(y_1) = \varphi \bvdash \varphi$, $\ell(y_2) =
\Theta;\varphi \bvdash \chi$ and $\ell(x) = \Theta;\varphi\bvdash\chi$.
In this case, we may simply skip the application of cut by
dropping the nodes $x$ and $y_1$. This reduces the degree of the
cut by $2 \cdot |\varphi|$, since this application of (cut) has been
eliminated without trace. Case (2). Suppose that $y_2$ is a leaf.
Then $\ell(y_2) = \chi \bvdash \chi$, $\ell(y_1) = \Delta \bvdash
\varphi$, whence $\varphi  = \chi$ and $\ell(x) = \Delta \bvdash
\varphi = \ell(y_1)$. Eliminate $x$ and $y_2$. This reduces the
cut--weight by the weight of that cut. Case (3). Suppose that
$y_1$ has been obtained by application of (mon). Then the proof is
as shown on the left.
%%
\begin{equation}
\begin{array}{r@{\;\bvdash\;}rcc}
\Delta & \varphi & \quad & \\\cline{1-2}
\Delta;\Delta' & \varphi & \Theta;\varphi \bvdash \chi \\\hline
\multicolumn{3}{c}{\Delta;\Delta';\Theta \bvdash \chi}
\end{array}
\qquad
\qquad
\begin{array}{r@{\;\bvdash\;}l}
\multicolumn{2}{c}{\Delta \bvdash \varphi \qquad
    \Theta;\varphi \bvdash \chi } \\\hline
\Delta;\Theta & \chi \\\hline
\Delta;\Delta';\Theta & \chi
\end{array}
\end{equation}
%%%
We may assume that $\Delta' > 0$. We replace the local tree by the 
one on the right. The cut weight is reduced by 
%%%
\begin{equation}
2^{|\Delta| + |\Delta'| + |\Theta| + |\varphi| + |\chi|}  
- 2^{|\Delta| + |\Theta| + |\varphi| + |\chi|} > 0  
\end{equation}
%%%
Case (4). $\ell(y_2)$ has been obtained by application of (mon). This 
is similar to the previous case. Case (5). $\ell(y_1)$ has been obtained 
by ({\mtt\symbol{25}}\textbf{I}). Then the main formula is not the cut 
formula.
%%%
\begin{equation}
\begin{array}{ccc}
\Delta \bvdash \rho \quad \Delta;\tau \bvdash \varphi & \quad &
    \\\cline{1-1}
\Delta;\mbox{\mtt ($\rho$\symbol{25}$\tau$)} \bvdash \varphi & &
    \Theta;\varphi \bvdash \chi \\\hline
\multicolumn{3}{c}{\Delta;\Theta;\mbox{\mtt ($\rho$\symbol{25}$\tau$)}
    \bvdash \chi}
\end{array}
\end{equation}
%%
And the cut can be rearranged as follows.
%%
\begin{equation}
\begin{array}{r@{\;\bvdash\;}l@{\quad}ccc}
\Delta & \rho & & \Delta;\tau \bvdash \varphi &
    \Theta;\varphi \bvdash \chi \\\cline{1-2}\cline{4-5}
\Delta;\Theta & \rho & &
    \multicolumn{2}{c}{\Delta;\Theta;\tau \bvdash \chi}
\\\hline
\multicolumn{5}{c}{\Delta;\Theta;\mbox{\mtt ($\rho$\symbol{25}$\tau$)}
    \bvdash \chi}
\end{array}
\end{equation}
%%
Here, the degree of the cut is reduced by $|\mbox{\mtt ($\rho$\symbol{25}%
$\tau$)}| - |\tau| > 0$. Thus the cut--weight is reduced as well. 
Case (6). $\ell(y_2)$ has been obtained by ({\mtt\symbol{25}}\textbf{I}).
Assume $\varphi \neq \mbox{\mtt ($\rho$\symbol{25}$\tau$)}$.
%%%
\begin{equation}
\begin{array}{r@{\;\bvdash\;}l@{\quad}ccc}
\multicolumn{2}{c}{} & & \Theta;\varphi \bvdash \rho &
    \Theta;\varphi;\tau \bvdash \chi \\\cline{4-5}
\Delta & \varphi & &
    \multicolumn{2}{c}{\Theta;\varphi;\mbox{\mtt ($\rho$\symbol{25}$\tau$)} 
	\bvdash \chi}
\\\hline
\multicolumn{5}{c}{\Delta;\Theta;\mbox{\mtt ($\rho$\symbol{25}$\tau$)}
    \bvdash \chi}
\end{array}
\end{equation}
%%%
In this case we can replace the one cut by two as follows. 
%%%
\begin{equation}
\begin{array}{c}
\Delta\;\bvdash\; \varphi \qquad \Theta;\varphi\;\bvdash\; \rho 
\\\hline
\Delta;\Theta\;\bvdash\;\rho
\end{array}
\qquad
\begin{array}{c}
\Delta\;\bvdash\; \varphi \qquad \Theta;\varphi;\tau\;\bvdash\; \chi
\\\hline
\Delta;\Theta;\tau\;\bvdash\;\chi
\end{array}
\end{equation}
%%%
If we now apply ({\mtt\symbol{25}}\textbf{I}), we get the same sequent. 
The cut--weight has been diminished by
%%%%
\begin{equation}
2^{|\Delta| + |\Theta| + |\rho| + |\tau| + 3} - 
2^{|\Delta| + |\Theta| + |\rho|} - 
2^{|\Delta| + |\Theta| + |\tau|} > 0
\end{equation}
%%%
(See also below for the same argument.) Suppose however 
$\varphi = \mbox{\mtt ($\rho$\symbol{25}$\tau$)} \not\in \Theta$. 
Then either $\varphi$ is not the main formula of $\ell(y_1)$, 
in Case (1), (3), (5), or it actually is the main formula, and 
then we are in Case (7), to which we now turn.
%%%
Case (7). $\ell(y_1)$ has been introduced by (\textbf{I}{\mtt\symbol{25}}). 
If the cut formula is not the main formula, we are in cases 
(2), (4), (6) or (8), which we dealt with separately. Suppose 
however the main formula is the cut formula. Here, we cannot 
simply permute the cut unless $\ell(y_2)$ is the result of applying 
({\mtt\symbol{25}}\textbf{I}). In this case we proceed as follows. 
$\varphi =\mbox{\mtt ($\rho$\symbol{25}$\tau$)}$ for some $\rho$ 
and $\tau$. The local proof is as follows.
%%%
\begin{equation}
\begin{array}{r@{\;\bvdash\;}lccc}
\Delta;\rho & \tau & &
    \Theta \bvdash \rho & \Theta;\tau\bvdash\chi
\\\cline{1-2}\cline{4-5}
\Delta & \mbox{\mtt ($\rho$\symbol{25}$\tau$)}
    & &
\multicolumn{2}{c}{\Theta;\mbox{\mtt ($\rho$\symbol{25}$\tau$)}
    \bvdash \chi} \\\hline
\multicolumn{5}{c}{\Delta;\Theta \bvdash \chi}
\end{array}
\end{equation}
%%
This is rearranged in the following way.
%%
\begin{equation}
\begin{array}{ccr@{\;\bvdash\;}l}
\Delta;\rho \bvdash \tau
\qquad
\Theta \bvdash  \rho & & \Theta;\tau & \chi \\\cline{1-1}\cline{3-4}
\Delta;\Theta \bvdash \tau & & \Delta;\Theta;\tau  & \chi \\\hline
\multicolumn{4}{c}{
\Delta;\Theta \bvdash \chi}
\end{array}
\end{equation}
%%%
This operation eliminates the cut in favour of two cuts. The overall
degree of these cuts may be increased, but the weight has been
decreased. Let $d := |\Delta;\Theta|$, $p := |\mbox{\mtt ($\rho$%
\symbol{25}$\tau$)}|$. Then the first cut has weight $2^{d + p + |\chi|}$.
The two other cuts have weight
%%
\begin{equation}
2^{d + |\rho| + |\tau|} + 2^{d + |\tau| + |\chi|} \leq
2^{d + |\rho| + |\tau| + |\chi|} <
2^{d + p + |\chi|}
\end{equation}
%%
since $p > |\rho| + |\tau| > 0$. (Notice
that $2^{a+c} + 2^{a+d} = 2^a \cdot (2^c + 2^d) \leq 2^a \cdot
2^{c+d} = 2^{a+c+d}$ if $c, d > 0$.) 
Case (8). $\ell(y_2)$ has been obtained by (\textbf{I}{\mtt\symbol{25}}). 
Then $\chi = \mbox{\mtt ($\rho$\symbol{25}$\tau$)}$ for some $\rho$ 
and $\tau$. We replace the left hand proof part by the right hand part, 
and the degree is reduced by $|\mbox{\mtt ($\rho$\symbol{25}$\tau$)}| 
- |\tau| > 0$.
%%
\begin{equation}
\begin{array}{cr@{\;\bvdash\;}l}
    & \Theta;\varphi;\rho & \tau \\\cline{2-3}
\Delta \bvdash \varphi & \Theta;\varphi & 
	\mbox{\mtt ($\rho$\symbol{25}$\tau$)} \\\hline
\multicolumn{3}{c}{\Delta;\Theta \bvdash 
	\mbox{\mtt ($\rho$\symbol{25}$\tau$)}}
\end{array}
\qquad
\begin{array}{c}
\Delta \bvdash \varphi \qquad \Theta;\rho;\varphi \bvdash \tau \\\hline
\begin{array}{r@{\;\bvdash \;}l}
\Delta;\Theta;\rho & \tau \\\hline
\Delta;\Theta & \mbox{\mtt ($\rho$\symbol{25}$\tau$)}
\end{array}
\end{array}
\end{equation}
%%
So, in each case we managed to decrease the cut--weight.
This concludes the proof.
%%%
\proofend

Before we conclude this section we shall mention another deductive
%%%
\index{natural deduction}%%
%%%
calculus, called \textbf{Natural Deduction}. It uses proof trees, but
is based on the Deduction Theorem. First of all notice that we can
write Hilbert style proofs also in tree format. Then the leaves of
the proof tree are axioms, or assumptions, and the only rule we
%%%
\index{Modus Ponens (MP)}%%
%%%
are allowed to use is \textbf{Modus Ponens}.
%%
\begin{equation}
\mbox{\rm (MP)}\quad\begin{array}{c}
    \mbox{\mtt ($\varphi$\symbol{25}$\psi$)}\qquad
    \varphi \\\hline
    \psi
    \end{array}
\end{equation}
%%
This, however, is a mere reformulation of the previous calculus. The
idea behind natural deduction is that we view Modus Ponens as a rule
to {\it eliminate\/} the arrow, while we add another rule that allows
to introduce it. It is as follows.
%%
\begin{equation}
\mbox{\rm (\textbf{I}{\mtt\symbol{25}})}\quad
    \begin{array}{c}\psi \\\hline
    \mbox{\mtt ($\varphi$\symbol{25}$\psi$)}
    \end{array}
\end{equation}
%%
However, when this rule is used, the formula $\varphi$ may be
eliminated from the assumptions. Let us see how this goes. Let
$x$ be a node. Let us call the set $A(x) := \{\auf y, \ell(y)\zu
%%%
\index{assumption}%%
%%%
: y > x, y \mbox{ leaf}\}$ the set of \textbf{assumptions of} $x$.
If (\textbf{I}{\mtt\symbol{25}}) is used to introduce {\mtt ($\varphi$%
\symbol{25}$\psi$)}, any number of assumptions of $x$ that have the 
form $\auf y, \varphi\zu$ may be retracted. In order to know what
assumption has been effectively retracted, we check mark the
retracted assumptions by a superscript (e.~g.~$\varphi^{\surd}$).
%%%
\index{$\varphi^{\surd}$, $[\varphi]$}%%
%%%
Here are the standard rules for the other connectives. The fact
that the assumption $\varphi$ is or may be removed is annotated as
follows:
%%
\begin{equation}
\mbox{\rm (\textbf{I}{\mtt\symbol{25}})}\quad
    \begin{array}{c}[\varphi] \\
    \vdots \\
    \psi \\\hline
    \mbox{\mtt ($\varphi$\symbol{25}$\psi$)}
    \end{array}
\qquad
\mbox{\rm (\textbf{E}{\mtt\symbol{25}})}\quad
    \begin{array}{c} 
    \mbox{\mtt ($\varphi$\symbol{25}$\psi$)} \quad 
	\varphi \\\hline
	\psi 
    \end{array}
\end{equation}
%%
Here, $[\varphi]$ means that any number of assumptions of the form
$\varphi$ above the node carrying $\varphi$ may be check marked
when using the rule. (So, it does {\it not\/} mean that it
requires these formulae to be assumptions.) The rule 
(\textbf{E}{\mtt\symbol{25}}) is nothing but (MP). First, conjunction.
%%%
\begin{equation}
\mbox{\rm (\textbf{I}{\mtt\symbol{4}})}\quad\begin{array}{c}
\varphi \quad\psi \\\hline
\mbox{\mtt ($\varphi$\symbol{4}$\psi$)}
\end{array}
\qquad
\mbox{\rm (\textbf{E}$_1${\mtt\symbol{4}})}\quad
\begin{array}{c}
\mbox{\mtt ($\varphi$\symbol{4}$\psi$)} \\\hline
\varphi
\end{array}\qquad
\mbox{\rm (\textbf{E}$_2${\mtt\symbol{4}})}\quad
\begin{array}{c}
\mbox{\mtt ($\varphi$\symbol{4}$\psi$)} \\\hline
\psi
\end{array}
\end{equation}
%%%
The next is $\sbot$:
%%
\begin{equation}
\mbox{\rm (\textbf{E}$\sbot$)}\quad\begin{array}{c}
    \sbot \\\hline
    \varphi
    \end{array}
\end{equation}
%%%
For negation we need some administration of the check mark.
%%
\begin{equation}
\mbox{\rm (\textbf{I}{\mtt\symbol{5}})}\quad
\begin{array}{c}
[\varphi] \\
\vdots \\
\sbot \\\hline
\mbox{\mtt (\symbol{5}$\varphi$)}
\end{array}
\qquad
\mbox{\rm (\textbf{E}{\mtt\symbol{5}})}\quad
\begin{array}{c}
\varphi \quad\mbox{\mtt (\symbol{5}$\varphi$)} \\\hline
\sbot
\end{array}
\end{equation}
%%
So, using the rule (\textbf{I}{\mtt\symbol{5}}) any number of assumptions 
of the form $\varphi$ may be check marked.
%%
Disjunction is even more complex.
%%%
\begin{equation}
\begin{array}{l}
\mbox{\rm (\textbf{I}$_1${\mtt\symbol{31}})}\quad
\begin{array}{c}
\varphi \\\hline
\mbox{\mtt ($\varphi$\symbol{31}$\psi$)}
\end{array}\qquad
\mbox{\rm (\textbf{I}$_2${\mtt\symbol{31}})}\quad
\begin{array}{c}
\psi \\\hline
\mbox{\mtt ($\varphi$\symbol{31}$\psi$)}
\end{array} \\
\\
\mbox{\rm (\textbf{E}{\mtt\symbol{31}})}\quad
\begin{array}{ccc}
    & [\varphi] & [\psi] \\
    & \vdots & \vdots \\
\mbox{\mtt ($\varphi$\symbol{31}$\psi$)} & \chi  & \chi\\\hline
\multicolumn{3}{c}{\chi}
\end{array}
\end{array}
\end{equation}
%%%
In the last rule, we have three assumptions. As we have indicated,
whenever it is used, we may check mark any number of assumptions
of the form $\varphi$ in the second subtree and any number of
assumptions of the form $\psi$ in the third.

We shall give a characterization of natural deduction trees.
%%%
\index{rule}%%
\index{rule!finitary}%%%
%%%
A \textbf{finitary rule} is a pair $\rho = \auf \{\chi_i[A_i] : i < n\},
\varphi\zu$, where for $i < n$, $\chi_i$ is a formula, $A_i$ a finite
set of formulae and $\varphi$ a single formula. A \textbf{natural
deduction calculus} 
%%%
\index{natural deduction calculus}%%%
%%%%
$\GN$ is a set of finitary rules. A \textbf{proof tree for} 
%%%
\index{proof tree}%%
%%%
$\GN$ is a quadruple $\BT = \auf T, \succ, \ell, \CC\zu$ such that
$\auf T, \prec\zu$ is a tree, $\CC \subseteq T$
a set of leaves and $\BT$ is derived in the following way.
(Think of $\CC$ as the set of leaves carrying discharged
assumptions.)
%%%
\begin{dinglist}{43}
\item
$\BT = \auf \{x\}, \varnothing, \ell, \varnothing\zu$,
where $\ell \colon x \mapsto \varphi$.
\item
There is a rule $\auf \{\chi_i[A_i] : i < n\}, \gamma\zu$, and
$\BT$ is formed from trees $\BS_i$, $i < n$, with roots
$s_i$, by adding a new root node $r$, such that
$\ell_{\BS_i}(y_i) = \chi_i$, $i < n$, $\ell_{\BT}(x) =
\gamma$. Further, $\CC_{\BT} = \bigcup_{i < n} \CC_{\BS_i} \cup
\bigcup_{i < n} N_i$, where $N_i$ is a set of leaves
of $\BS_i$ such that for all $i < n$ and all $x \in N_i$:
$\ell_{\BS_i}(x) \in A_i$.
\end{dinglist}
%%
(Notice that the second case includes $n = 0$, in which case
$\BT = \auf \{x\}, \varnothing, \ell, \{x\}\zu$ where
$\ell(x)$ is simply an axiom.) We say that $\BT$ \textbf{proves}
$\ell(r)$ \textbf{in} $\GN$ from $\{\ell(x) : x \mbox{ leaf}, x
\not\in \CC\}$. Here now is a proof tree ending in (a0).
%%%
%%
\begin{equation}
\begin{array}{c}
\varphi^{\surd} \\\hline
\mbox{\mtt ($\psi$\symbol{25}$\varphi$)} \\\hline
\mbox{\mtt ($\varphi$\symbol{25}($\psi$\symbol{25}$\varphi$))}
\end{array}
\end{equation}
%%
Further, here is a proof tree ending in (a1).
%%
\begin{equation}
\begin{array}{ccc}
\mbox{\mtt ($\varphi$\symbol{25}($\psi$\symbol{25}$\chi$))}^{\surd}
    \qquad \varphi^{\surd} &
    & \mbox{\mtt ($\phi$\symbol{25}$\psi$)}^{\surd} \qquad \phi^{\surd}
    \\\cline{1-1}\cline{3-3}
\mbox{\mtt ($\psi$\symbol{25}$\chi$)} & & \psi \\\cline{1-3}
\multicolumn{3}{c}{\chi} \\\hline
\multicolumn{3}{c}{\mbox{\mtt ($\varphi$\symbol{25}$\chi$)}} \\\hline
\multicolumn{3}{c}{\mbox{\mtt (($\varphi$\symbol{25}$\psi$)\symbol{25}%
($\varphi$\symbol{25}$\chi$))}} \\\hline
\multicolumn{3}{c}{\mbox{\mtt (($\varphi$\symbol{25}($\psi$\symbol{25}%
$\chi$))\symbol{25}(($\varphi$\symbol{25}$\psi$)\symbol{25}(%
$\varphi$\symbol{25}$\chi$)))}}
\end{array}
\end{equation}
%%
A formula depends on all its assumptions that have not been
retracted in the following sense.
%%%
\begin{lem}
Let $\BT$ be a natural deduction tree with root $x$. Let $\Delta$
be the set of all formulae $\psi$ such that $\auf y, \psi\zu$ is an
unretracted assumption of $x$ and let $\varphi := \ell(x)$. Then
$\Delta \vdash^{\mathsf{H}} \varphi$.
\end{lem}
%%%
\proofbeg
By induction on the derivation of the proof tree.
\proofend

The converse also holds. If $\Delta \vdash^{\mathsf{H}} \varphi$
then there is a natural deduction proof for $\varphi$ with
$\Delta$ the set of unretracted assumptions (this is 
Exercise~\ref{ex:unretracted}).

{\it Notes on this section.} Proofs are graphs whose labels are
sequents. The procedure that eliminates cuts can be described
using a graph grammar. Unfortunately, the replacements also
manipulate the labels (that is, the sequents), so either one
uses infinitely many rules or one uses schematic rules.
%%%
\vplatz%%
\exercise%%
Show (a) $\mbox{\mtt ($\varphi$\symbol{25}($\psi$\symbol{25}$\chi$))} 
\vdash^{\mathsf{B}} \mbox{\mtt ($\psi$\symbol{25}($\varphi$%
\symbol{25}$\chi$))}$ and (b) $\mbox{\mtt ($\varphi$\symbol{4}$\psi$)}
\vdash^{\mathsf{H'}} \mbox{\mtt ($\psi$\symbol{4}$\varphi$)}$, 
where $\mathsf{H'}$ is \textsf{H} with the axioms for {\mtt\symbol{4}} 
added.
%%%
\vplatz 
\exercise 
Show that a set $\Sigma$ is inconsistent iff for every $\varphi$:  
$\Sigma \vdash^{\mathsf{B}} \varphi$.
%%%
\vplatz%%
\exercise%%
Show that a Hilbert style calculus satisfies DT for {\mtt\symbol{25}}
iff the formulae (a0) and (a1) are derivable in it.
(So, if we add, for example, the connectives {\mtt\symbol{5}}, 
{\mtt\symbol{4}} and {\mtt\symbol{31}} together with the corresponding 
axioms, DT remains valid.)
%%%
\vplatz%%
\exercise%%
Define $\varphi \approx \psi$ by $\varphi \vdash^{\mathsf{H}}
\psi$ and $\psi \vdash^{\mathsf{H}} \varphi$. Show that if
$\varphi \approx \psi$ then (a) for all $\Delta$ and $\chi$:
$\Delta; \varphi \vdash^{\mathsf{H}} \chi$ iff
$\Delta;\psi \vdash^{\mathsf{H}} \chi$, and (b) for all $\Delta$:
$\Delta \vdash^{\mathsf{H}} \varphi$ iff $\Delta
\vdash^{\mathsf{H}} \psi$.
%%%
\vplatz%%
\exercise%%
Let us call \textsf{Int} 
%%%
\index{\textsf{Int}}%%%
%%%%
the Hilbert calculus for {\mtt\symbol{25}}, 
$\sbot$, {\mtt\symbol{5}}, {\mtt\symbol{31}} and {\mtt\symbol{4}}. 
Further, call the Gentzen calculus for these connectives $\CI$. Show 
that $\Delta \vdash^{\mathsf{Int}} \varphi$ iff 
$\stackrel{\GI}{\rightsquigarrow} \Delta \bvdash
\varphi$.
%%
\vplatz%%
\exercise%%
\label{ex:unretracted}
Show the following claim: {\it If $\Delta \vdash^{\mathsf{H}}
\varphi$ then there is a natural deduction proof for $\varphi$
with $\Delta$ the set of unretracted assumptions}.
%%%
\vplatz
\exercise
Show that the rule of {\it Modus Tollens} is admissible in the natural
deduction calculus defined above (with added negation).
%%%
\begin{equation}
\mbox{\rm Modus Tollens:}\quad
\begin{array}{c}
\mbox{\mtt ($\varphi$\symbol{25}$\psi$)} \qquad 
	\mbox{\mtt (\symbol{5}$\psi$)} \\\hline
\mbox{\mtt (\symbol{5}$\varphi$)}
\end{array}
\end{equation}
