\section{Fundamentals of Linguistics}
\label{kap1-3}
%
%
%
In this section we shall say some words about our conception of
language and introduce some linguistic terminology. Since we
cannot define all the linguistic terms we are using, this section 
is more or less meant to get those readers acquainted with the 
basic linguistic terminology who wish to read the book without going 
through an introduction into linguistics proper. (However, it is 
recommended to have such a book at hand.)

A central tool in linguistics is that of postulating abstract
units and hierarchization. Language is thought to be more than a
mere relation between sounds and meanings. In between the two
realms we find a rather rich architecture that hardly exists in 
formal languages. This architecture is most clearly
articulated in \cite{harris:structural} and also
\cite{lamb:stratificationalism}. Even though linguists might
disagree with many details, this basic architecture is assumed even
in most current linguistic theories. We shall outline what we
think is minimal consensus.
%%%
\begin{figure}
\begin{center}
\begin{picture}(20,15)
\put(10,1){\makebox(0,0){Phonological Stratum}}
\put(10,1.6){\line(0,1){2.9}}
\put(10,5){\makebox(0,0){Morphological Stratum}}
\put(10,5.7){\line(0,1){2.8}}
\put(10,9){\makebox(0,0){Syntactical Stratum}}
\put(10,9.6){\line(0,1){2.9}}
\put(10,13){\makebox(0,0){Semantical Stratum}}
\end{picture}
\end{center}
\caption{The Strata of Language}
\label{fig:strata}
\end{figure}
%%%%
\index{stratum}%%
\index{stratum!semantical}%
\index{stratum!syntactical}%
\index{stratum!morphological}%
\index{stratum!phonological}%
\index{phonology}%%
\index{morphology}%
\index{syntax}%%
%%%
Language is organized in four levels or layers, which are also
called \textbf{strata}, see Figure~\ref{fig:strata}: the 
\textbf{phonological stratum}, the \textbf{morphological stratum}, 
the \textbf{syntactic stratum} and the \textbf{semantical stratum}. 
Each stratum possesses elementary units and 
rules of combination. The phonological stratum and the morphological 
stratum are adjacent, the morphological stratum and the syntactic 
stratum are adjacent, and the syntactic stratum and the semantic 
stratum are adjacent. Adjacent
%%%
\index{rule of realization}%%
%%%
strata are interconnected by so--called \textbf{rules of realization}.
On the phonological stratum we find the mere representation
of the utterance in its phonetic and phonological form. The
%%%
\index{phone}%%
%%%%
elementary units are the \textbf{phones}. An utterance is composed
from phones (more or less) by concatenation.
%%%
\index{phone}%%
%%%
The terms `phone', `syllable', `accent' and `tone' refer to this stratum. 
In the morphological stratum we find the elementary signs 
of the language (see Section~\ref{kap3}.\ref{kap3-1}),
%%%
\index{morph}%%
%%%
which are called \textbf{morphs}. These are defined to be the smallest
units that carry meaning, although the definition of `smallest'
may be difficult to give. They are different from words.
The word {\tt sees} is a word, but it is the combination
of two morphs, the root {\tt see} and the ending of the third
person singular present, {\tt s}. The units of the syntactical
stratum
%%
\index{lex}%%
\index{seme}%%
%%%
are called \textbf{lexes}, and they more or less are the same as words. 
The units of the semantical stratum are the \textbf{semes}.

On each stratum we distinguish concrete from abstract units. The concrete 
forms represent {\it substance}, while the abstract ones represent the 
{\it form} only. While the relationship between these two levels is 
far from easy, we will simplify the matter as follows. The abstract 
units are seen as sets of concrete ones. The abstraction is done
in such a way that the concrete member of each class that appears
in a construction is defined by its context, and that substitution
of another member results simply in a non well--formed unit (or
else in a virtually identical one). This definition is
deliberately vague; it is actually hard to make precise. The
interested reader is referred to the excellent
\cite{harris:structural} for a thorough discussion of the
structural method. We shall also return to this question in 
Section~\ref{kap5}.\ref{kap5-3}. The abstract counterpart of a phone is a
%%%
\index{phoneme}%%
%%%
\textbf{phoneme}. A phoneme is simply a set of phones. The sounds of
a single language are a subset of the entire space of human
sounds, partitioned into phonemes. This is to say that two
distinct phonemes of a languages are disjoint. We shall deal with
the relationship between phones and phonemes in 
Section~\ref{kap5}.\ref{kap5-3}. We use the following notation. We enclose
phonemes in slashes while square brackets are used to name phones.
So, if [p] denotes a phone then /p/ is a phoneme containing [p].
(Clearly, there are infinitely many sounds that may be called [p],
but we pick just one of them.) An index is used to make clear which
language the phoneme belongs to. For phonemes are strictly language 
bound. It makes little sense to compare phonemes across languages. 
Languages cut up the sound continuum in a different way. For example, 
let [p] and [p\textsuperscript{h}] be two distinct phones, 
where [p] is a phone corresponding to the letter {\tt p} in {\tt spit}, 
[p\textsuperscript{h}] a phone corresponding to the letter {\tt p} 
in {\tt put}. 
%%%%
\index{Hindi}%%
\index{English}%%
%%%
Hindi distinguishes these two phones as instantiations of different
phonemes: $/\mbox{\rm p}/_H \cap /\mbox{\rm p\textsuperscript{h}}/_H 
= \varnothing$.  English does not. So, $/\mbox{\rm p}/_E = 
/\mbox{\rm p\textsuperscript{h}}/_E$.
Moreover, the context determines whether what is written {\tt p}
is pronounced either as [p] or as [p\textsuperscript{h}]. Actually, 
in English there is no context in which both will occur. Finally, 
%%%
\index{French}%%
%%%
French does not even have the sound [p\textsuperscript{h}].  We give 
another example. The combination of the letters {\tt ch} is pronounced 
in two noticeably distinct ways in 
%%%
\index{German}%%%
%%%
German. After [\i], it sounds like 
[\c{c}], for example in {\tt Licht} [l{\i}\c{c}t], but after [a] it 
sounds like [x] as in {\tt Nacht} [naxt]; the choice between these 
two variants is conditioned solely by the preceding vowel. It is 
therefore assumed that German does not possess two phonemes but only 
one, written {\tt ch}, which is pronounced in these two ways depending 
on the context.

In the same way one assumes that German has only one plural 
\textbf{morpheme}
%%%
\index{morpheme}%%
%%%
even though there is a fair number of individual plural morphs.
Table~\ref{tab:gerplur} shows some possibilities of forming the 
plural in German.
%%
\begin{table}
\caption{German Plural Morphs}
\label{tab:gerplur}
\begin{center}
\begin{tabular}{|l|l||l|}
\hline
singular & plural & \\\hline\hline
{\tt Wagen}    & {\tt Wagen}  & `car' \\
{\tt Auto}     & {\tt Autos}  & `car' \\
{\tt Bus}      & {\tt Busse}  &  `bus' \\
{\tt Licht}    & {\tt Lichter}  &  `light' \\
{\tt Vater}    & {\tt V\"ater}  &  `father' \\
{\tt Nacht}    & {\tt N\"achte} &  `night' \\\hline
\end{tabular}
\end{center}
\end{table}
%%
The plural can be expressed either by no change, or by
adding an {\tt s}--suffix, an {\tt e}--suffix (the reduplication of
{\tt s} in {\tt Busse} is  a phonological effect and needs no
accounting for in the morphology), an {\tt er}--suffix, or
by umlaut or a combination of umlaut together with an {\tt e}--suffix.
%%%
\index{umlaut}%%
%%%
(\textbf{Umlaut} is another name for the following change of vowels:
{\tt a} becomes {\tt \"a}, {\tt o} becomes {\tt \"o}, and {\tt u} 
becomes {\tt \"u}. All other vowels remain the same. Umlaut is 
triggered by certain inflectional or derivational suffixes.)
All these are clearly different morphs. But they
%%%
\index{allomorph}%%
%%%
belong to the same morpheme. We therefore call them \textbf{allomorphs} 
of the plural morpheme. The differentiation into
strata allows to abstract away from irregularities.
Moving up one stratum, the different members of an abstraction
class are not distinguished. The different plural morphs for
example, are defined as sequences of phonemes, not of phones. To
decide which phone is to be inserted is the job of the phonological
stratum. Likewise, the word {\tt Lichter} is `known' to the
syntactical stratum only as a plural nominative noun. That it
consists of the root morph {\tt Licht} together with the morph
{\tt er} rather than any other plural morph is not visible in the
syntactic stratum. The difference between concrete and abstract
carries over in each stratum in the distinction
%%%
\index{stratum!surface}%%
\index{stratum!deep}%%
%%%
between a \textbf{surface} and a \textbf{deep} sub--stratum. The
morphotaxis has at deep level only the root {\tt Licht} and the
plural morpheme. At the surface, the latter gets realized as {\tt
er}. The step from deep to surface can be quite complex. For
example, the plural {\tt N\"achte} of {\tt Nacht} is formed by
changing the root vowel and adding the suffix {\tt e}. 
%%%
\index{umlaut}%%
%%% 
(Which of the vowels of the root are subject to umlauted must be 
determined by the phonological stratum. For example, the plural of 
{\tt Altar} `altar' is {\tt Alt\"are} and not {\tt \"Altare} or {\tt
\"Alt\"are}!) As we have already said, on the so--called deep
morphological (sub--)stratum we find only the combination of two
morphemes, the morpheme {\tt Nacht} and the plural morpheme. On
the syntactical stratum (deep or surface) nothing of that
decomposition is visible. We have one lex(eme), {\tt N\"achte}.
On the phonological stratum we find a sequence of 5 (!) phonemes,
which in writing correspond to {\tt n}, {\tt \"a}, {\tt ch}, {\tt
t} and {\tt e}. This is the deep phonological representation. On
the surface, we find the allophone [\c{c}] for the phoneme
(written as) {\tt ch}.

In Section~\ref{kap3}.\ref{kap3-1} we shall propose an approach to language
by means of signs. This approach distinguishes only 3 dimensions: a
sign has a {\it realization}, it has a {\it combinatorics\/} and 
it has a {\it meaning}. While the meaning is uniquely identifiable 
to belong to the semantic stratum, for the other two this is not
clear. The combinatorics may be seen as belonging to the
syntactical stratum. The realization of a sign, finally, could be
spelled out either as a sequence of phonemes, as a sequence of
morphemes or as a sequence of lexemes. Each of these choices is
legitimate and yields interesting insights. However, notice that
choosing sequences of morphemes or lexemes is somewhat incomplete
since it further requires an additional algorithm that realizes
these sequences in writing or speaking.

Language is not only spoken, it is also written. However, one must
distinguish between letters and sounds. The difference between
them is foremost a physical one. They use a different {\it
channel}.
%%
\index{channel}%%
%%
A \textbf{channel} is a physical medium in which the message is 
manifested. Language manifests itself first and foremost acoustically, 
even though a lot of communication is done in writing. We principally
learn a language by hearing and speaking it. Mastery of writing is
achieved only after we are fully fluent just speaking the language, 
even though our views of language are to a large extent shaped by 
our writing culture (see \cite{coulmas:writing} on that). (Sign 
languages form an exception that will not be dealt with here.) 
Each channel allows --- by its mere physical properties 
--- a different means of combination. A piece of paper is a two 
dimensional thing, and we are not forced to write down symbols 
linearly, as we are with acoustical signals. Think for example of 
the fact that Chinese characters are composite entities which contain 
parts in them. These are combined typically by juxtaposition, but 
characters are aligned vertically. Moreover, the graphical composition 
internally to a sign is of no relevance for the actual sound that goes 
with it. To take another example, Hindi is written in a syllabic
script, which
%%%
\index{Devanagari}%%
%%%
is called \textbf{Devanagari}. Each simple consonantal letter denotes
a consonant plus {\tt a}. Vowel letters may be added to these in
case the vowel is different from {\tt a}. (There are special
characters for word initial vowels.) Finally, to denote
consonantal clusters, the consonantal characters are melted into
each other in a particular way. There is only a finite number of
consonantal clusters and the way the consonants are melted is
fixed. The individual consonants are usually recognizable from the
graphical complex. In typesetting there is a similar phenomenon 
%%%%
\index{ligature}%%
%%%%
known as \textbf{ligature}. The graphemes {\tt f} and {\tt i} melt 
into one when the first is before the second: `fi'. (Typewriters 
have no ligature, for obvious reasons. So you get {\tt fi}.) Also, 
in mathematics the possibilities of the graphical channel are widely 
used. We use indices, superscripts, subscripts, underlining, arrows 
and so on. Many diagrams are therefore not so easy to linearize. 
(For example, $\widehat{x}$ is spelled out as {\tt x hat}, $\overline{x}$ 
as {\tt x bar}.) Sign languages also make use of the three--dimensional 
space, which proves to require different perceptual skills than 
spoken language. 

While the acoustic manifestation of language is in some sense
essential for human language, its written manifestation is
typically secondary, not only for the individual human being, as
said above, but also from a cultural historic point of view. The
sounds of the language and the pronunciation of words is something
that comes into existence naturally, and they can hardly be fixed
or determined arbitrarily. Attempts to stop language from changing
are simply doomed to failure. Writing systems, on the other hand,
are cultural products, and subject to sometimes severe regimentation. 
The effect is that writing systems show much greater variety across 
languages than sound systems. The number of primitive letters varies 
between some two dozen and a few thousand. This is so since some 
languages have letters for sounds (more or less) like Finnish 
%%%%
\index{Finnish}%%
%%%
(English is a difficult case), others have letters for syllables 
%%%
\index{Hindi}%%
%%%%
(Hindi, written in Devanagari) and yet others have letters for 
words (Chinese). 
%%%%
\index{Chinese}%%
%%%
It may be objected that in Chinese a character always stands for a 
syllable, but words may consist of several syllables, hence 
of several characters. Nevertheless, the difference with 
Devanagari is clear. The latter shows you how the word sounds like, 
the former does not, unless you know character by character how it 
is pronounced. If you were to introduce a new syllable into Chinese 
you would have to create a new character, but not so in Devanagari. But 
all this has to be taken with care. Although French 
%%%
\index{French}%%
%%%
uses the Latin 
alphabet it becomes quite similar to Chinese. You may still know how 
to pronounce a word that you see written down, but from hearing it 
you are left in the dark as to how to spell it. For example, the 
following words are pronounced completely alike: {\tt au}, {\tt haut}, 
{\tt eau}, {\tt eaux}; similarly {\tt vers}, {\tt vert}, {\tt verre}, 
{\tt verres}.

In what is to follow, language will be written language. This is
the current practice in such books as this one; but it requires 
comment. We are using the so--called Latin alphabet. It is used in 
almost all European countries, while each country typically uses a 
different set of symbols. The difference is slight, but needs 
accounting for (for example, when you wish to produce keyboards 
or design fonts). Finnish, Hungarian and German,
%%%%
\index{Finnish}%%%
\index{Hungarian}%%%
\index{German}%%
%%%%
for example, use {\tt \"a}, {\tt \"o} and {\tt \"u}. The letter
{\tt {\ss}} is used in the German alphabet (but not in
Switzerland). In French, 
%%%
\index{French}%%
%%%
one uses {\tt \c{c}}, also accents, and
so on. The resource of single characters,
%%%
\index{letter}%%
%%%
which we call \textbf{letters}, is for the European languages somewhere 
between 60 and 100. Besides each letter, both in upper and lower case,
we also have the punctuation marks and some extra symbols, not to 
forget the ubiquitous blank. Notice, however, that not all languages 
have a blank (Chinese is a case in point, and also the Romans did not 
use any blanks). 
On the other hand, one blank is not distinct from two. We can either 
decide to disallow two blanks in a row, or postulate that they are 
equal to one. (So, the structure we look at is 
$\GZ(A)/\{\Box \boldsymbol{\doteq} \Box\conc\Box\}$.) A final problem 
area to be considered is our requirement that sign composition is 
additive. This means that every change that occurs is underlyingly viewed 
as adding something that was not there. This can yield awkward 
%%%
\index{German}\index{umlaut}%%
%%%
results. While the fact that German umlaut is graphically 
speaking just the addition of two dots ({\tt a} becomes {\tt \"a}, 
{\tt o} becomes {\tt \"o}, {\tt u} becomes {\tt \"u}), the change 
of a lower case letter to an upper case letter cannot be so 
analysed. This requires another level of representation, one 
at which the process is completely additive. This is harmless, 
if we only change the material aspect (substance) rather than the 
form. 

The counterpart of a letter in the spoken languages is the
phoneme.
%%%
\index{phoneme}%%
%%%
Every language utterance can be analyzed into a sequence of
phonemes (plus some residue about which we will speak briefly
below). There is generally no biunique correspondence between
phonemes and letters. The connection between the visible and the
audible shape of language is everything but predictable or unambiguous 
in either direction. English is a perfect example. There is hardly any 
letter that can unequivocally be related to a phoneme. For example, 
the letter {\tt g} represents in many cases the phoneme [g] unless it 
is followed by {\tt h}, in which case the two typically together
represent a sound that can be zero (as in {\tt sought}
[s{\textopeno}:t]), or {\tt f} (as in {\tt laughter} ([la:ft\textschwa]).
To add to the confusion, the letters represent different sets of 
phones in different languages. (Note that it makes no sense to 
speak of the same {\it phoneme\/} in two different languages, 
as phonemes are abstractions that are formed within a single 
language.) The letter {\tt u} has many different manifestations 
in English, German and French 
%%%
\index{English}\index{German}\index{French}%%%
%%%%
that are hardly compatible. This has prompted the invention
of an international standard, the so--called \textbf{International
Phonetic Alphabet} (\textbf{IPA}, see \cite{ipahandbook}). Ideally,
every sound of a given language can be uniquely transcribed into
IPA such that anyone who is not acquainted with the language can
reproduce the utterances correctly. The transcription of a word
into this alphabet therefore changes whenever its sound
manifestation changes, irrespective of the spelling norm.
Unfortunately, the transcription must ultimately remain 
inconsequential, because even in the IPA letters stand for 
sets of phones, but in every language the width of a phoneme 
(= the set of phones it contains) is different. For example, 
%%%
\index{English}\index{Hindi}%%
%%%%
if (English) $/\mbox{\rm p}/_E$ contains both (Hindi) 
$/\mbox{\rm p}/_H$ and  $/\mbox{\rm p\textsuperscript{h}}/_H$, 
we either have to represent {\tt p} in English by (at least) two 
letters or else give up the exact correspondence. 

The carriers of meaning are however not the sounds or letters
(there is simply not enough of them); it is certain sequences
thereof. Sequences of letters that are not separated by a
blank or a punctuation mark other than `{\tt -}' are called
%%%
\index{word}%%
%%%
\textbf{words}. Words are units which can be analyzed further, for
example into letters, but for the most part we shall treat them
as units. This is the reason why the alphabet $A$ in the technical
sense will often {\it not\/} be the alphabet in the sense of
`stock of letters' but in the sense of `stock of words'. However,
since most languages have infinitely many words (due to
compounding), and since the alphabet $A$ must be finite, some care
must be exercised in choosing the alphabet. Typically, it will 
exclude the compound words, but it will have to include all idioms. 

We have analyzed words into sequences of letters or sounds, and
sentences into sequences of words. This implies that sentences and
words can always be so analyzed. This is what we shall assume
throughout this book. The individual occurrences of
%%%
\index{segment}%%
%%%
sounds (letters) are called \textbf{segments}. For example, the 
(occurrences of the) letters {\tt n}, {\tt o}, and {\tt t} are the 
segments of {\tt not}. The fact that words can be segmented is called
%%%
\index{segmentability}%%
%%%
\textbf{segmentability property}. At closer look it turns out that
segmentability is an idealization. For example, a question differs
from an assertion in its \textbf{intonation contour}, which is 
the rise and fall of the pitch during the utterance. The contour 
shows distribution over the whole sentence but follows
specific rules. It is of course different in different languages.
(Falling pitch at the end of a sentence, for example, may
accompany questions in English, but not in German.) Because of its
nature, intonation contour is called a
%%%
\index{feature!suprasegemental}%%
%%%
\textbf{suprasegmental feature}. There are more, for example emphasis.
Segmentability differs also with the channel.  In writing, a question
is marked by a segmental feature (the question mark), but emphasis is
not. Emphasis is typically marked by underlining or italics. For
example, if we want to emphasize the word `board', we write
$\uli{\mbox{\tt board}}$ or {\it board}. As can be seen, every
letter is underlined or set in italics, but underlining or italics is 
usually not something that is meant to emphasize those letters
that are marked by it; rather, it marks emphasis of the entire
word that is composed from them. We could have used a segmental
symbol, just like quotes, but the fact of the matter is that we
do not. Disregarding this, language typically is segmentable.

However, even if this is true, the idea that the morphemes of the 
language are sequences of letters is largely mistaken. To give an 
extreme example, the plural is formed in Bahasa Indonesia
%%%
\index{Bahasa Indonesia}%%
%%%
by reduplicating the noun. For example, the word {\tt anak} means
`child', the word {\tt anak-anak} therefore means `children', the
word {\tt orang} means `man', and {\tt orang-orang} means `men'.
Clearly, there is no sequence of letters or phonemes that can be
literally said to constitute a plural morph. Rather, it is the
function $f \colon A^{\ast} \pf A^{\ast} \colon \vec{x} \mapsto
\vec{x}\mbox{\tt -}\vec{x}$, sending each string to its duplicate
(with an interspersed hyphen). Actually, in writing the
abbreviation {\tt anak2} and {\tt orang2} is commonplace. Here,
{\tt 2} is a segmentable marker of plurality. However, notice that
the words in the singular or the plural are each fully
segmentable. Only the marker of plurality cannot be identified
with any of the segments. This is to some degree also the case in
German, where the rules are however much more complex, as we have
seen above. The fact that morphs are (at closer look) not simply
strings will be of central concern in this book.

Finally, we have to remark that letters and phonemes are not
unstructured either. Phonemes consist of various so--called
%%%
\index{feature!distinctive}%%
%%%
\textbf{distinctive features}. These are features that distinguish
the pho\-ne\-mes from each other. For example, [p] is distinct from
[b] in that it is voiceless, while [b] is voiced. Other voiceless
consonants are [k], [t], while [g] and [d] are once again voiced.
Such features can be relevant for the description of a language.
There is a rule of German (and other languages, for example
Russian) that forbids voiced consonants to occur at the end of a
syllable. For example, the word {\tt Jagd} `hunting' is pronounced
[\textprimstress ja:kt], not [\textprimstress ja:gd]. This is so 
since [g] and [d] may not occur at the end of the syllable, since 
they are voiced. Now, first of all, why do we not write {\tt Jakt} 
then? This is so since inflection and derivation show that when these 
consonants occur non--finally in the syllable they are voiced: we 
have {\tt Jagden} [\textprimstress ya:kden] `huntings', with [d] 
now in fact being voiced, and also {\tt jagen} 
[\textprimstress ya:g\textschwa n] `to hunt'. Second: why
do we not propose that voiceless consonants become voiced when
syllable initial? Because there is plenty of evidence that this
does not happen. Both voiced and voiceless sounds may appear at
the beginning of the syllable, and those ones that are analyzed as
underlyingly voiceless remain so in whatever position. Third: why
bother writing the underlying consonant rather than the one we
hear? Well, first of all, since we know how to pronounce the word
anyway, it does not matter whether we write [d] or [t]. On the
other hand, if we know how to write the word, we also know a
little bit about its morphological behaviour. What this comes down
to is that to learn how to write a language is to learn how the
language works. Now, once this is granted, we shall explain why we
find [k] in place of [g] and [t] in place of [d]. This is because
of the internal organisation of the phoneme. The phoneme is a set
of distinctive features, one of which (in German) is $[\pm
\mbox{\rm voiced}]$. The rule is that when the voiced consonant
may not occur, it is only the feature $[+ \mbox{\rm voiced}]$ that
is replaced by $[- \mbox{\rm voiced}]$. Everything else remains
the same. A similar situation is the relationship between upper
and lower case letters. The rule says that a sentence may not
begin with a lower case letter. So, when the sentence begins, the
first letter is changed to its upper case counterpart if
necessary. Hence, letters too contain distinctive features. Once
again, in a dictionary a word always appears as if it would normally 
appear elsewhere. Notice by the way that although each letter is by 
itself an upper or a lower case letter, written
language attributes the distinction upper versus lower case to the
word not to the initial letter.  Disregarding some modern
spellings in advertisements (like in Germany {\tt InterRegio},
{\tt eBusiness} and so on) this is a reasonable strategy. However,
it is nevertheless not illegitimate to call it a suprasegmental
feature.

In the previous section we have talked extensively about representations 
of terms by means of strings. In linguistics this is an important 
issue, which is typically discussed in conjunction 
with {\it word order}. Let us give an example.  Disregarding
word classes, each word of the language has one (or several)
arities. The finite verb \textsf{see} has arity 2. The proper
names \textsf{Paul} and \textsf{Marcus} on the other hand have arity
0. Any symbol of arity $> 0$ is called a \textbf{functor} with
respect to its argument.
%%%
\index{functor}%%
\index{argument}%%
%%%
In syntax one also speaks of
%%%
\index{head}%%
\index{complement}%%
%%%
\textbf{head} and \textbf{complement}. These are relative notions. In
the term $\textsf{see}(\textsf{Marcus},\textsf{Paul})$,
the functor or head is \textsf{see}, and its arguments are
\textsf{Paul} and \textsf{Marcus}. To distinguish these arguments from
each other, we use the terms {\it subject\/} and {\it object}.
%%%
\index{subject}%%
\index{object}%%
%%%
\textsf{Marcus} is the \textbf{subject} and \textsf{Paul} is the 
\textbf{object} of the sentence. The notions `subject' and `object' 
denote so--called \textbf{grammatical relations}.
%%%
\index{grammatical relation}%%
%%%
The correlation between argument places and grammatical relations
is to a large extent arbitrary, and is of central concern in
syntactical theory. Notice also that not all arguments are
complements. Here, syntactical theories diverge as to which of
the arguments may be called `complement'. In generative grammar,
for example, it is assumed that only the direct object is a
complement.

Now, how is a particular term represented?
The representation of \textsf{see} is {\tt sees}, that of
\textsf{Marcus} is {\tt Marcus} and that of \textsf{Paul} is {\tt Paul}.
The whole term \eqref{ex:1} is represented by the string
\eqref{ex:2}.
%%
\begin{align}
\label{ex:1}
& \textsf{see}(\textsf{Marcus},\textsf{Paul}) \\
\label{ex:2}
& \mbox{\tt Marcus sees Paul.}
\end{align}
%%
So, the verb appears after the subject, which in turn precedes the
object. At the end, a period is placed. However, to spell out the
relationship between a language and a formal representation is not
as easy as it appears at first sight. For first of all, the term 
should be something that does not depend on the particular
language we choose and which gives us the full meaning of the term
(so it is like a language of thought or an interlingua, if you
wish). So the above term shall mean that Marcus sees Paul. We
could translate the English sentence \eqref{ex:2} by choosing a
different representation language, but the choice between
languages of representation should actually be immaterial as long
as they serve the purpose. This is a very rudimentary picture but
it works well for our purposes. We shall return to the idea of
producing sentences from terms in Chapter~\ref{kap3}. Now look
first at the representatives of the basic symbols in some other
languages.
%%
%%%
\index{Latin}%%%
%%%
\begin{equation}
\mbox{\begin{tabular}{|l||l|l|l|}
\hline
           & \textsf{see} & \textsf{Marcus} & \textsf{Paul} \\\hline
German     & {\tt sieht} & {\tt Marcus} & {\tt Paul} \\
Latin      & {\tt vidit} & {\tt Marcus} & {\tt Paulus} \\
Hungarian  & {\tt l\'{a}tja} & {\tt Marcus} & {\tt P\'al} \\\hline
\end{tabular}}
\end{equation}
%%%
Here is how \eqref{ex:1} is phrased in these languages.
%%
\begin{align}
\label{ex:1349}
& \mbox{\tt Marcus sieht Paul.} \\
\label{ex:1350}
& \mbox{\tt Marcus Paulum vidit.} \\
\label{ex:1351}
& \mbox{\tt Marcus l\'atja P\'alt.}
\end{align}
%%
English is called an \textbf{SVO--language}, since in transitive
%%%
\index{language!SOV--, SVO--, VSO--}%%
\index{language!OSV--, OVS--, VOS--}%%
%%%
constructions the subject precedes the verb, and the verb in turn
the object. This is exactly the infix notation. (However, notice
that languages do not make use of brackets.) One uses the mnemonic
symbols `S', `V' and `O' to define the following basic 6 types of
languages: SOV, SVO, VSO, OSV, OVS, VOS. These names tell us how
the subject, verb and object follow each other in a basic
transitive sentence. We call a language of type VSO or VOS 
%%%
\index{language!verb final}%
\index{language!verb medial}%
\index{language!verb initial}%
%%%
\textbf{verb initial}, a language of type SOV or OSV \textbf{verb
final} and a language of type SVO or OVS \textbf{verb medial}. By
this definition, German 
%%%%
\index{German}%%
\index{Hungarian}%%
%%%
is SVO, Hungarian too, hence both are verb
medial and Latin is SOV, hence verb final. These types are not
equally distributed. Depending on the method of counting, 
40 -- 50~\% of the world's
languages are SOV languages, up to 40~\% SVO languages and another
10~\% are VSO languages. This means that in the vast majority of
languages the order of the two arguments is: subject before
object. This is why one does not generally emphasize the relative
order of the subject with respect to the object. There is a bias
against placing the verb initially (VSO), and a slight bias to put
it finally (SOV) rather than medially (SVO).

One speaks of a \textbf{head final} 
%%%
\index{language!head final}%%
\index{language!head initial}%%
%%%
(\textbf{head initial}) language if
a head is consistently put at the end behind all of its arguments
(at the beginning, before all the arguments). One denotes the type
of order by XH (HX), X being the complement, H the head.
There is no notion of a {\it head medial\/} language for the reason
that most heads only have one complement. It is often understood 
that the direct object is the only complement of the verb.
Hence, the word orders SVO and VOS are head initial, OVS and SOV head
final. (The orders VSO and OSV are problematic since the verb is
not adjacent to its object.) A verb is a head, however a very
important one, since it basically builds the clause. Nevertheless,
different heads may place their arguments differently, so a
language that is verb initial need not be head initial, a language
that is verb final need not be head final. Indeed, there are few
languages that are consistently head initial (medial, final).
%%%
\index{Japanese}%%%
%%%
Japanese is rather consistently head final. Even a
relative clause precedes the noun it modifies. Hungarian 
%%%
\index{Hungarian}%%
%%%
is a mixed case: adjectives precede nouns, there are no prepositions,
only postpositions, but the verb tends to precede its object.

For the interested reader we give some more information on the
languages shown above. First, Latin 
%%%
\index{Latin}%%
%%%
was initially an SOV language,
however word order was not really fixed (see
\cite{lehmann:bases} and \cite{bauer:latin}). In fact, any of the six
permutations of the sentence \eqref{ex:1350} is grammatical.
Hungarian is more complex, again the word order shown in
\eqref{ex:1351} is the least marked, but the rule is that
discourse functions determine word order. (Presumably this is true 
for Latin as well.) German 
%%%
\index{German}%%
%%%
is another special case. Against all appearances there is all reason to
believe that it is actually an SOV language. You can see this by
noting first that only the carrier of inflection appears in second
place, for example only the auxiliary if present. Second, in a
subordinate clause all parts of the verb including the carrier of
inflection are at the end.
%%
\begin{align}
%\begin{tabular}{ll}
%\begin{split}
& \mbox{\tt Marcus sieht Paul.} \\\notag
& \mbox{\it Marcus sees Paul.} \\
%\end{split}
%\begin{split}
& \mbox{\tt Marcus will Paul sehen.} \\\notag
& \mbox{\it Marcus wants to see Paul.} \\
%\end{split}
%\begin{split}
& \mbox{\tt Marcus will Paul sehen k\"onnen.} \\\notag
& \mbox{\it Marcus wants to be able to see Paul.} \\
%\end{split}
%\begin{split}
& \mbox{..., {\tt weil Marcus Paul sieht.}} \\\notag
& \mbox{\it ..., because Marcus sees Paul.} \\
%\end{split}
%\begin{split}
& \mbox{..., {\tt weil Marcus Paul sehen will.}} \\\notag
& \mbox{\it ..., because Marcus wants to see Paul.} \\
%\end{split}
%\begin{split}
& \mbox{..., {\tt weil Marcus Paul sehen k\"onnen will.}} \\\notag
& \mbox{\it ..., because Marcus wants to be able to see Paul.}
%\end{split}
%\end{tabular}
\end{align}
%%
So, the main sentence is not always a good indicator of the
word order. Some languages allow for alternative word orders,
like Latin and Hungarian. This is not to say that all variants
have the same meaning or significance; it is only that they
are equal as representatives of \eqref{ex:1}. We therefore
speak of Latin as having \textbf{free word order}.
%%%
\index{word order!free}%%
%%%
However, this only means that the head and the argument can
assume any order with respect to each other, not that simply
all permutations of the words mean the same.

Now, notice that subject and object are coded by means
%%%
\index{case}%%%
%%%
of so--called \textbf{cases}. In Latin, the object carries accusative
case, so we find {\tt Paulum} instead of {\tt Paulus}. Likewise,
in Hungarian we have {\tt P\'alt} in place of {\tt P\'al},
the nominative. So, the way a representing string is arrived at
is rather complex.  We shall return again to case marking in
Chapter~\ref{kap4}.

Natural languages also display so--called
%%%
\index{polyvalency}%%
%%%
\textbf{polyvalency}. We say that a word is polyvalent if it can have
several arities (even with the same meaning). The verb {\tt to roll}
%%%
\index{verb!transitive}%%
\index{verb!intransitive}%%
%%%
can be unary (= \textbf{intransitive}) as well as binary 
(= \textbf{transitive} if the second argument is accusative, 
\textbf{intransitive} otherwise).
This is not allowed in our definition of signature. However,
it can easily be modified to account for polyvalent symbols.

{\it Notes on this section.} The rule that spells out 
the letters \mbox{\tt ch} in German is more complex than the above 
explications show. For example, it is [x] in {\tt fauchen} 
but [\c{c}] in {\tt Frauchen}. This may have two reasons: 
(a) There is a morpheme boundary between {\tt u} and {\tt ch} 
in the second word but not in the first. This morpheme boundary 
induces the difference. (b) The morpheme {\tt chen} is special 
in that \mbox{\tt ch} will always be realized as [\c{c}]. The 
difference between (a) and (b) is that while (a) defines a 
realization rule that uses only the phonological 
representation, (b) uses morphological information to define 
the realization. Mel'\v{c}uk 
%%%
\index{Mel'\v{c}uk, Igor}%%%
%%%
defines the realization rules 
as follows. In each stratum, there are rules that define how 
deep representations get mapped to surface representations. 
Across strata, going down, the surface representations of the 
higher stratum get mapped into abstract representations of 
the lower stratum. (For example, a sequence of morphemes is 
first realized as a sequence of morphs and then spelled out 
as a sequence of phonemes, until, finally, it gets mapped 
onto a sequence of phones.) Of course, one may also reverse 
the process. However, adjacency between (sub-)strata remains 
as defined. 
%%%
\vplatz
\exercise
%%%
\index{Polish Notation}%%%
%%%
Show that in Polish Notation, unique readability is lost when
there exist polyvalent function symbols.
%%%
\vplatz
\exercise
Show that if you have brackets, unique readability is guaranteed
even if you have polyvalency.
%%%
\vplatz
\exercise
We have argued that German is a verb final language. But is it
strictly head final? Examine the data given in this section
as well as the data given below.
%%
\begin{align}
 & \mbox{\tt Josef pfl\"uckt eine sch\"one Rose f\"ur Maria.} \\\notag
 & \mbox{Josef is.picking a beautiful rose for Mary} \\
 & \mbox{\tt Heinrich ist dicker als Josef.} \\\notag
 & \mbox{Heinrich is fatter than Josef}
\end{align}
%%
%Knowing that long ago the ancestor of German (and English) was
%consistently head final, can you give an explanation for the
%syntactic facts shown in this section?
%%%
\vplatz
\exercise
Even if languages do not have brackets, there are elements that
indicate clearly the left or right periphery of a constituent.
Such elements are the determiners {\tt the} and {\tt a}({\tt n}).
Can you name more? Are there elements in English indicating the
right periphery of a constituent? How about demonstratives like
{\tt this} or {\tt that}?
%%%
\vplatz %%
\exercise %%
By the definitions, Unix is head initial. For example, the
command {\tt lpr} precedes its arguments. Now study the way
in which optional arguments are encoded. (If you are sitting
behind a computer on which Unix (or Linux) is running, 
type {\tt man lpr} and you get a synopsis of the command and its 
syntax.) Does the syntax guarantee unique readability? (For the more 
linguistic minded reader: which type of marking strategy does 
Unix employ?  Which natural language you know of corresponds 
best to it?)
