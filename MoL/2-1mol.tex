\chapter{Context Free Languages}
\thispagestyle{empty}
\label{kap2}
%
%
%
\section{Regular Languages}
\label{zweieins}
\label{kap2-1}
%
%
%
\index{language!regular}%%%
%%%%
Type 3 or regular grammars are the most simple grammars in the
Chomsky Hierarchy.  There are several characterizations of regular
languages: by means of finite state automata, by means of 
equations over strings, and by means of so--called regular expressions.
Before we begin, we shall develop a simple form for regular
grammars. First, all rules of the form $X \pf Y$ can be eliminated.
To this end, the new set of rules will be
%%
\begin{equation}
\begin{split}
R^{\heartsuit} := & \phantom{\mbox{}\cup\mbox{}} \{X \pf a Y :
X \vdash_G a Y\} \\ 
& \cup \{X \pf \vec{x} : X \vdash_G \vec{x}, \vec{x} \in A_{\varepsilon}\}
\end{split}
\end{equation}
%%
It is easy to show that the grammar with $R^{\heartsuit}$ in place
of $R$ generates the same strings. We shall introduce another
simplification.  For each $a \in A$ we introduce a new nonterminal
$U_a$. In place of the rules $X \pf a$ we now add the rules $X \pf a U_a$
as well as $U_a \pf \varepsilon$. Now every rule with the
exception of $U_a \pf \varepsilon$ is strictly expanding.
This grammar is therefore not regular if $\varepsilon \in 
L(G)$ but it generates the same language. However, the last kind 
of rules can be used only once, at the end of the derivation.
For the derivable strings all have the form $\vec{x} \conc Y$
with $\vec{x} \in A^{\ast}$ and $Y \in N$. If one applies a
rule $Y \pf \varepsilon$ then the nonterminal disappears
and the derivation is terminated. We call a regular grammar
\textbf{strictly binary}
%%%
\index{grammar!strictly binary}%%
%%%
if there are only rules of the form $X \pf a Y$ or $X \pf \varepsilon$.
%%
\begin{defn}
%%%
\index{automaton!finite state}%%
\index{automaton!deterministic finite state}%%
\index{transition function}%%
\index{state!initial}%%
\index{state!accepting}%%
%%%
Let $A$ be an alphabet. A (\textbf{partial}) \textbf{finite state
automaton} is a quintuple $\GA = \auf A, Q, i_0, F, \delta\zu$
such that $Q$ is a finite set, $i_0 \in Q$, $F \subseteq Q$ and 
$\delta \colon Q \times A \pf \wp(Q)$. $Q$ is the set of \textbf{states}, 
$i_0$ is called \textbf{the initial state}, $F$ the set of 
\textbf{accepting states} and $\delta$ the \textbf{transition function}. 
$\GA$ is called \textbf{deterministic} if $\delta(q,a)$ contains exactly 
one element for each $q \in Q$ and $a \in A$.
\end{defn}
%%
$\delta$ can be extended to sets of states and strings in the
following way ($S \subseteq Q$, $a \in A$).
%%
\begin{subequations}
\label{eq:sextend}
\begin{align}
\delta(S,\varepsilon) & := S \\
\delta(S,a) & := \bigcup \auf \delta(q,a) : q \in S\zu \\
\delta(S, \vec{x} \conc a) & := \delta(\delta(S, \vec{x}), a)
\end{align}
\end{subequations}
%%
With this defined, we can now define the accepted language.
%%
\index{$L(\GA)$}%%%
%%%
\begin{equation}
L(\GA) = \{\vec{x} : \delta(\{i_0\}, \vec{x}) \cap F \neq
\varnothing\}
\end{equation}
%%
$\GA$ is strictly partial if there is a state $q$
and some $a \in A$ such that $\delta(q,a) = \varnothing$.
An automaton can always be transformed into an equivalent
automaton which is not partial. Just add another state
$q_{\varnothing}$ and add to the transition function the
following transitions.
%%
\begin{equation}
\delta^+(q,a) := 
\begin{cases}
\delta(q,a) & \text{if $\delta(q,a) \neq \varnothing$ and 
	$q \neq q_{\varnothing}$,} \\
q_{\varnothing} & \text{if $\delta(q,a) = \varnothing$ or 
$q = q_{\varnothing}$.}
\end{cases}
\end{equation}
%%
Furthermore, $q_{\varnothing}$ shall {\it not\/} be an
accepting state. In the case of a deterministic automaton
we have $\delta(q, \vec{x}) = \{q'\}$ for some $q'$. In this
case we think of the transition function as yielding
states from states plus strings, that is, we now have
$\delta(q, \vec{x}) = q'$. Then the definition of the
language of an automaton $\GA$ can be refined as follows.
%%
\begin{equation}
L(\GA) = \{\vec{x} : \delta(i_0, \vec{x}) \in F\}
\end{equation}
%%
For every given automaton there is a deterministic automaton 
that accepts the same language. Put
%%
\index{$\GA^d$}%%
%%%
\begin{equation}
\GA^d := \auf A, \wp(Q), \{i_0\}, F^d, \delta\zu 
\end{equation}
%%
where $F^d := \{G \subseteq Q : G \cap F \neq \varnothing\}$
and $\delta$ is the transition function of $\GA$ extended to
sets of states.
%%
\begin{prop}
$\GA^d$ is deterministic and $L(\GA^d) = L(\GA)$.
Hence every language accepted by a finite state automaton is
a language accepted by a deterministic finite state
automaton.
\end{prop}
%%
The proof is straightforward and left as an exercise. Now we
shall first show that a regular language is a language
accepted by a finite state automaton. We may assume that
$G$ is (almost) strictly binary, as we have seen above. So, let
$G = \auf \mbox{\tt S}, N, A, R\zu$. We put $Q_G := N$, $i_0 := 
\mbox{\tt S}$, $F_G := \{X : X \pf \varepsilon \in R\}$ as well as
%%
\begin{equation}
\delta_G(X,a) :=  \{Y : X \pf a Y \in R\} 
\end{equation}
%%
Now put $\GA_G := \auf A, Q_G, i_0, F_G, \delta_G\zu$.
%%
\begin{lem}
\label{lem:labelchar}
For all $X, Y \in N$ and $\vec{x}$ we have
$Y \in \delta(X,\vec{x})$ iff $X \Pf^{\ast}_R \vec{x} \conc Y$.
\end{lem}
%%
\proofbeg
Induction over the length of $\vec{x}$. The case $|\vec{x}| =
\varepsilon$ is evident. Let $\vec{x} = a \in A$.
Then $Y \in \delta_G(X,a)$ by definition iff
$X \pf a Y \in R$, and from this we get $X \Pf_R^{\ast} a Y$.
Conversely, from $X \Pf^{\ast}_R a Y$ follows that $X \pf a Y \in
R$. For since the derivation uses only strictly expanding rules 
except for the last step, the derivation of $a Y$ from $X$ must 
be the application of a single rule. This finishes the case of 
length 1. Now let $\vec{x} = \vec{y} \conc a$. By definition of 
$\delta_G$ we have
%%
\begin{equation}
\delta_G(X, \vec{x}) = \delta_G(\delta_G(X,\vec{y}),a) 
\end{equation}
%%
Hence there is a $Z$ such that $Z \in \delta_G(X, \vec{y})$ and
$Y \in \delta_G(Z,a)$. By induction hypothesis this is equivalent
with $X \Pf_R^{\ast} \vec{y}\conc  Z$ and $Z \Pf^{\ast}_R a Y$. 
From this we get $X \Pf^{\ast}_R \vec{y}\conc  a \conc Y = \vec{x}\conc Y$.
Conversely, from $X \Pf^{\ast}_R \vec{x}\conc Y$ we get 
$X \Pf^{\ast}_R \vec{y} \conc Z$ and $Z \Pf^{\ast}_R a Y$ for some $Z$,
since $G$ is regular. Now, by induction hypothesis,
$Z \in \delta_G(X, \vec{y})$ and $Y \in \delta_G(Z,a)$, and
so $Y \in \delta_G(\vec{x},X)$.
\proofend
%%
\begin{prop}
$L(\GA_G) = L(G)$.
\end{prop}
%%
\proofbeg
It is easy to see that
$L(G) = \{\vec{x} : G \vdash \vec{x}\conc Y, Y \pf \varepsilon \in R\}$.
By Lemma~\ref{lem:labelchar} $\vec{x} \conc Y \in L(G)$ iff
$S \Pf^{\ast}_R \vec{x} \conc Y$. The latter is equivalent with
$Y \in \delta_G(\mbox{\tt S}, \vec{x})$. And this is nothing but
$\vec{x} \in L(\GA_G)$. Hence $L(G) = L(\GA_G)$.
\proofend

Given a finite state automaton $\GA = \auf A, Q, i_0, F, \delta\zu$ 
put $N_{\GA} := Q$, $S_{\GA} := i_0$. $R_{\GA}$ consists of all 
rules of the form $X \pf a Y$ where $Y \in \delta(X,a)$ as well 
as all rules of the form $X \pf \varepsilon$ for $X \in F$. Finally, 
$G_{\GA} := \auf S_{\GA}, N_{\GA}, A, R_{\GA}\zu$. 
$G_{\GA}$ is strictly binary and $\GA_{G_{\GA}} = \GA$. 
Therefore we have $L(G_{\GA}) = L(\GA)$.
%%
\begin{thm}
The regular languages are exactly those languages that are
accepted by some deterministic finite state automaton.
\proofend
\end{thm}
%%
\newcommand{\nll}{0}
%%
Now we shall turn to a further characterization of regular
languages.
%%%
\index{term!regular}%%
%%%
A \textbf{regular term over} $A$ is a term which is composed from $A$
with the help of the symbols $\nll$ (0--ary), $\varepsilon$
(0--ary), $\cdot$ (binary), $\cup$ (binary) and $^{\ast}$ (unary).
A regular term defines a language over $A$ as follows.
%%
\index{$\cdot$, $\cup$, $^{\ast}$}%%%
\begin{subequations}
\begin{align}
L(\nll)        & := \varnothing \\
L(\varepsilon)  & := \{\varepsilon\} \\
L(a)         & := \{a\} \\
L(R \cdot S) & := L(R) \cdot L(S) \\
L(R \cup S)  & := L(R) \cup L(S) \\
L(R^{\ast})  & := L(R)^{\ast}
\end{align}
\end{subequations}
%%
(Commonly, one writes $R$ in place of $L(R)$, a usage that we will
follow in the sequel to this section.) Also, 
%%%
\index{$^+$}%%%
%%%
$R^+ := R^{\ast} \cdot R$ is an often used abbreviation.
Languages which are defined by a regular term can also be viewed
as solutions of some very simple systems of equations. We introduce
variables (say $X$, $Y$ and $Z$) which are variables for subsets
of $A^{\ast}$ and we write down equations for the terms over
these variables and the symbols $\nll$, $\varepsilon$,
$a$ ($a \in A$), $\cdot$, $\cup$ and $^{\ast}$. An example is the
equation $X = \mbox{\tt b} \cup \mbox{\tt a} X$, whose solution is
$X = \mbox{\tt a}^{\ast}\mbox{\tt b}$.
%%
\begin{lem}
Assume $R \neq 0$ and $\varepsilon \not\in L(R)$.
Then $R^{\ast}$ is the unique solution of
$X = \varepsilon \cup R \cdot X$.
\end{lem}
%%
\proofbeg
The proof is by
induction over the length of $\vec{x}$. $\vec{x} \in X$ means by
definition that $\vec{x} \in \varepsilon \cup R \cdot X$.
If $\vec{x} = \varepsilon$ then $\vec{x} \in R^{\ast}$.
Hence let $\vec{x} \neq \varepsilon$; then $\vec{x} \in R
\cdot X$ and so it is of the form $\vec{u}_0 \conc \vec{x}_0$
where $\vec{u}_0 \in R$ and $\vec{x}_0 \in X$.  Since
$\vec{u}_0 \neq \varepsilon$, $\vec{x}_0$ has smaller length
than $\vec{x}$. By induction hypothesis we therefore have
$\vec{x}_0 \in R^{\ast}$. Hence $\vec{x} \in R^{\ast}$. The
other direction is as easy.
\proofend
%%
\begin{lem}
\label{lem:regsolution}
Let $C, D$ be regular terms, $D \neq 0$ and $\varepsilon \not\in L(D)$.
The equation 
%%
\begin{equation}
X = C \cup D \cdot X
\end{equation}
%%%
has exactly one solution, namely $X = D^{\ast} \cdot C$.
\proofend
\end{lem}
%%
We shall now show that regular languages can be seen as solutions
of systems of equations. A general system of string equations is a 
set of equations of the form $X_j = Q \cup \bigcup_{i < m} T^i$ where 
$Q$ is a regular term and the $T^i$ have the form $R \cdot X_k$ where
$R$ is a regular term. Here is an example.
%%
\begin{equation}
\begin{array}{lr@{\; \cup\; }r}
X_0 & = \mbox{\tt a}^{\ast} & \mbox{\tt c} \cdot
    \mbox{\tt a} \cdot \mbox{\tt b} \cdot X_1   \\
X_1 & = \mbox{\tt c}        & \mbox{\tt c}
    \cdot \mbox{\tt b}^3 \cdot X_0
\end{array}
\end{equation}
%%
Notice that like in other systems of equations a variable
need not occur to the right in every equation. Moreover,
a system of equations contains any given variable only once
on the left. The system is called \textbf{proper} if for all
$i$ and $j$ we have $\varepsilon \not\in L(T^i_j)$. We shall call
a system of equations \textbf{simple} if it is proper and $Q$ as well
as the $T^i_j$ consist only of terms made from elements of
$A$ using $\varepsilon$ and $\cup$.
%%%
\index{system of equations!proper}%%
\index{system of equations!simple}%%
%%%
The system displayed above is proper but not simple.

Let now $\auf \mbox{\tt S}, N, A, R\zu$ be a strictly binary
regular grammar. Introduce for each nonterminal
$X$ a variable $Q_X$. This variable $Q_X$ shall stand for the
set of all strings which can be generated from $X$ in this
grammar, that is, all strings $\vec{x}$ for which
$X \Pf^{\ast}_R \vec{x}$. This latter set we denote by
$[X]$. We claim that the $Q_X$ so interpreted satisfy the
following system of equations.
%%
\begin{equation}
\begin{split}
Q_Y = & \phantom{\mbox{}\cup\mbox{}} 
	\bigcup \{\varepsilon : Y \pf \varepsilon \in R \} \\
    & \cup \bigcup \{a \cdot Q_X : Y \pf aX \in R\}
\end{split}
\end{equation}
%%
This system of equations is simple. We show $Q_Y = [Y]$ for all $Y
\in N$. The proof is by induction over the length of the string.
To begin, we show that $Q_Y \subseteq [Y]$. For let $\vec{y} \in
Q_Y$. Then either $\vec{y} = \varepsilon$ and $Y \pf \varepsilon
\in R$ or we have $\vec{y} = a \conc \vec{x}$ with $\vec{x} \in
Q_X$ and $Y \pf a \conc \vec{x} \in R$. In the first case $Y \pf
\varepsilon \in R$, whence $\varepsilon \in [Y]$. In the second
case $|\vec{x}| < |\vec{y}|$ and so by induction hypothesis
$\vec{x} \in [X]$, hence $X \Pf^{\ast}_R \vec{x}$. Then we have 
$Y \Pf^{\ast}_R a \conc \vec{x} = \vec{y}$, from which $\vec{y} 
\in [Y]$. This shows the first inclusion. Now we show that 
$[Y] \subseteq Q_Y$. To this end let $Y \Pf^{\ast}_R \vec{y}$. 
Then either $\vec{y} = \varepsilon$ and so $Y \pf \varepsilon 
\in R$ or $\vec{y} = a \conc \vec{x}$ for some $\vec{x}$.  In 
the first case $\vec{y} \in Q_Y$, by definition. In the second 
case there must be an $X$ such that $Y \pf a X \in R$ and 
$X \Pf^{\ast}_R \vec{x}$. Then $|\vec{x}| < |\vec{y}|$ and 
therefore by induction hypothesis $\vec{x} \in Q_X$. Finally, 
by definition of $Q_Y$, $\vec{y} \in Q_Y$, which had to be shown.

So, a regular language is the solution of a simple system of
equations. Conversely, every simple system of equations can be
rewritten into a regular grammar which generates the solution
of this system. Finally, it remains to be shown that regular terms
describe nothing but regular languages. What we shall establish is
more general and derives the desired conclusion. We shall show
that every proper system of equations which has as many
equations as it has variables has as its solution for each
variable a regular language. To this end, let such a system
$X_j = \bigcup_{i < m_j} T_j^i$ be given. We begin by eliminating
$X_0$ from the system of equations. We distinguish two cases.
(1) $X_0$ appears in the equation $X_0 = \bigcup_{i < m_0}
T^i_j$ only to the left. This equation is fixed, and called the
\textbf{pivot equation for} $X_0$. Then we can replace $X_0$
in the other equations by $\bigcup_{i < m_0} T^i_j$. 
(2) The equation is of the form $X_0 = C \cup D \cdot X_0$, 
$C$ a regular term, which does not contain $X_0$, $D$
free of variables and $\varepsilon \not\in L(D)$. Then
$X_0 = D^{\ast} \cdot C$ by Lemma~\ref{lem:regsolution}.
Now $X_0$ does not occur and we can replace $X_0$ in the other
equations as in (1). The system of equations
that we get is not simple, even if it was simple at the beginning.
We can proceed in this fashion and eliminate step by step the
variables from the right hand side (and putting aside the
corresponding pivot equations) until we reach the last equation.
The solution for $X_{n -1}$ does not contain any variables at all
and is a regular term. The solution can be inserted into the
other equations, and then we continue with $X_{n-2}$, then
with $X_{n-3}$, and so on.
%%
As an example, we take the following system of equations.
%%
$$\begin{array}{ll@{\quad = \quad}rrr}
\mbox{(I)} & X_0 & \mbox{\tt a} \cup\; \mbox{\tt a}\cdot X_0
    & \cup\; \mbox{\tt b} \cdot X_1
    & \cup\; \mbox{\tt c} \cdot X_2 \\
    & X_1 &   \mbox{\tt c} \cdot X_0 &
    & \cup \; \mbox{\tt a} \cdot X_2 \\
    & X_2 & \mbox{\tt b} \cup\; \mbox{\tt a} \cdot X_0
    & \cup \; \mbox{\tt b} \cdot X_1
    & \\
    \multicolumn{5}{c}{} \\
\mbox{(II)} & X_0 & \mbox{\tt a}^+ & \cup \; \mbox{\tt a}^{\ast}
    \mbox{\tt b} \cdot X_1
    & \cup \; \mbox{\tt a}^{\ast}\mbox{\tt c} \cdot X_2 \\
     & X_1 & \mbox{\tt ca}^+  & \cup \; \mbox{\tt ca}^{\ast}%
    \mbox{\tt b} \cdot X_1
    & \cup \; (\mbox{\tt ca}^{\ast}\mbox{\tt c} \cup %
    \mbox{\tt a}) \cdot X_2 \\
     & X_2 & \mbox{\tt b} \cup \mbox{\tt aa}^+ & \cup\;
     (\mbox{\tt a}^+ \mbox{\tt b} \cup \mbox{\tt b})
    \cdot X_1 & \cup \; \mbox{\tt a}^{\ast} \mbox{\tt c}
    \cdot X_2 \\
    \multicolumn{5}{c}{} \\
\mbox{(III)} & X_1 & 
    \multicolumn{3}{r}{%
	(\mbox{\tt ca}^{\ast}\mbox{\tt b})^{\ast} \mbox{\tt ca}^+ 
    \cup (\mbox{\tt ca}^{\ast}\mbox{\tt b})^{\ast}%
    (\mbox{\tt ca}^{\ast}\mbox{\tt c} \cup \mbox{\tt a})
    \cdot X_2} \\
       & X_2 &  
    \multicolumn{3}{r}{%
       (\mbox{\tt b} \cup \mbox{\tt aa}^+) 
    \cup [\mbox{\tt a}^{\ast}\mbox{\tt b}
    (\mbox{\tt ca}^{\ast}\mbox{\tt b})^{\ast}
    (\mbox{\tt ca}^{\ast}\mbox{\tt c} \cup \mbox{\tt a}) %
    \cup \mbox{\tt a}^{\ast}\mbox{\tt c}] \cdot X_2} \\
    \multicolumn{5}{c}{} \\
\mbox{(IV)} & X_2 & \multicolumn{3}{l}{%
    [\mbox{\tt a}^{\ast}\mbox{\tt b} 
    (\mbox{\tt ca}^{\ast}\mbox{\tt b})^{\ast}
    (\mbox{\tt ca}^{\ast}\mbox{\tt c} \cup \mbox{\tt a}) %
    \cup \mbox{\tt a}^{\ast}\mbox{\tt c}]^{\ast} %
    (\mbox{\tt b} \cup \mbox{\tt aa}^+)}
    \end{array}$$
%%
Now that $X_2$ is known, $X_1$ can be determined by inserting the 
regular term for $X_2$, and, finally, $X_0$ is obtained by inserting 
the values for $X_2$ and $X_1$.
%%
\nocite{kleene:regular}
\index{Kleene, Stephen C.}%%
\begin{thm}[Kleene]
\label{thm:reg}
Let $L$ be a language over $A$. Then the following are equivalent:
%%
\begin{dingautolist}{192}
\item
$L$ is regular.
\item
$L = L(\GA)$ for a finite, deterministic automaton
$\GA$ over $A$.
\item
$L = L(R)$ for some regular term $R$ over $A$.
\item
$L$ is the solution for $X_0$ of a simple system of equations
over $A$ with variables $X_i$, $i < m$.
\end{dingautolist}
%%
Further, there exist algorithms which (i) for a given automaton
$\GA$ compute a regular term $R$ such that $L(\GA) = L(R)$; (ii)
for a given regular term $R$ compute a simple system of equations
$\Sigma$ over $\vec{X}$ whose solution for a given variable $X_0$
is exactly $L(R)$; and (iii) which for a given simple system
of equations $\Sigma$ over $\{X_i : i < m\}$ compute an automaton 
$\GA$ such that $\vec{X}$ is its set of states and the solution for
$X_i$ is exactly the set of strings which send the automaton from
state $X_0$ into $X_i$. \proofend
\end{thm}
%%
This is the most important theorem for regular languages.
We shall derive a few consequences. Notice we can turn a 
finite state automaton $\GA$ into a Turing machine $T$ 
accepting the same language in linear time and no additional 
space.  Therefore, the recognition problem for regular languages 
is in $\textbf{DTIME}(n)$ and in $\textbf{DSPACE}(n)$. This 
also applies to the parsing problem, as is easily seen.
%%
\begin{cor}
The recognition and the parsing problem are in
$\textbf{DTIME}(n)$ and $\textbf{DSPACE}(n)$.
\end{cor}
%%
\begin{cor}
\label{cor:con} The set of regular languages over $A$ is closed
under intersection and relative complement. Further, for given
regular terms $R$ and $S$ one can determine terms $U$ and $V$ such
that $L(U) = A^{\ast} - L(R)$ and $L(V) = L(R) \cap L(S)$.
\end{cor}
%%
\proofbeg
It is enough to do this construction for automata.  Using
Theorem~\ref{thm:reg} it follows that we can do it also for
the corresponding regular terms. Let $\GA = \auf A, Q, i_0, F,
\delta\zu$. Without loss of generality we may assume that
$\GA$ is deterministic. Then let $\GA^- := \auf A, Q, i_0, Q - F,
\delta\zu$. We then have $L(\GA^-) = A^{\ast} - L(\GA)$.
This shows that for given $\GA$ we can construct an automaton
which accepts the complement of $L(\GA)$.
Now let $\GA' = \auf A, Q', i_0', F', \delta'\zu$.
Put 
%%
\begin{equation}
\GA  \times \GA' := \auf A, Q\times Q', \auf i_0, i_0'\zu,
F \times F', \delta \times \delta'\zu
\end{equation}
%%%
where
%%%
\begin{equation}
(\delta \times \delta')(\auf q,q'\zu, a) := \{\auf
r,r'\zu : r \in \delta(q,a), r' \in \delta'(q',a)\} 
\end{equation}
%%
It is easy to show that $L(\GA \times \GA') =
L(\GA) \cap L(\GA')$.
\proofend
%%

\noindent
The proof of the next theorem is an exercise.
%%
\begin{thm}
\label{thm:abschluss}
Let $L$ and $M$ be regular languages. Then so are $L/M$ and
$M\backslash L$. Moreover, $L^T$, $L^P := L/A^{\ast}$
as well as $L^S := A^{\ast}\backslash L$ are regular.
\end{thm}
%%
Furthermore, the following important consequence can be
established.
%%
\begin{thm}
Let $\GA$ and $\GB$ be finite state automata.
Then it is decidable whether $L(\GA) = L(\GB)$.
\end{thm}
%%
\proofbeg
Let $\GA$ and $\GB$ be given. By Theorem~\ref{thm:reg}
we can compute a regular term $R$ with $L(R) = L(\GA)$
as well as a regular term $S$ with $L(S) = L(\GB)$.
Then $L(\GA) = L(\GB)$ iff $L(R) = L(S)$
iff $(L(R) - L(S)) \cup (L(S) - L(R))
= \varnothing$. By Corollary~\ref{cor:con} we can compute
a regular term $U$ such that 
%%
\begin{equation}
L(U) = (L(R) - L(S)) \cup (L(S) - L(R))
\end{equation}
%%%
Hence $L(\GA) = L(\GB)$ iff $L(U) = \varnothing$. This is 
decidable by Lemma~\ref{lem:leer}.
\proofend
%%
\begin{lem}
\label{lem:leer}
The problem `$L(R) = \varnothing$', where $R$ is a regular 
term, is decidable.
\end{lem}
%%
\proofbeg
By induction on $R$. If $R = \varepsilon$
or $R = a$ then $L(R) \neq  \varnothing$. If $R = \nll$
then by definition $L(R) = \varnothing$. Now assume that the
problems `$L(R) = \varnothing$' and `$L(S) = \varnothing$'
are decidable. Notice that (a) $L(R \cup S) = \varnothing$ iff
$L(R) = \varnothing$ and $L(S) = \varnothing$, 
(b) $L(R \cdot S) = \varnothing$ iff $L(R) = \varnothing$ or 
$L(S) = \varnothing$ and (c) $L(R^{\ast}) = \varnothing$ iff
$L(R) = \varnothing$. All three problems are decidable.
\proofend
%%

We conclude with the following theorem, which we have used
already in Section~\ref{kap1}.\ref{kap1-5}.
%%
\begin{thm}
\label{thm:cfintersekt} Let $L$ be context free and $R$
regular. Then $L \cap R$ is context free.
\end{thm}
%%
\proofbeg
Let be $G = \auf S, N, A, R\zu$ be a CFG
with $L(G) = L$ and $\GA = \auf n, 0, F, \delta\zu$
a deterministic automaton consisting of $n$  states such that
$L(\GA) = R$. We may assume that rules of $G$ are of the form
$X \pf a$ or $X \pf \vec{Y}$. We define new nonterminals, which
are all of the form $^i X^j$, where $i, j < n$ and $X \in N$. The
interpretation is as follows. $X$ stands for the set of all
strings $\vec{\alpha} \in A^{\ast}$ such that $X \vdash_G
\vec{\alpha}$. $^i X^j$ stands for the set of all
$\vec{\alpha}$ such that $X \vdash_G \vec{\alpha}$ and
$\delta(i,\vec{\alpha}) = j$. We have a set of start symbols,
consisting of all $^0 S^{j}$ with $j \in F$. As we already know,
this does not increase the generative power. A rule 
$X \pf Y_0 Y_1 \dotsb Y_{k-1}$ is now replaced by the set of all
rules of the form
%%
\begin{equation}
^i X^j \pf ^i Y_0^{i_0}\conc ^{i_0}Y_1^{i_1} \conc \dotso
\conc ^{i_{k-2}}Y_{k-1}^j
\end{equation}
%%
Finally, we take all rules of the form $^i X^j \pf a$,
$\delta(i,a) = j$. This defines the grammar $G_r$. We shall show:
$\vdash_{G^r} \vec{x}$ iff $\vdash_G \vec{x}$ and
$\vec{x} \in L(\GA)$. ($\Pf$) Let $\GB$ be a $G^r$--tree with
associated string $\vec{x}$. The map ${^i X}^j \mapsto X$ turns
$\GB$ into a $G$--tree. Hence $\vec{x} \in L(G)$. Further, it
is easily shown that $\delta(0,x_0x_1\dotsb x_j) = k_j$, where
$^{k_{j-1}} X^{k_j}$ is the node dominating $x_j$.
Also, if $|\vec{x}| = n$, then $^0 S^{k_n}$ is the top node
and by construction $k_n \in F$. Hence $\delta(\vec{x}, 0)
\in F$ and so $\vec{x} \in L(\GA)$.
($\Leftarrow$) Let $\vec{x} \in L(G)$ and $\vec{x} \in L(\GA)$.
We shall show that $\vec{x} \in L(G^r)$. We take a $G$--tree
$\GB$ for $\vec{x}$.  We shall now prove that one can replace the
$G$--nonterminals in $\GB$ in such a way by $G^r$--nonterminals
that we get a $G_r$--tree. The proof is by induction on the height
of a node. We begin with nodes of height 1. Let
$\vec{x} = \prod_{i < n} x_i$; and let $X_i$ be the nonterminal
above $x_i$. Further let $\delta(0,\prod_{i < j} x_i) =
j_i$. Then $p_0 = 0$ and $p_n \in F$. We replace $X_i$ by
$^{p_i} X^{p_{i+1}}$. We say that two nodes $x$ and
$y$ \textbf{connect} if they are adjacent and for the labels
${^i X^j}$ of $x$ and $^k Y^{\ell}$ of $y$ we have $j = k$.
Let $x$ be a node of height $n+1$ with label $X$ and let $x$ be
mother of the nodes with labels $Y_0 Y_1 \dotsb Y_{n-1}$ in $G$.
We assume that below $x$ all nodes carry labels from $G^r$ in such
a way that adjacent nodes connect. Then there exists  a rule
in $G_r$ such that $X$ can be labelled with superscripts,
the left hand superscript of $Y_0$ to its left and the right hand
superscript of $Y_{n-1}$ to its right. All adjacent nodes of height
$n+1$ connect, as is easily seen. Further, the leftmost node
carries the left superscript 0, the rightmost node carries a
right superscript $p_n$, which is an accepting state. Eventually,
the root has superscripts as well. It carries the label
$^0 S^{p_n}$, and so we have a $G_r$--tree.
\proofend
%%
\vplatz
\exercise
Prove Theorem~\ref{thm:abschluss}.
%%
\vplatz
\exercise
Show that a language is regular iff it can be
generated by a grammar with rules of the form $X \pf Y$,
$X \pf Ya$, $X \pf a$ and $X \pf \varepsilon$. Such a
grammar is called
%%%
\index{grammar!left regular}%%
\index{grammar!right regular}%%
%%%
\textbf{left regular}, in contrast to the grammars of
Type 3, which we also call \textbf{right regular}.
Show also that it is allowed to add rules of the form
$X \pf \vec{x}$ and $X \pf Y \vec{x}$.
%%
\vplatz
\exercise
Show that there is a grammar with rules of the form
$X \pf a$, $X \pf aY$ and $X \pf Ya$ which generates a
nonregular language. This means that a Type 3 grammar may 
contain (in general) only left regular rules or only right 
regular rules, but not both.
%%
\vplatz
\exercise
Show that if $L$ and $M$ are regular, then so are
$L/M$ and $M\backslash L$.
%%
\vplatz
\exercise
Let $L$ be a language over $A$. Define an equivalence relation
$\sim_S$ over $A^{\ast}$ as follows.  $\vec{x} \sim_S \vec{y}$ iff
for all $\vec{z} \in A^{\ast}$ we have
$\vec{x} \conc \vec{z} \in L \Dpf \vec{y} \conc
\vec{z} \in L$. $L$ is said to have \textbf{finite index}
%%%%
\index{language!finite index}%%
%%%%
if there are only finitely many equivalence classes
with respect to $\sim_S$. Show that $L$ is regular iff
it has finite index.
%%
\vplatz
\exercise
\label{ue:index}
Show that the language $\{\mbox{\tt a}^n\mbox{\tt b}^n :
n \in \omega\}$ does not have finite index. Hence it is
not regular.
%%
\vplatz
\exercise
Show that the intersection of a context sensitive language with
a regular language is again context sensitive.
%%%
\vplatz
\exercise
Show that $L$ is regular iff it is accepted by a read
only 1--tape Turing machine.
