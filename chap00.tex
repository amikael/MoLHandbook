\preface %\chapter*{Preface}
\addcontentsline{toc}{chapter}{Preface}
%\thispagestyle{empty}
%\section*{Preface}

Mathematical linguistics is rooted both in Euclid's (circa 325--265 BCE)
axiomatic method and in P\={a}\d{n}ini's (circa 520--460 BCE) method of
grammatical description. To be sure, both Euclid and P\={a}\d{n}ini built upon
a considerable body of knowledge amassed by their precursors, but the
systematicity, thoroughness, and sheer scope of the {\it Elements} and the
{\it Ash\d{t}\={a}dhy\={a}y\={\i}} would place them among the greatest
landmarks of all intellectual history even if we disregarded the key
methodological advance they made. \index{Euclid} \index{P\={a}\d{n}ini}
\index{axiomatic method}

As we shall see, the two methods are fundamentally very similar: the axiomatic
method starts with a set of statements assumed to be true and transfers truth
from the axioms to other statements by means of a fixed set of logical rules,
while the method of grammar is to start with a set of expressions assumed to
be grammatical both in form and meaning and to transfer grammaticality to
other expressions by means of a fixed set of grammatical rules. 

Perhaps because our subject matter has attracted the efforts of some of the
most powerful minds (of whom we single out A. A. Markov here) from antiquity
to\index{Markov} the present day, there is no single easily accessible introductory text in
mathematical linguistics. Indeed, to the mathematician the whole field of
linguistics may appear to be hopelessly mired in controversy, and neither the
formidable body of empirical knowledge about languages nor the standards of
linguistic argumentation offer an easy entry point. 

Those with a more postmodern bent may even go as far as to doubt the existence
of a solid core of mathematical knowledge, often pointing at the false
theorems and incomplete or downright wrong proofs that slip through the peer
review process at a perhaps alarming rate. Rather than attempting to drown
such doubts in rivers of philosophical ink, the present volume will simply
proceed {\it more geometrico} in exhibiting this solid core of knowledge. In
Chapters~3--6, a mathematical overview of the traditional main branches of
linguistics, phonology, morphology, syntax, and semantics, is presented. 

\subsection*{Who should read this book?} 

The book is accessible to anyone with sufficient general mathematical maturity
(graduate or advanced undergraduate). No prior knowledge of linguistics or
languages is assumed on the part of the reader. The book offers a single entry
point to the central methods and concepts of linguistics that are made largely
inaccessible to the mathematician, computer scientist, or engineer by the
surprisingly adversarial style of argumentation (see Section~1.2), the
apparent lack of adequate definitions (see Section~1.3), and the proliferation
of unmotivated notation and formalism (see Section~1.4) all too often
encountered in research papers and monographs in the humanities. Those
interested in linguistics can learn a great deal more about the subject here
than what is covered in introductory courses just from reading through the
book and consulting the references cited. Those who plan to approach
linguistics through this book should be warned in advance that many branches
of linguistics, in particular psycholinguistics, child language acquisition,
and the study of language pathology, are largely ignored here -- not because
they are viewed as inferior to other branches but simply because they do not
offer enough grist for the mathematician's mill. Much of what the
linguistically naive reader may find interesting about language turns out to
be more pertinent to cognitive science, the philosophy of language, and
sociolinguistics, than to linguistics proper, and the Introduction gives these
issues the shortest possible shrift, discussing them only to the extent
necessary for disentangling mathematical linguistics from other concerns.

Conversely, issues that linguists sometimes view as peripheral to their
enterprise will get more discussion here simply because they offer such a rich
variety of mathematical techniques and problems that no book on mathematical
linguistics that ignored them could be considered complete.  After a brief
review of information theory in Chapter~7, we will devote Chapters~8 and 9 to
phonetics, speech recognition, the recognition of handwriting and machine
print, and in general to issues of linguistic signal processing and pattern
matching, including information extraction, information retrieval, and
statistical natural language processing. Our treatment assumes a bit more
mathematical maturity than the excellent textbooks by Jelinek (1997)
\nocite{Jelinek:1997} and Manning and Sch\"{u}tze (1999) \nocite{Manning:1999}
and intends to complement them.  Kracht (2003) conveniently summarizes and
extends much of the discrete (algebraic and combinatorial) work on
mathematical linguistics.  It is only because of the timely appearance of this
excellent reference work that the first six chapters could be kept to a
manageable size and we could devote more space to the continuous (analytic and
probabilistic) aspects of the subject. In particular, expository simplicity
would often dictate that we keep the underlying parameter space discrete, but
in the later chapters we will be concentrating more on the case of continuous
parameters, and discuss the issue of quantization losses explicitly. 

In the early days of computers, there was a great deal of overlap between the
concerns of mathematical linguistics and computer science, and a surprising
amount of work that began in one field ended up in the other, sometimes
explicitly as part of computational linguistics, but often as general theory
with its roots in linguistics largely forgotten. In particular, the basic
techniques of syntactic analysis are now firmly embedded in the computer
science curriculum, and the student can already choose from a large variety of
textbooks that cover parsing, automata, and formal language theory. Here we
single out the classic monograph by \nocite{Kracht:2003} \nocite{Salomaa:1973}
Salomaa (1973), which shows the connection to formal syntax in a way readily
accessible to the mathematically minded reader. We will selectively cover only
those aspects of this field that address specifically linguistic concerns, and
again our guiding principle will be mathematical content, as opposed to
algorithmic detail. Readers interested in the algorithms should consult the
many excellent natural language processing textbooks now available, of which
we single out Jurafsky and Martin (2000, with a new edition planned in
2008). \nocite{Jurafsky:2000}

\subsection*{How is the book organized?} 

To the extent feasible we follow the structure of the standard introductory
courses to linguistics, but the emphasis will often be on points only covered
in more advanced courses.  The book contains many exercises. These are, for
the most part, rather hard (over level 30 in the system of Knuth 1971)
\nocite{Knuth:1971} but extremely rewarding.  Especially in the later
chapters, the exercises are often based on classical and still widely cited
theorems, so the solutions can usually be found on the web quite easily simply
by consulting the references cited in the text. However, readers are strongly
advised not to follow this route before spending at least a few days attacking
the problem.  Unsolved problems presented as exercises are marked by an
asterisk, a symbol that we also use when presenting examples and
counterexamples that native speakers would generally consider wrong
(ungrammatical): {\it Scorsese is a great director} is a positive
(grammatical) example while {\it *Scorsese a great director is} is a negative
(ungrammatical) example.  \index{grammaticality} \index{ungrammaticality} Some
exercises, marked by a dagger $^\dagger$, require the ability to manipulate
sizeable data sets, but no in-depth knowledge of programming, data structures,
or algorithms is presumed. Readers who write code effortlessly will find these
exercises easy, as they rarely require more than a few simple scripts. Those
who find such exercises problematic can omit them entirely. They may fail to
gain direct appreciation of some empirical properties of language that drive
much of the research in mathematical linguistics, but the research itself
remains perfectly understandable even if the motivation is taken on faith. A
few exercises are marked by a raised $M$ -- these are major research projects
the reader is not expected to see to completion, but spending a few days on
them is still valuable. 

Because from time to time it will be necessary to give examples from languages
that are unlikely to be familiar to the average undergraduate or graduate
student of mathematics, we decided, somewhat arbitrarily, to split languages
into two groups. {\it Major} languages are those that have a chapter in
Comrie's (1990) {\it The World's Major Languages} \nocite{Comrie:1990} --
these will be familiar to most people and are left unspecified in the text.
{\it Minor} languages usually require some documentation, both because
language names are subject to a great deal of spelling variation and because
different groups of people may use very different names for one and the same
language. Minor languages are therefore identified here by their three-letter
Ethnologue code (15th edition, 2005) given in square brackets [].
\nocite{Gordon:2005}

Each chapter ends with a section on further reading.  We have endeavored to
make the central ideas of linguistics accessible to those new to the field,
but the discussion offered in the book is often skeletal, and readers are
urged to probe further. Generally, we recommend those papers and books that
presented the idea for the first time, not just to give proper credit but
also because these often provide perspective and insight that later
discussions take for granted. Readers who industriously follow the
recommendations made here should do so for the benefit of learning the basic
vocabulary of the field rather than in the belief that such reading will
immediately place them at the forefront of research. 

The best way to read this book is to start at the beginning and to progress
linearly to the end, but the reader who is interested only in a particular
area should not find it too hard to jump in at the start of any chapter. To
facilitate skimming and alternative reading plans, a generous amount of
forward and backward pointers are provided -- in a hypertext edition these
would be replaced by clickable links. The material is suitable for an
aggressively paced one-semester course or a more leisurely paced two-semester
course. 

\subsection*{Acknowledgments}

Many typos and stylistic infelicities were caught, and excellent references
were suggested, by M\'{ar}ton Makrai (Budapest Institute of Technology),
D\'aniel Marg\'ocsy (Harvard), Doug Merritt (San Jose), Reinhard Muskens
(Tilburg), Almerindo Ojeda (University of California, Davis), B\'alint Sass
(Budapest), Madeleine Thompson (University of Toronto), and Gabriel Wyler (San
Jose). The painstaking work of the Springer editors and proofreaders, 
Catherine Brett, Frank Ganz, Hal Henglein, and Jeffrey Taub, is gratefully 
acknowledged. 

The comments of Tibor Beke (University of Massachusetts, Lowell), Michael
Buka\-tin (MetaCarta), Anssi Yli-Jyr\"{a} (University of Helsinki), P\'eter
G\'acs (Boston University), Marcus Kracht (UCLA), Andr\'{a}s Ser\'{e}ny (CEU),
P\'{e}ter Sipt\'{a}r (Hungarian Academy of Sciences), Anna Sza\-bolcsi (NYU),
P\'{e}ter V\'{a}mos (Budapest Institute of Technology), K\'{a}roly Varasdi
(Hungarian Academy of Sciences), and D\'aniel Varga (Budapest Institute of
Technology) resulted in substantive improvements.

Writing this book would not have been possible without the generous support of
MetaCarta Inc. (Cambridge, MA), the MOKK Media Research center at the Budapest
Institute of Technology Department of Sociology, the Farkas Heller Foundation,
and the Hungarian Telekom Foundation for Higher Education -- their help and
care is gratefully acknowledged. 

\endinput


